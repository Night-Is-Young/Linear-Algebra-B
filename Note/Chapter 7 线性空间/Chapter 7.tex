\documentclass{ctexart}
\usepackage{Note}
\begin{document}
\section{线性空间}
\subsection{线性空间的结构}
\begin{definition}[线性空间]
    设$V$是非空集合, $\K$是一个数域.在$V$上定义代数运算$(\bs\alpha,\bs\beta)\mapsto\bs\gamma$称作\tbf{加法},记作$\bs\alpha+\bs\beta=\bs\gamma$.在$\K$与$V$之间定义一种运算,即$\K\times V$到$V$的映射$(k,\bs\alpha)\mapsto\bs\delta$称作\tbf{数量乘法},记作$\bs\delta=k\bs\alpha$.如果上述定义的加法和数量乘法满足下述运算规则:对任意$\bs\alpha,\bs\beta,\bs\gamma\in V$和任意$k,l\in\K$都有
    \begin{enumerate}[label=\tbf{\roman*.}]
        \item $\bs\alpha+\bs\beta=\bs\beta+\bs\alpha$.
        \item $(\bs\alpha+\bs\beta)+\bs\gamma=\bs\alpha+(\bs\beta+\bs\gamma)$.
        \item 存在$\mbf0\in V$使得
        \[\bs\alpha+\mbf0=\bs\alpha,\quad\forall\bs\alpha\in V\]
        具有此性质的元素$\mbf0$称作$V$的\tbf{零元}.
        \item 对于任意$\bs\alpha\in V$总存在$\bs\beta\in V$使得
        \[\bs\alpha+\bs\beta=\mbf0\]
        具有此性质的元素$\bs\beta$称作$\bs\alpha$的\tbf{负元}.
        \item $1\bs\alpha=\bs\alpha$.
        \item $(kl)\bs\alpha=k(l\bs\alpha)$.
        \item $(k+l)\bs\alpha=k\bs\alpha+l\bs\alpha$.
        \item $k(\bs\alpha+\bs\beta)=k\bs\alpha+K\bs\beta$.
    \end{enumerate}
    则称$V$是数域$\K$上的一个线性空间.
\end{definition}
\subsection{子空间}
\begin{definition}[子空间]
    数域$\K$上的一个线性空间$V$的子集$U$如果对于$V$的加法和数量乘法也形成$\K$上的线性空间,则称$U$是$V$的一个\tbf{线性子空间},简称\tbf{子空间}.
\end{definition}
\begin{theorem}
    数域$\K$上的一个线性空间$V$的子集$U$是$V$的子空间当且仅当$U$对$V$的加法和数量乘法都封闭,即
    \begin{enumerate}[label=\tbf{\roman*.}]
        \item $\vec{u}+\vec{v}\in U,\quad\forall\vec{u},\vec{v}\in U$.
        \item $k\vec{u}\in U,\quad\forall k\in\K,\vec{u}\in U$.
    \end{enumerate}
\end{theorem}
\begin{theorem}
    设$U$是数域$\K$上线性空间$V$的子空间,则
    \[\dim U\leq\dim V\]
    并且等号成立当且仅当$U=V$.
\end{theorem}
\begin{theorem}
    设$U$是数域$\K$上线性空间$V$的子空间,则$U$的一组基可以扩充为$V$的一组基.
\end{theorem}
\begin{theorem}
    如果$V_1,V_2$都是数域$\K$上线性空间$V$的子空间,则$V_1\cap V_2$也是$V$的子空间.
\end{theorem}
\begin{theorem}
    如果$V_1,V_2$都是数域$\K$上线性空间$V$的子空间,则$V_1+V_2=\{\bs\alpha_1+\bs\alpha_2:\bs\alpha_1\in V_1,\bs\alpha_2\in V_2\}$也是$V$的子空间.称$V_1+V_2$是$V_1$与$V_2$的\tbf{和}.
\end{theorem}
\begin{theorem}[线性代数基本定理]
    如果$V_1,V_2$都是数域$\K$上线性空间$V$的子空间,则
    \[\dim V_1+\dim V_2=\dim(V_1+V_2)+\dim(V_1\cap V_2)\]
\end{theorem}
\begin{definition}[直和]
    如果$V_1,V_2$都是数域$\K$上线性空间$V$的子空间,并且$V_1+V_2$中的每个向量$\bs\alpha$恰能唯一的表示为
    \[\bs\alpha=\bs\alpha_1+\bs\alpha_2,\quad\bs\alpha_1\in V_1,\bs\alpha_2\in V_2\]
    则称$V_1+V_2$是\tbf{直和},记作$V_1\oplus V_2$.
\end{definition}
\begin{theorem}
    设$V_1,V_2$都是数域$\K$上线性空间$V$的子空间,下面的命题相互等价.
    \begin{enumerate}[label=\tbf{\roman*.}]
        \item $V_1+V_2$是直和.
        \item $V_1+V_2$中零向量的表法唯一.
        \item $V_1\cap V_2=\mbf0$.
        \item $\dim V_1+\dim V_2=\dim(V_1+V_2)$.
        \item $V_1$的一个基与$V_2$的一个基合起来是$V_1+V_2$的一个基.
    \end{enumerate}
\end{theorem}
上述定理对多个子空间的情形也是成立的.
\subsection{线性空间的同构}
\begin{definition}[线性空间的同构]
    设$V$和$V'$都是数域$\K$上的线性空间.如果在$V$与$V'$的元素存在双射$\sigma$使得对任意$\bs\alpha,\bs\beta\in V$,$k\in\K$总有
    \[\sigma(\bs\alpha+\bs\beta)=\sigma(\bs\alpha)+\sigma(\bs\beta)\]
    \[\sigma(k\bs\alpha)=k\sigma(\bs\alpha)\]
    则称$\sigma$是$V$到$V'$的一个\tbf{同构映射},简称\tbf{同构},此时的$V$和$V'$是\tbf{同构的},记作$V\cong V'$.
\end{definition}
\end{document}