\documentclass{ctexart}
\usepackage{note}
\begin{document}\pagestyle{empty}
\begin{center}
    \tbf{\Large 北京大学数学科学学院2024-25学年第二学期线性代数B期中试题}
\end{center}
\begin{homework}[1]
    求下列线性方程组的导出组的基础解系,方程组的特解和通解.
    \[\left\{\begin{array}{l}
        x_1+2x_2-x_3-x_4=0\\
        x_1+2x_2+x_4=4\\
        -x_1-2x_2+2x_3+4x_4=5
    \end{array}\right.\]
\end{homework}
\begin{homework}[2]
    求方程组
    \[\left\{\begin{array}{l}
        (\lambda-2)x_1+2x_2=0\\
        2x_1+(\lambda-3)x_2+2x_3=1\\
        2x_2+(\lambda-4)x_3=-3
    \end{array}\right.\]
    的解集与$\lambda$的关系.
\end{homework}
\begin{homework}[3]
    计算下面的行列式:
    \[\begin{vmatrix}
        5&7&-2&4\\
        1&1&0&-5\\
        -1&3&1&3\\
        2&-4&-1&-3
    \end{vmatrix}\]
\end{homework}
\begin{homework}[4]
    现有以下一向量组:
    \[\bs\alpha_1=\begin{bmatrix}
        2\\3\\4\\7
    \end{bmatrix},\quad\bs\alpha_2=\begin{bmatrix}
        5\\-1\\3\\2
    \end{bmatrix},\quad\bs\alpha_3=\begin{bmatrix}
        -3\\4\\1\\5
    \end{bmatrix},\quad\bs\alpha_4=\begin{bmatrix}
        0\\-1\\7\\2
    \end{bmatrix},\quad\bs\alpha_5=\begin{bmatrix}
        6\\2\\1\\5
    \end{bmatrix}\]
    \begin{enumerate}[label=\tbf{(\arabic*)},topsep=0pt,parsep=0pt,itemsep=0pt,partopsep=0pt]
        \item 证明:向量组$\bs\alpha_1,\bs\alpha_2$线性无关.
        \item 求上述向量组的所有包含$\bs\alpha_1,\bs\alpha_2$的极大线性无关组.
    \end{enumerate}
\end{homework}
\begin{homework}[5]
    现有矩阵
    \[\mat{A}=\begin{bmatrix}
        3&2&0&5&0\\
        3&-2&3&6&-1\\
        1&6&-4&-1&4\\
        2&0&1&5&-3
    \end{bmatrix}\]
    写出$\mat{A}$的一个子式,使得它为$\mat{A}$的最高阶非零子式,并证明你的结论.
\end{homework}
\begin{homework}[6]
    已知某$n$元齐次线性方程组的系数矩阵$\mat{A}$满足$\det\mat{A}=0$,其$(2,3)$元的代数余子式$A_{23}\neq0$.求该线性方程组的解集.
\end{homework}
\begin{homework}[7]
    设$n>100$,向量组$\li{\bs\alpha},n$线性无关,则向量组$\bs\alpha_1+\bs\alpha_2,\cdots,\bs\alpha_{n-1}+\bs\alpha_n,\bs\alpha_n+\bs\alpha_1$是否线性无关?证明你的结论.
\end{homework}
\begin{homework}[8]
    现有矩阵
    \[\mat{A}=\begin{bmatrix}
        3&a&-1&4\\
        1&-1&1&1\\
        5&3&b&6
    \end{bmatrix}\]
    求$a,b$使得$\rank\ \mat{A}=2$.
\end{homework}
\begin{homework}[9]
    求以下行列式的值:
    \[\begin{vmatrix}
        x_1+a_1b_1&a_1b_2&\cdots&a_1b_n\\
        a_2b_1&x_2+a_2b_2&\cdots&a_2b_n\\
        \vdots&\vdots&\ddots&\vdots\\
        a_nb_1&a_nb_2&\cdots&x_n+a_nb_n
    \end{vmatrix}\]
\end{homework}
\end{document}