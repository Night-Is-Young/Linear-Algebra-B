\documentclass{ctexart}
\usepackage{note}
\begin{document}\pagestyle{empty}
\begin{center}
    \tbf{\Large 北京大学数学科学学院2024-25学年第一学期线性代数B期末试题}
\end{center}
\begin{homework}[1(10')]
    设$V$是一个$n$维线性空间, $\mathcal{A}$和$\mathcal{B}$是$V$到$V$的线性映射,判断下列结论是否正确.
    \begin{enumerate}
        \item $\mathcal{A}$是可逆线性映射当且仅当$\mathcal{A}$的行列式不为$0$.
        \item $(\mathcal{A}+\mathcal{B})^2=\mathcal{A}^2+2\mathcal{A}\mathcal{B}+\mathcal{B}^2$.
        \item 如果$\rank(\mathcal{A})=\rank(\mathcal{B})$,那么$\rank(\mathcal{A}^2)=\rank(\mathcal{B}^2)$.
        \item $\rank(\mathcal{A})+\rank(\mathcal{B})\geq\rank(\mathcal{A}+\mathcal{B})$.
        \item $\mathcal{A},\mathcal{B}$在基$\li{\bs\alpha},n$下的矩阵分别为$\mat{A},\mat{B}$,则$\mathcal{A}\mathcal{B}$在上述基下的矩阵为$\mat{A}\mat{B}$.
    \end{enumerate}
\end{homework}
\begin{solution}
\begin{enumerate}
    \item 正确.
    \item 错误.实际上应当为
    \[(\mathcal{A}+\mathcal{B})^2=\mathcal{A}^2+\mathcal{A}\mathcal{B}+\mathcal{B}\mathcal{A}+\mathcal{B}^2\]
    \item 错误.
    \item 正确.
    \item 错误.实际上有
    \[\mathcal{A}\begin{bmatrix}
        \bs\alpha_1&\cdots&\bs\alpha_n
    \end{bmatrix}=\begin{bmatrix}
        \mathcal{A}\bs\alpha_1&\cdots&\mathcal{A}\bs\alpha_n
    \end{bmatrix}=\begin{bmatrix}
        \bs\alpha_1&\cdots&\bs\alpha_n
    \end{bmatrix}\mat{A}\]
    同理
    \[\mathcal{A}\begin{bmatrix}
        \bs\alpha_1&\cdots&\bs\alpha_n
    \end{bmatrix}=\begin{bmatrix}
        \bs\alpha_1&\cdots&\bs\alpha_n
    \end{bmatrix}\mat{B}\]
    于是
    \[\mathcal{A}\mathcal{B}\begin{bmatrix}
        \bs\alpha_1&\cdots&\bs\alpha_n
    \end{bmatrix}=\begin{bmatrix}
        \bs\alpha_1&\cdots&\bs\alpha_n
    \end{bmatrix}\mat{B}\mat{A}\]
\end{enumerate}
\end{solution}
\begin{homework}[2(10')]
    用非退化线性替换将下面的六元二次型化为标准形:
    \[f(\li x,6)=x_1x_6+x_2x_5+x_3x_4\]
\end{homework}
\begin{solution}
    首先设
    \[\begin{bmatrix}
        x_1\\x_2\\x_3\\x_4\\x_5\\x_6
    \end{bmatrix}=\begin{bmatrix}
        y_1-y_6\\y_2-y_5\\y_3-y_4\\
        y_3+y_4\\y_2+y_5\\y_1+y_6
    \end{bmatrix}\]
    则有
    \[f(\li x,6)=y_1^2+y_2^2+y_3^2-y_4^2-y_5^2-y_6^2\]
    容易看出上述线性变换是非退化的,因此做上述变换即可将题设二次型化为标准形.
\end{solution}
\begin{homework}[3(12')]
    设$\li{\bs\ep},5$为$\R^5$的标准正交基.令
    \[\bs\alpha_1=\bs\ep_1+\bs\ep_5,\quad\bs\alpha_2=\bs\ep_1-\bs\ep_2+\bs\ep_4,\quad\bs\alpha_3=2\bs\ep_1+\bs\ep_2+\bs\ep_3\]
    求与$\bs\alpha_1,\bs\alpha_2,\bs\alpha_3$等价的正交单位向量组.
\end{homework}
\begin{solution}
    根据Schmidt正交化的过程,令
    \[\bs\beta_1=\bs\alpha_1\]
    \[\bs\beta_2=\bs\alpha_2-\dfrac{\inprod{\bs\alpha_2}{\bs\beta_1}}{\inprod{\bs\beta_1}{\bs\beta_1}}\bs\beta_1=(\bs\ep_1-\bs\ep_2+\bs\ep_4)-\dfrac{1}{2}(\bs\ep_1+\bs\ep_5)=\dfrac12\bs\ep_1-\bs\ep_2+\bs\ep_4-\dfrac12\bs\ep_5\]
    \[\begin{aligned}
        \bs\beta_3
        &=\bs\alpha_3-\dfrac{\inprod{\bs\alpha_3}{\bs\beta_1}}{\inprod{\bs\beta_1}{\bs\beta_1}}\bs\beta_1-\dfrac{\inprod{\bs\alpha_3}{\bs\beta_2}}{\inprod{\bs\beta_2}{\bs\beta_2}}\bs\beta_2\\
        &=(2\bs\ep_1+\bs\ep_2+\bs\ep_3)-\dfrac{2}{2}(\bs\ep_1+\bs\ep_5)-\dfrac{0}{\frac52}\left(\dfrac12\bs\ep_1-\bs\ep_2+\bs\ep_4-\dfrac12\bs\ep_5\right)\\
        &=\bs\ep_1+\bs\ep_2+\bs\ep_3-\bs\ep_5
    \end{aligned}\]
    然后令
    \[\bs\gamma_1=\dfrac{1}{||\bs\beta_1||}\bs\beta_1=\dfrac{\sqrt2}{2}\bs\ep_1+\dfrac{\sqrt2}{2}\bs\ep_5\]
    \[\bs\gamma_2=\dfrac{1}{||\bs\beta_2||}\bs\beta_2=\dfrac{\sqrt{10}}{10}\bs\ep_1-\dfrac{\sqrt{10}}{5}\bs\ep_2+\dfrac{\sqrt{10}}{5}\bs\ep_4-\dfrac{\sqrt{10}}{10}\bs\ep_5\]
    \[\bs\gamma_3=\dfrac{1}{||\bs\beta_3||}\bs\beta_3=\dfrac12\bs\ep_1+\dfrac12\bs\ep_2+\dfrac12\bs\ep_3-\dfrac12\bs\ep_5\]
    则$\bs\gamma_1,\bs\gamma_2,\bs\gamma_3$是与$\bs\alpha_1,\bs\alpha_2,\bs\alpha_3$等价的正交单位向量组.
\end{solution}
\begin{homework}[4(16')]
    设
    \[\mat{A}=\begin{bmatrix}
        0&0&0\\
        1&0&-a\\
        0&1&a+1
    \end{bmatrix}\]
    \begin{enumerate}
        \item 求$a$的取值范围使得$\mat{A}$可对角化.
        \item 当$\mat{A}$可对角化时,求可逆矩阵$\mat{P}$使得$\mat{P}^{-1}\mat{A}\mat{P}$为对角矩阵.
    \end{enumerate}
\end{homework}
\begin{solution}
\begin{enumerate}
    \item 考虑$\mat{A}$的特征多项式:
    \[\det(\lambda\mat{I}-\mat{A})=\begin{vmatrix}
        \lambda&0&0\\
        -1&\lambda&a\\
        0&-1&\lambda-(a+1)
    \end{vmatrix}=\lambda\begin{vmatrix}
        \lambda&a\\
        -1&\lambda-(a+1)
    \end{vmatrix}=\lambda(\lambda-1)(\lambda-a)\]
    首先当$a\neq0$且$a\neq1$时, $\mat{A}$有$3$个不同的特征值,因此此时$\mat{A}$可以对角化.\\
    当$a=0$时, $\mat{A}$的对应于特征值$0$的特征向量为$\begin{bmatrix}
        0&1&-1
    \end{bmatrix}^\t$,特征子空间的维数为$1$,与代数重数不等,故此时$\mat{A}$不可对角化.\\
    当$a=1$时, $\mat{A}$的对应于特征值$1$的特征向量为$\begin{bmatrix}
        0&1&-1
    \end{bmatrix}^\t$,特征子空间的维数与代数重数不等,故此时$\mat{A}$不可对角化.\\
    于是$a\neq0$且$a\neq1$时$\mat{A}$可对角化.
    \item 对于特征值$0$,考虑齐次方程$(0\mat{I}-\mat{A})\vec{x}=\mbf0$,对系数矩阵做行变换可得
    \[\begin{bmatrix}
        0&0&0\\
        -1&0&a\\
        0&-1&-a-1
    \end{bmatrix}\longrightarrow\begin{bmatrix}
        1&0&-a\\
        0&1&a+1\\
        0&0&0
    \end{bmatrix}\]
    于是对应于特征值$0$的特征向量$\bs\alpha_1=\begin{bmatrix}
        a&-(a+1)&1
    \end{bmatrix}^\t$.\\
    对于特征值$1$,考虑齐次方程$(1\mat{I}-\mat{A})\vec{x}=\mbf0$,对系数矩阵做初等行变换可得
    \[\begin{bmatrix}
        1&0&0\\
        -1&1&a\\
        0&-1&-a
    \end{bmatrix}\longrightarrow\begin{bmatrix}
        1&0&0\\
        0&1&a\\
        0&0&0
    \end{bmatrix}\]
    于是对应于特征值$1$的特征向量$\bs\alpha_2=\begin{bmatrix}
        0&-a&1
    \end{bmatrix}^\t$.\\
    对于特征值$a$,考虑齐次方程$(a\mat{I}-\mat{A})\vec{x}=\mbf0$,对系数矩阵做行变换可得
    \[\begin{bmatrix}
        a&0&0\\
        -1&a&a\\
        0&-1&-1
    \end{bmatrix}\longrightarrow\begin{bmatrix}
        a&0&0\\
        0&1&1\\
        0&0&0
    \end{bmatrix}\]
    于是对应于特征值$a$的特征向量为$\bs\alpha_3=\begin{bmatrix}
        0&-1&1
    \end{bmatrix}^\t$.\\
    于是令
    \[\mat{P}=\begin{bmatrix}
        \bs\alpha_1&\bs\alpha_2&\bs\alpha_3
    \end{bmatrix}=\begin{bmatrix}
        a&0&0\\
        -(a+1)&-a&-1\\
        1&1&1
    \end{bmatrix}\]
    即有
    \[\mat{P}^{-1}\mat{A}\mat{P}=\diag\{0,1,a\}\]
\end{enumerate}
\end{solution}
\begin{homework}[5(18')]
    令矩阵
    \[\mat{A}(x,a,n)=\begin{bmatrix}
        x&a&\cdots&a\\
        a&x&\cdots&a\\
        \vdots&\vdots&\ddots&\vdots\\
        a&a&\cdots&x
    \end{bmatrix}_{n\times n}\]
    于是有如下等式成立:
    \[\mat{A}(x,a,n)\mat{A}(y,b,n)=\mat{A}(z,c,n)\]
    \begin{enumerate}
        \item 试用$x,y,a,b$表示出$z,c$.
        \item 判断矩阵$\mat{A}(0,1,4)$是否可逆,若可逆则求出其逆矩阵.
    \end{enumerate}
\end{homework}
\begin{solution}
\begin{enumerate}
    \item 令$\mat{E}_n$为系数全为$1$的$n\times n$矩阵,则$\mat{E}_n^2=n\mat{E}_n$.注意到
    \[\mat{A}(x,a,n)=(x-a)\mat{I}_n+a\mat{E}_n\]
    从而
    \[\begin{aligned}
        \mat{A}(x,a,n)\mat{A}(y,b,n)
        &=[(x-a)\mat{I}_n+a\mat{E}_n][(y-b)\mat{I}_n+b\mat{E}_n]\\
        &=(x-a)(y-b)\mat{I}_n+(x-a)b\mat{E}_n+a(y-b)\mat{E}_n+abn\mat{E}_n\\
        &=(x-a)(y-b)\mat{I}_n+[(x-a)b+a(y-b)+abn]\mat{E}_n
    \end{aligned}\]
    从而
    \[c=ay+bx+(n-2)ab\]
    \[z=(x-a)(y-b)+c=xy+(n-1)ab\]
    \item 假定$\mat{A}(0,1,4)$可逆,于是存在$\mat{A}(y,b,n)$使得
    \[\mat{A}(0,1,4)\mat{A}(y,b,n)=\mat{A}(1,0,4)=\mat{I}\]
    于是可以列出方程
    \[\left\{\begin{array}{l}
        3b=1\\
        y+2b=0
    \end{array}\right.\]
    从而
    \[y=-\dfrac23,\quad b=\dfrac13\]
    于是题设矩阵可逆,其逆矩阵为$\mat{A}\left(-\dfrac23,\dfrac13,4\right)$.
\end{enumerate}
\end{solution}
\begin{homework}[6(24')]
    令$V=\{\mat{A}\in M_2(\R):\tr(\mat{A})=0\}$是迹为$0$的$2\times2$实矩阵构成的线性空间.取可逆矩阵$\mat{P}=\begin{bmatrix}
        0&1\\1&0
    \end{bmatrix}$,定义映射$\Phi_\mat{P}:V\to V$为$\Phi_{\mat{P}}(\mat{A})=\mat{P}^{-1}\mat{A}\mat{P}$.
    \begin{enumerate}
        \item 证明:$\mat{E}_{11}-\mat{E}_{22}=\begin{bmatrix}
            1&0\\0&-1
        \end{bmatrix},\mat{E}_{12}=\begin{bmatrix}
            0&1\\0&0
        \end{bmatrix},\mat{E}_{21}=\begin{bmatrix}
            0&0\\1&0
        \end{bmatrix}$构成$V$的一组基.
        \item 求$\Phi_{\mat{P}}$在基$\{\mat{E}_{11}-\mat{E}_{22},\mat{E}_{12},\mat{E}_{21}\}$下的矩阵.
        \item 求线性映射$\Phi_{\mat{P}}$的特征值及其在$V$中的一个特征向量.
        \item 对于一个可逆的$2\times2$矩阵$\mat{U}$,类似地定义线性映射$\Phi_{\mat{U}}:V\to V$为$\Phi_\mat{U}(\mat{A})=\mat{U}^{-1}\mat{A}\mat{U}$.如果$\mat{U}$有特征值$2$和$4$,求映射$\Phi_\mat{U}$的特征值并说明理由.
    \end{enumerate}
\end{homework}
\begin{solution}
\begin{enumerate}
    \item 对于任意$\mat{A}\in V$,由于$\tr(\mat{A})=0$,因此其对角元互为相反数.这样的$\mat{A}$总可以表示为$\mat{A}=\begin{bmatrix}
        a&b\\
        c&-a
    \end{bmatrix}$,于是则有
    \[\mat{A}=a(\mat{E}_{11}-\mat{E}_{22})+b\mat{E}_{12}+c\mat{E}_{21}\]
    于是$V$中的任意元素都可以被上述三个矩阵线性表出;并且$\mat{A}=\mbf0$当且仅当$a=b=c=0$,因此上述三个矩阵线性无关.于是它们构成$V$的一组基.
    \item 首先有
    \[\mat{P}^{-1}=\dfrac{1}{\det\mat{P}}\mat{P}^\ast=\begin{bmatrix}
        0&1\\1&0
    \end{bmatrix}\]
    于是
    \[\mat{P}^{-1}(\mat{E}_{11}-\mat{E}_{22})\mat{P}=\begin{bmatrix}
        -1&0\\0&1
    \end{bmatrix}=-(\mat{E}_{11}-\mat{E}_{22})\]
    \[\mat{P}^{-1}\mat{E}_{12}\mat{P}=\begin{bmatrix}
        0&0\\1&0
    \end{bmatrix}=\mat{E}_{21}\]
    \[\mat{P}^{-1}\mat{E}_{21}\mat{P}=\begin{bmatrix}
        0&1\\0&0
    \end{bmatrix}=\mat{E}_{12}\]
    于是$\Phi_{\mat{P}}$在题设的基下的矩阵为
    \[\mat{M}=\begin{bmatrix}
        -1&0&0\\
        0&0&1\\
        0&1&0
    \end{bmatrix}\]
    \item $\Phi_\mat{P}$的矩阵$\mat{M}$的特征多项式为
    \[\det(\lambda\mat{I}-\mat{M})=\begin{vmatrix}
        \lambda+1&0&0\\
        0&\lambda&-1\\
        0&-1&\lambda
    \end{vmatrix}=(\lambda+1)^2(\lambda-1)\]
    对于特征值$-1$,其特征向量为$\begin{bmatrix}
        1&0&0
    \end{bmatrix}^\t$和$\begin{bmatrix}
        0&-1&1
    \end{bmatrix}^\t$,即$\begin{bmatrix}
        1&0\\
        0&-1
    \end{bmatrix}$和$\begin{bmatrix}
        0&1\\
        -1&0
    \end{bmatrix}$.\\
    对于特征值$1$,其特征向量为$\begin{bmatrix}
        0&1&1
    \end{bmatrix}^\t$,即$\begin{bmatrix}
        0&1\\1&0
    \end{bmatrix}$.
    \item 设$\Phi_{\mat{U}}(\mat{A})=\lambda\mat{A}$,则有
    \[\mat{U}^{-1}\mat{A}\mat{U}=\lambda\mat{A}\]
    考虑$\mat{U}$的对应于特征值$2$的特征向量$\bs\alpha$,对上式两边右乘$\bs\alpha$可得
    \[2\mat{U}^{-1}\mat{A}\bs\alpha=\lambda\mat{A}\bs\alpha\]
    假定$\mat{A}\bs\alpha\neq\mbf0$,则它是$\mat{U}^{-1}$的对应于特征值$\dfrac{\lambda}{2}$的特征向量.由于$\mat{U}^{-1}$的特征值分别为$\dfrac12$和$\dfrac14$,因此$\lambda=1$或$\lambda=\dfrac12$.\\
    同理考虑$\mat{U}$的对应于特征值$4$的特征向量,可得$\lambda=1$或$\lambda=2$.综上可知$\Phi_\mat{U}$的特征值为$\dfrac12,1,2$.
\end{enumerate}
\end{solution}
\begin{homework}[7(10')]
    证明:如果$\mat{A}$是$n$级正定矩阵, $\mat{B}$是$n$级实对称矩阵,则存在一个$n$级实可逆矩阵$\mat{C}$使得$\mat{C}^\t\mat{A}\mat{C}$和$\mat{C}^\t\mat{B}\mat{C}$均为对角矩阵.
\end{homework}
\begin{proof}
    由于$\mat{A}$是正定矩阵,因此其合同于$\mat{I}_n$,于是存在$n$级可逆矩阵$\mat{Q}$使得
    \[\mat{Q}^\t\mat{A}\mat{Q}=\mat{I}\]
    由于$\mat{B}$是实对称矩阵,因此$\mat{Q}^\t\mat{B}\mat{Q}$也是实对称矩阵,于是存在$n$级可逆矩阵$\mat{P}$使得
    \[\mat{P}^\t(\mat{Q}^\t\mat{B}\mat{Q})\mat{P}=\mat{D}\]
    其中对角矩阵$\mat{D}$是$\mat{Q}^\t\mat{B}\mat{Q}$的合同标准形.令$\mat{C}=\mat{Q}\mat{P}$,则
    \[\mat{C}^\t\mat{A}\mat{C}=\mat{P}^\t\mat{I}_n\mat{P}=\mat{I}_n\]
    于是存在$n$级实可逆矩阵$\mat{C}$满足题设条件.
\end{proof}
\end{document}