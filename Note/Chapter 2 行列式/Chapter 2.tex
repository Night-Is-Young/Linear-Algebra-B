\documentclass{ctexart}
\usepackage{Note}
\begin{document}
\section{行列式}
\subsection{$n$元排列}
\subsubsection{$n$元排列的相关定义}
\begin{definition}[$n$元排列]
    $n$个不同正整数的一个全排列称为一个$n$元排列.
\end{definition}
\begin{lemma}
    $n$元排列的总数是$n!$.
\end{lemma}
\begin{proof}
    小学二年级的同学就学过了.
\end{proof}
\begin{definition}[顺序与逆序]
    在$n$元排列$a_1a_2\cdots a_n$中,任取$1\leqslant i<j\leqslant n$,如果$a_i<a_j$,称这一对数构成\tbf{顺序};如果$a_i>a_j$,称这一对数构成\tbf{逆序}.
\end{definition}
\begin{definition}[逆序数]
    一个$n$元排列$a_1a_2\cdots a_n$中的逆序的总数称为\tbf{逆序数},记作$\tau\left(a_1a_2\cdots a_n\right)$.
\end{definition}
\begin{definition}[奇排列与偶排列]
    逆序数为奇数的排列称为\tbf{奇排列},逆序数为偶数的排列称为\tbf{偶排列}.
\end{definition}
\begin{definition}[对换]
    保持$n$元排列$a_1a_2\cdots a_n$中的其余数不变,交换$a_i$与$a_j$的位置(其中$1\leqslant i<j\leqslant n$),这一操作称\tbf{对换}.
\end{definition}
\subsubsection{$n$元排列的性质}
\begin{theorem}[对换操作的特性]
    对换改变$n$元排列的奇偶性.
\end{theorem}
\begin{proof}
    我们假定对换操作是对$n$元排列$a_1\cdots a_n$中的$a_i,a_j$(不妨假定$1\leqslant i<j\leqslant n$)进行的.现在分类讨论.\\
    \indent 如果$a_i$与$a_j$相邻,那么交换两者不会改变它们与前后的数的大小关系,因此只需考虑$a_i$与$a_j$即可.如果$a_i<a_j$,那么对换后逆序数增大$1$;如果$a_i>a_j$,那么对换后逆序数减小$1$,两种情形下逆序数的奇偶性都会发生改变.\\
    \indent 如果$a_i$与$a_j$不相邻,那么假定它们之间有$k$个数.我们做如下操作:将$a_i$与其后面$k$个数依次对换,再将$a_i$与$a_j$对换,最后将$a_j$与前面$n$个数依次对换.容易看出,这一系列操作的结果就是将$a_i$与$a_j$对换而不改变其它数.这相当于进行了$2n+1$次相邻对换,每次都改变逆序数的奇偶性,因此最终仍改变排列的奇偶性.\\
    \indent 综上,命题得证.
\end{proof}
\begin{theorem}
    任一$n$元排列与排列$12\cdots n$可以通过一系列对换互变,并且作对换的次数与该$n$元排列的奇偶性相同.
\end{theorem}
\begin{proof}
    这容易从前面的定理推得.
\end{proof}
\begin{problem}
    如果$n$元排列$j_1j_2\cdots j_n$的逆序数为$r$,求$j_{n}j_{n-1}\cdots j_1$的逆序数.
\end{problem}
\begin{solution}
    $j_1j_2\cdots j_n$中构成顺序的数对在$j_{n}j_{n-1}\cdots j_1$构成逆序,反之亦然.又因为$n$个数一共构成
    \[\dfrac{n(n+1)}{2}\]
    对数,于是
    \[\tau\left(j_{n}j_{n-1}\cdots j_1\right)=\dfrac{n(n+1)}{2}-r\]
\end{solution}
\begin{problem}
    设$c_1\cdots c_kd_1\cdots d_{n-k}$是由$1,2,\cdots,n$形成的$n$元排列,试证明:
    \[(-1)^{\tau\left(c_1\cdots c_kd_1\cdots d_{n-k}\right)}=(-1)^{\tau\left(c_1\cdots c_k\right)+\tau\left(d_1\cdots d_{n-k}\right)}\cdot(-1)^{\li c+k}\cdot(-1)^{\frac{k(k+1)}{2}}\]
\end{problem}
\begin{proof}
    将$c_1\cdots c_k$经过$s$次对换成$a_1\cdots a_k$,其中$a_1<\cdots <a_k$.后者是偶排列,因此$c_1\cdots c_k$的奇偶性与$s$相同.\\
    \indent 对于变换后的$a_1\cdots a_k$而言,考虑$1\leqslant i\leqslant k$,$a_i$后比它小的数共有$a_i-i$个.于是
    \[\begin{aligned}
        (-1)^{\tau\left(c_1\cdots c_kd_1\cdots d_{n-k}\right)}
        &= (-1)^s(-1)^{\tau\left(a_1\cdots a_kd_1\cdots d_{n-k}\right)} \\
        &= (-1)^{\tau(c_1\cdots c_k)}(-1)^{\left(a_1-1\right)+\cdots+\left(a_k-k\right)+\tau\left(d_1\cdots d_{n-k}\right)}\\
        &= (-1)^{\tau\left(c_1\cdots c_k\right)+\tau\left(d_1\cdots d_{n-k}\right)}\cdot(-1)^{\li a+k}\cdot(-1)^{-\frac{k(k+1)}{2}}\\
        &= (-1)^{\tau\left(c_1\cdots c_k\right)+\tau\left(d_1\cdots d_{n-k}\right)}\cdot(-1)^{\li c+k}\cdot(-1)^{\frac{k(k+1)}{2}}
    \end{aligned}\]
\end{proof}
\subsection{$n$阶行列式的定义}
\begin{definition}[$n$阶行列式]
    定义$n$阶行列式
    \[\begin{vmatrix}
        a_{11}&\cdots&a_{1n}\\
        \vdots&\ddots&\vdots\\
        a_{n1}&\cdots&a_{nn}
    \end{vmatrix}\xlongequal{\text{def}}\sum_{j_1\cdots j_n}(-1)^{\tau\left(j_1\cdots j_n\right)}a_{1j_1}\cdots a_{nj_n}\]
    其中$\displaystyle\sum_{j_1\cdots j_n}$表示对所有$n$元排列求和.上式称为$n$元行列式的\tbf{完全展开式}.\\
    考虑矩阵
    \[\mat{A}=\begin{bmatrix}
        a_{11}&\cdots&a_{1n}\\
        \vdots&\ddots&\vdots\\
        a_{n1}&\cdots&a_{nn}
    \end{bmatrix}\]
    前面定义的$n$阶行列式也称为方阵$\mat{A}$的行列式,记作$\left|\mat{A}\right|$或$\det\mat A$.
\end{definition}
\begin{theorem}
    上三角矩阵的行列式的值等于其主对角线上各元素的积.
\end{theorem}
\begin{proof}
    考虑$n$阶上三角矩阵
    \[\mat{A}=\begin{bmatrix}
        a_{11}&a_{12}&\cdots&a_{1(n-1)}&a_{1n}\\
        0     &a_{22}&\cdots&a_{2(n-1)}&a_{2n}\\
        \vdots&\vdots&\ddots&\vdots&\vdots\\
        0     &0     &\cdots&a_{(n-1)(n-1)}&a_{(n-1)n}\\
        0     &0     &\cdots&0     &a_{nn}
    \end{bmatrix}\]
    其行列式$\det\mat{A}$的展开式中的各项为
    \[(-1)^{\tau\left(j_1\cdots j_n\right)}a_{1j_1}\cdots a_{nj_n}\]
    当$j_n\neq n$时,$a_{nj_n}=0$,从而求和项为$0$,仅当$j_n=n$时才有可能不为$0$.\\
    同样地,仅当$j_{n-1}=n-1$时求和项才有可能不为$0$.\\
    依次类推,当且仅当$j_i=i$对所有$1\leqslant i\leqslant n$成立时,求和项才有可能不为$0$.又因为$12\cdots n$是偶排列,于是这一项恰好就是$\mat{A}$的主对角线上各元素的乘积,即
    \[\det\mat{A}=a_{11}a_{22}\cdots a_{nn}=\prod_{i=1}^{n}a_{ii}\]
    命题得证.
\end{proof}
\begin{theorem}
    对于$n$阶方阵$\mat{A}$,给定行指标的排列$i_1\cdots i_n$,有
    \[\det\mat{A}=\sum_{k_1\cdots k_n}(-1)^{\tau\left(i_1\cdots i_n\right)+\tau\left(k_1+\cdots k_n\right)}a_{i_1k_1}\cdots a_{i_nk_n}\]
    或者给定列指标的排列$k_1\cdots k_n$,有
    \[\det\mat{A}=\sum_{i_1\cdots i_n}(-1)^{\tau\left(i_1\cdots i_n\right)+\tau\left(k_1+\cdots k_n\right)}a_{i_1k_1}\cdots a_{i_nk_n}\]
\end{theorem}
\begin{proof}
    我们考虑$n$阶行列式的每一项
    \[(-1)^{\tau\left(j_1\cdots j_n\right)}a_{1j_1}\cdots a_{nj_n}\]
    忽略指数项,这必将一一对应于命题中求和的某一项
    \[(-1)^{\tau\left(i_1\cdots i_n\right)+\tau\left(k_1+\cdots k_n\right)}a_{i_1k_1}\cdots a_{i_nk_n}\]
    现在只需要证明
    \[(-1)^{\tau\left(j_1\cdots j_n\right)}=(-1)^{\tau\left(i_1\cdots i_n\right)+\tau\left(k_1+\cdots k_n\right)}\]
    即可.考虑$a_{1j_1}\cdots a_{nj_n}$经$s$次对换成$a_{i_1k_1}\cdots a_{i_nk_n}$,那么$12\cdots n$经$s$次对换成$i_1\cdots i_n$,$j_1\cdots j_n$经$s$次对换成$k_1\cdots k_n$.\\
    于是
    \[(-1)^{\tau\left(i_1\cdots i_n\right)}=(-1)^s\]
    \[(-1)^{\tau\left(k_1\cdots k_n\right)}=(-1)^{\tau\left(j_1\cdots j_n\right)}(-1)^s\]
    上述两式左右分别相乘就有
    \[(-1)^{\tau\left(i_1\cdots i_n\right)+\tau\left(k_1+\cdots k_n\right)}=(-1)^{2s}(-1)^{\tau\left(j_1\cdots j_n\right)}=(-1)^{\tau\left(j_1\cdots j_n\right)}\]
    于是命题得证.
\end{proof}
由上述命题可以得到以下的推论:
\begin{lemma}
    行列式中的行和列是等价的,也即对于$n$阶矩阵$\mat{A}$而言有
    \[\det\mat{A}=\sum_{j_1\cdots j_n}(-1)^{\tau\left(j_1\cdots j_n\right)}a_{1j_1}\cdots a_{nj_n}=\sum_{j_1\cdots j_n}(-1)^{\tau\left(j_1\cdots j_n\right)}a_{j_11}\cdots a_{j_nn}\]
\end{lemma}
\subsection{行列式的性质}
\begin{theorem}
    对于方阵$\mat{A}$而言,$\det\mat{A}=\det\mat{A}^\text{t}$.
\end{theorem}
\begin{proof}
    这由行列式中行和列等价这一推论即可得到.
\end{proof}
由此我们知道,关于行列式中行的性质对列同样成立.因此我们下面只讨论行的性质.
\begin{theorem}
    行列式的行公因子可以提出,即
    \[\begin{vmatrix}
        a_{11}&a_{12}&\cdots&a_{1n}\\
        \vdots&\vdots&\ddots&\vdots\\
        pa_{i1}&pa_{i2}&\cdots&pa_{in}\\
        \vdots&\vdots&\ddots&\vdots\\
        a_{n1}&a_{n2}&\cdots&a_{nn}
    \end{vmatrix}=p
    \begin{vmatrix}
        a_{11}&a_{12}&\cdots&a_{1n}\\
        \vdots&\vdots&\ddots&\vdots\\
        a_{i1}&a_{i2}&\cdots&a_{in}\\
        \vdots&\vdots&\ddots&\vdots\\
        a_{n1}&a_{n2}&\cdots&a_{nn}
    \end{vmatrix}\]
\end{theorem}
\begin{proof}
    \[ LHS
    =\sum_{j_1\cdots j_n}(-1)^{\tau\left(j_1\cdots j_n\right)}a_{1j_1}\cdots (pa_{ij_i})\cdots a_{nj_n}
    =p\sum_{j_1\cdots j_n}(-1)^{\tau\left(j_1\cdots j_n\right)}a_{1j_1}\cdots a_{nj_n}
    =RHS\]
\end{proof}
\begin{theorem}
    行列式按行可加,即
    \[\begin{vmatrix}
        a_{11}&a_{12}&\cdots&a_{1n}\\
        \vdots&\vdots&\ddots&\vdots\\
        b_{i1}+c_{ia}&b_{i2}+c_{i2}&\cdots&b_{in}+c_{in}\\
        \vdots&\vdots&\ddots&\vdots\\
        a_{n1}&a_{n2}&\cdots&a_{nn}
    \end{vmatrix}=
    \begin{vmatrix}
        a_{11}&a_{12}&\cdots&a_{1n}\\
        \vdots&\vdots&\ddots&\vdots\\
        b_{i1}&b_{i2}&\cdots&b_{in}\\
        \vdots&\vdots&\ddots&\vdots\\
        a_{n1}&a_{n2}&\cdots&a_{nn}
    \end{vmatrix}+
    \begin{vmatrix}
        a_{11}&a_{12}&\cdots&a_{1n}\\
        \vdots&\vdots&\ddots&\vdots\\
        c_{i1}&c_{i2}&\cdots&c_{in}\\
        \vdots&\vdots&\ddots&\vdots\\
        a_{n1}&a_{n2}&\cdots&a_{nn}
    \end{vmatrix}\]
\end{theorem}
\begin{proof}
    \[\begin{aligned}
        LHS
        &=\sum_{j_1\cdots j_n}(-1)^{\tau\left(j_1\cdots j_n\right)}a_{1j_1}\cdots (b_{ij_i}+c_{ij_i})\cdots a_{nj_n}\\
        &=\sum_{j_1\cdots j_n}(-1)^{\tau\left(j_1\cdots j_n\right)}a_{1j_1}\cdots b_{ij_i}\cdots a_{nj_n}+\sum_{j_1\cdots j_n}(-1)^{\tau\left(j_1\cdots j_n\right)}a_{1j_1}\cdots c_{ij_i}\cdots a_{nj_n}\\
        &=RHS
    \end{aligned}\]
\end{proof}
\begin{theorem}
    两行互换,行列式反号,即
    \[\begin{vmatrix}
        a_{11}&a_{12}&\cdots&a_{1n}\\
        \vdots&\vdots&\ddots&\vdots\\
        a_{i1}&a_{i2}&\cdots&a_{in}\\
        \vdots&\vdots&\ddots&\vdots\\
        a_{k1}&a_{k2}&\cdots&a_{kn}\\
        \vdots&\vdots&\ddots&\vdots\\
        a_{n1}&a_{n2}&\cdots&a_{nn}
    \end{vmatrix}=-
    \begin{vmatrix}
        a_{11}&a_{12}&\cdots&a_{1n}\\
        \vdots&\vdots&\ddots&\vdots\\
        a_{k1}&a_{k2}&\cdots&a_{kn}\\
        \vdots&\vdots&\ddots&\vdots\\
        a_{i1}&a_{i2}&\cdots&a_{in}\\
        \vdots&\vdots&\ddots&\vdots\\
        a_{n1}&a_{n2}&\cdots&a_{nn}
    \end{vmatrix}\]
\end{theorem}
\begin{proof}
    \[\begin{aligned}
        RHS
        &= -\sum_{j_1\cdots j_i\cdots j_k\cdots j_n}(-1)^{\tau\left(j_1\cdots j_i\cdots j_k\cdots j_n\right)}a_{1j_1}\cdots a_{kj_i}\cdots a_{ij_k}\cdots a_{nj_n} \\
        &= -\sum_{j_1\cdots j_k\cdots j_i\cdots j_n}(-1)\cdot(-1)^{\tau\left(j_1\cdots j_k\cdots j_i\cdots j_n\right)}a_{1j_1}\cdots a_{ij_k}\cdots a_{kj_i}\cdots a_{nj_n} \\
        &= \sum_{j_1\cdots j_k\cdots j_i\cdots j_n}\cdot(-1)^{\tau\left(j_1\cdots j_k\cdots j_i\cdots j_n\right)}a_{1j_1}\cdots a_{ij_k}\cdots a_{kj_i}\cdots a_{nj_n} \\
        &= LHS
    \end{aligned}\]
\end{proof}
\begin{theorem}
    两行相同,行列式的值为$0$,即
    \[\begin{vmatrix}
        a_{11}&a_{12}&\cdots&a_{1n}\\
        \vdots&\vdots&\ddots&\vdots\\
        a_{i1}&a_{i2}&\cdots&a_{in}\\
        \vdots&\vdots&\ddots&\vdots\\
        a_{i1}&a_{i2}&\cdots&a_{in}\\
        \vdots&\vdots&\ddots&\vdots\\
        a_{n1}&a_{n2}&\cdots&a_{nn}
    \end{vmatrix}\]
\end{theorem}
\begin{proof}
    记行列式对应的矩阵为$\mat{A}$,互换相同的两行后对应的矩阵仍为$\mat{A}$.根据前面的定理可得
    \[\det\mat{A}=-\det\mat{A}\]
    于是
    \[\det\mat{A}=0\]
\end{proof}
\begin{theorem}
    两行成倍数,行列式的值为$0$.
\end{theorem}
\begin{proof}
    根据前面的定理易证.
\end{proof}
\begin{theorem}
    将行列式的一行的倍数加到另一行上,行列式的值不变.
\end{theorem}
\begin{proof}
    根据前面的定理易证.
\end{proof}
综上所述,我们可以得到下面的命题.
\begin{theorem}
    如果方阵$\mat{A}$经初等行变换可以得到方阵$\mat{B}$,那么存在$l\in\F$使得$\det\mat B=l\det\mat A$.
\end{theorem}
利用前面的定理,可以将行列式按行拆分成易于计算的行列式,也可以将行列式变换为上三角行列式进行计算.
\begin{problem}
    计算行列式:
    \[\begin{vmatrix}
        -2&1&-3\\
        98&101&97\\
        1&-3&4
    \end{vmatrix}\]
\end{problem}
\begin{solution}
    我们有
    \[\begin{aligned}
        \begin{vmatrix}-2&1&-3\\98&101&97\\1&-3&4\end{vmatrix}
        &= \begin{vmatrix}-2&1&-3\\100&100&100\\1&-3&4\end{vmatrix}+\begin{vmatrix}-2&1&-3\\-2&1&-3\\1&-3&4\end{vmatrix}=\begin{vmatrix}-2&1&-3\\100&100&100\\1&-3&4\end{vmatrix}=100\begin{vmatrix}-2&1&-3\\1&1&1\\1&-3&4\end{vmatrix}\\
        &=100\begin{vmatrix}1&1&1\\1&-3&4\\-2&1&-3\end{vmatrix}=100\begin{vmatrix}1&1&1\\0&-4&3\\0&3&-1\end{vmatrix}=100\begin{vmatrix}1&1&1\\0&-4&3\\0&0&\frac54\end{vmatrix}
        =-500
    \end{aligned}\]
\end{solution}
\begin{problem}
    计算$n$阶行列式:
    \[\begin{vmatrix}
        k&\lambda&\cdots&\lambda\\
        \lambda&k&\cdots&\lambda\\
        \vdots&\vdots&\ddots&\vdots\\
        \lambda&\lambda&\cdots&k
    \end{vmatrix}\]
\end{problem}
\begin{solution}
    这个行列式的特点在于每一行的元素之和均为$(n-1)\lambda+k$.为此,我们可以将第$2$到第$n$列的元素都加到第$1$列上,然后提取公因子,接着继续化简:
    \[\begin{aligned}
        \begin{vmatrix}
            k&\lambda&\cdots&\lambda\\
            \lambda&k&\cdots&\lambda\\
            \vdots&\vdots&\ddots&\vdots\\
            \lambda&\lambda&\cdots&k
        \end{vmatrix}
        &= \begin{vmatrix}
            (n-1)\lambda+k&\lambda&\lambda&\cdots&\lambda\\
            (n-1)\lambda+k&k&\lambda&\cdots&\lambda\\
            \vdots&\vdots&\vdots&\ddots&\vdots\\
            (n-1)\lambda+k&\lambda&\lambda&\cdots&k
        \end{vmatrix}=\left[(n-1)\lambda+k\right]\begin{vmatrix}
            1&\lambda&\lambda&\cdots&\lambda\\
            1&k&\lambda&\cdots&\lambda\\
            \vdots&\vdots&\vdots&\ddots&\vdots\\
            1&\lambda&\lambda&\cdots&k
        \end{vmatrix} \\
        &= \left[(n-1)\lambda+k\right]\begin{vmatrix}
            1&\lambda&\lambda&\cdots&\lambda\\
            0&k-\lambda&0&\cdots&0\\
            \vdots&\vdots&\vdots&\ddots&\vdots\\
            0&0&0&\cdots&k-\lambda
        \end{vmatrix}=\left[(n-1)\lambda+k\right](k-\lambda)^{n-1}
    \end{aligned}\]
\end{solution}
\subsection{行列式按行展开}
\begin{definition}[余子式和代数余子式]
    在$n$级方阵$\mat{A}$中删去元素$(i,j)$所在的行和列,剩下的元素按原来次序形成的$n-1$级方阵的行列式称为$\mat{A}$的$(i,j)$元的余子式,记作$M_{ij}$.令
    \[A_{ij}=(-1)^{i+j}M_{ij}\]
    称$A_{ij}$是$\mat{A}$的$(i,j)$元的代数余子式.
\end{definition}
\begin{theorem}[Laplace定理]
    $n$级方阵$\mat{A}$的行列式$\det\mat{A}$等于其第$i$行元素与自身代数余子式的乘积之和,即对任意$1\leqslant i\leqslant n$有
    \[\det\mat{A}=\sum_{j=1}^{n}a_{ij}A_{ij}\]
\end{theorem}
\begin{proof}
    我们可以把$\det\mat{A}$的完全展开式的$n!$项按照第$i$行的$n$个元素分组,即
    \[\begin{aligned}
        \det\mat{A}
        &= \sum_{k_1\cdots k_n}(-1)^{\tau\left(k_1\cdots\cdots k_n\right)}\prod_{p=1}^{n}a_{pk_p} \\
        &= \sum_{j=1}^{n}\sum_{k_1\cdots k_{i-1}jk_{i+1}\cdots k_n}(-1)^{\tau\left(k_1\cdots k_{i-1}jk_{i+1}\cdots k_n\right)}\prod_{p=1}^{n}a_{pk_p} \\
        &= \sum_{j=1}^{n}a_{ij}\left(\sum_{k_1\cdots j\cdots k_n}(-1)^{\tau\left(k_1\cdots j\cdots k_n\right)}\prod_{p=1,p\neq i}^{n}a_{pk_p}\right)
    \end{aligned}\]
    现在考虑把排列$k_1\cdots j\cdots k_n$.将$j$移动到第一位变成$jk_1\cdots k_{i-1}k_{i+1}\cdots k_n$需要经历$i-1$次对换(因为$j$在第$i$位),于是
    \[(-1)^{\tau\left(jk_1\cdots k_{i-1}k_{i+1}\cdots k_n\right)+i-1}=(-1)^{\tau\left(k_1\cdots j\cdots k_n\right)}\]
    而$jk_1\cdots k_{i-1}k_{i+1}\cdots k_n$与$k_1\cdots k_{i-1}k_{i+1}\cdots k_n$的逆序数之差就是比$j$小的数的数目,即$j-1$,因此
    \[(-1)^{\tau\left(k_1\cdots k_{i-1}k_{i+1}\cdots k_n\right)+j-1}=(-1)^{\tau\left(jk_1\cdots k_{i-1}k_{i+1}\cdots k_n\right)+j-1}\]
    于是就有
    \[(-1)^{\tau\left(k_1\cdots j\cdots k_n\right)}=(-1)^{\tau\left(k_1\cdots k_{i-1}k_{i+1}\cdots k_n\right)+i+j-2}=(-1)^{\tau\left(k_1\cdots k_{i-1}k_{i+1}\cdots k_n\right)+i+j}\]
    由于$j$的位置是固定的,因此我们在前述求和项中可以只考虑排列$k_1\cdots k_{i-1}k_{i+1}\cdots k_n$.于是,前面的式子可以改写为
    \[\begin{aligned}
        \det\mat{A}
        &= \sum_{j=1}^{n}a_{ij}\left(\sum_{k_1\cdots k_{i-1}k_{i+1}\cdots k_n}(-1)^{\tau\left(k_1\cdots k_{i-1}k_{i+1}\cdots k_n\right)+i+j}\prod_{p=1,p\neq i}a_{pk_{p}}\right) \\
        &= \sum_{j=1}^{n}a_{ij}(-1)^{i+j}M_{ij}=\sum_{j=1}^{n}a_{ij}A_{ij}
    \end{aligned}\]
    这就证明了Laplace定理.
\end{proof}
\begin{theorem}[Laplace定理按列展开的形式]
    $n$级方阵$\mat{A}$的行列式$\det\mat{A}$等于其第$j$列元素与自身代数余子式的乘积之和,即对任意$1\leqslant j\leqslant n$有
    \[\det\mat{A}=\sum_{i=1}^{n}a_{ij}A_{ij}\]
\end{theorem}
\begin{proof}
    考虑$\mat{A}$的转置$\mat{A}^{\text{t}}$,我们已经知道$\det\mat{A}=\det\mat{A}^{\text{t}}$,那么对$\mat{A}^{\text{t}}$使用Laplace定理即可证得命题.
\end{proof}
\begin{theorem}
    $n$级方阵$\mat{A}$的行列式$\det\mat{A}$的第$i$行与第$k$行($k\neq i$)的对应元素的代数余子式的乘积之和为$0$,即
    \[\sum_{j=1}^{n}a_{ij}A_{kj}=0\]
\end{theorem}
\begin{proof}
    为了便于使用Laplace定理,我们构造矩阵$\mat{B}$使得$\mat{B}$的第$k$行与$\mat{A}$的第$i$行一致,其余元素和对应的$\mat{A}$的元素相同.这样,$\mat{B}$的第$k$行各元素的代数余子式与$\mat{A}$相同.\\
    由于$\mat{B}$有相同的两行,因此
    \[\det\mat{B}=0\]
    对$\mat{B}$的第$k$行应用Laplace定理,有
    \[\sum_{j=1}^{n}b_{kj}B_{kj}=\det\mat{B}=0\]
    又因为$b_{kj}=b_{ij}=a_{ij}$,$B_{kj}=A_{kj}$,于是
    \[\sum_{j=1}^{n}a_{ij}A_{kj}=0\]
\end{proof}
\begin{theorem}
    $n$级方阵$\mat{A}$的行列式$\det\mat{A}$的第$j$列与第$k$列($k\neq j$)的对应元素的代数余子式的乘积之和为$0$,即
    \[\sum_{i=1}^{n}a_{ij}A_{ik}=0\]
\end{theorem}
\begin{proof}
    这由Laplace定理的列展开形式就可以得到.
\end{proof}
\begin{problem}
    计算$n$阶行列式:
    \[\begin{vmatrix}
        a&b&0&\cdots&0&0&0\\
        0&a&b&\cdots&0&0&0\\
        0&0&a&\cdots&0&0&0\\
        \vdots&\vdots&\vdots&\ddots&\vdots&\vdots&\vdots\\
        0&0&0&\cdots&0&a&b\\
        b&0&0&\cdots&0&0&a
    \end{vmatrix}\]
\end{problem}
\begin{solution}
    注意到除去第一行第一列后的矩阵为上三角矩阵,除去最后一行第一列后的矩阵为下三角矩阵,因此我们先按第一列展开原行列式:
    \[\begin{aligned}
        \begin{vmatrix}
            a&b&0&\cdots&0&0&0\\
            0&a&b&\cdots&0&0&0\\
            0&0&a&\cdots&0&0&0\\
            \vdots&\vdots&\vdots&\ddots&\vdots&\vdots&\vdots\\
            0&0&0&\cdots&0&a&b\\
            b&0&0&\cdots&0&0&a
        \end{vmatrix}
        &= a\begin{vmatrix}
            a&b&\cdots&0&0&0\\
            0&a&\cdots&0&0&0\\
            \vdots&\vdots&\ddots&\vdots&\vdots&\vdots\\
            0&0&\cdots&0&a&b\\
            0&0&\cdots&0&0&a
        \end{vmatrix}+b(-1)^{n+1}\begin{vmatrix}
            b&0&\cdots&0&0&0\\
            a&b&\cdots&0&0&0\\
            0&a&\cdots&0&0&0\\
            \vdots&\vdots&\ddots&\vdots&\vdots&\vdots\\
            0&0&\cdots&0&a&b
        \end{vmatrix}\\
        &= aa^{n-1}+b(-1)^{n+1}b^{n-1} \\
        &= a^n+(-1)^{n+1}b^n
    \end{aligned}\]
\end{solution}
\begin{definition}[Vandermonde行列式]
    形如
    \[\begin{vmatrix}
        1&1&\cdots&1\\
        a_1&a_2&\cdots&a_n\\
        a_1^2&a_2^2&\cdots&a_n^2\\
        \vdots&\vdots&\ddots&\vdots\\
        a_1^{n-1}&a_2^{n-1}&\cdots&a_n^{n-1}
    \end{vmatrix}\]
    的行列式称作Vandermonde行列式.
\end{definition}
\begin{theorem}
    Vandermonde行列式的值为
    \[\prod_{1\leqslant i<j\leqslant n}\left(a_j-a_i\right)\]
\end{theorem}
\begin{proof}
    我们用归纳法证明上述命题.对于$n=2$的情形,有
    \[\begin{vmatrix}
        1&1\\a_1&a_2
    \end{vmatrix}=a_2-a_1\]
    现在考虑$n\geqslant 3$的情形.把第$i$行的$-a_i$倍加到第$i+1$行上,就有
    \[\begin{aligned}
        &\begin{vmatrix}
            1&1&\cdots&1\\
            a_1&a_2&\cdots&a_n\\
            a_1^2&a_2^2&\cdots&a_n^2\\
            \vdots&\vdots&\ddots&\vdots\\
            a_1^{n-1}&a_2^{n-1}&\cdots&a_n^{n-1}
        \end{vmatrix}\\
        =&\begin{vmatrix}
            1&1&\cdots&1\\
            0&a_2-a_1&\cdots&a_n-a_1\\
            0&a_2^2-a_1a_2&\cdots&a_n^2-a_1a_n\\
            \vdots&\vdots&\ddots&\vdots\\
            0&a_2^{n-1}-a_1a_2^{n-2}&\cdots&a_n^{n-1}-a_1a_n^{n-2}
        \end{vmatrix}
        =\begin{vmatrix}
            1&1&\cdots&1\\
            0&a_2-a_1&\cdots&a_n-a_1\\
            0&a_2\left(a_2-a_1\right)&\cdots&a_n\left(a_n-a_1\right)\\
            \vdots&\vdots&\ddots&\vdots\\
            0&a_2^{n-2}\left(a_2-a_1\right)&\cdots&a_n^{n-2}\left(a_n-a_1\right)
        \end{vmatrix} \\
        &= \left(a_2-a_1\right)\cdots\left(a_n-a_1\right)\begin{vmatrix}
            1&1&\cdots&1\\
            0&1&\cdots&1\\
            0&a_2&\cdots&a_n\\
            \vdots&\vdots&\ddots&\vdots\\
            0&a_2^{n-2}&\cdots&a_n^{n-2}
        \end{vmatrix}=\left(a_2-a_1\right)\cdots\left(a_n-a_1\right)\begin{vmatrix}
            1&\cdots&1\\
            a_2&\cdots&a_n\\
            \vdots&\ddots&\vdots\\
            a_2^{n-2}&\cdots&a_n^{n-2}
        \end{vmatrix}
    \end{aligned}\]
    根据归纳假设,后面的行列式即$n-1$阶的Vandermonde行列式,于是
    \[LHS=\left(a_2-a_1\right)\cdots\left(a_n-a_1\right)\prod_{2\leqslant i<j\leqslant n}\left(a_j-a_i\right)=\prod_{1\leqslant i<j\leqslant n}\left(a_j-a_i\right)\]
    于是归纳可得原命题成立.
\end{proof}
\begin{definition}[三对角线行列式]
    形如
    \[\begin{vmatrix}
        a&b&0&\cdots&0&0&0\\
        c&a&b&\cdots&0&0&0\\
        0&c&a&\cdots&0&0&0\\
        \vdots&\vdots&\vdots&\ddots&\vdots&\vdots&\vdots\\
        0&0&0&\cdots&c&a&b\\
        0&0&0&\cdots&0&c&a
    \end{vmatrix}\]
    的行列式称作三对角线行列式.
\end{definition}
\begin{theorem}
    上述$n$阶三对角线行列式$D_n$的值为
    \[D_n=\left\{\begin{array}{l}
        (n+1)\left(\dfrac{a}{2}\right)^n,a^2=4bc\\
        \dfrac{\lambda_1^{n+1}-\lambda_2^{n+1}}{\lambda_1-\lambda_2},a^2\neq4bc
    \end{array}\right.\]
    其中$\lambda_1,\lambda_2$是方程$\lambda^2-a\lambda+bc=0$的两根.
\end{theorem}
\begin{proof}
    将$D_n$按第一列展开有
    \[D_n=aD_{n-1}+(-1)^{1+2}c\begin{vmatrix}
        b&0&\cdots&0&0&0\\
        c&a&\cdots&0&0&0\\
        0&c&\cdots&0&0&0\\
        \vdots&\vdots&\ddots&\vdots&\vdots&\vdots\\
        0&0&\cdots&c&a&b\\
        0&0&\cdots&0&c&a
    \end{vmatrix}=aD_{n-1}-bcD_{n-2}\]
    令
    \[D_n-\lambda_1 D_{n-1}=\lambda_2\left(D_{n-1}-\lambda_1 D_{n-2}\right)\]
    于是可得
    \[\lambda_1+\lambda_2=a\ \ \ \ \ \lambda_1\lambda_2=bc\]
    于是$\lambda_1,\lambda_2$是二元一次方程
    \[\lambda^2-a\lambda+bc=0\]
    的两个根.\\
    \indent 当$a^2=4bc$时,$\lambda_1=\lambda_2=\dfrac{a}{2}$.此时
    \[D_n-\dfrac{a}{2}D_{n-1}=\dfrac{a}{2}\left(D_{n-1}-\dfrac{a}{2}D_{n-2}\right)\]
    又因为$D_1=a,D_2=a^2-bc=\dfrac{3a^2}{4}$,于是
    \[D_{n}-\dfrac{a}{2}D_{n-1}=\left(\dfrac{a}{2}\right)^n\]
    于是
    \[D_n=(n+1)\left(\dfrac{a}{2}\right)^n\]
    \indent 当$a^2\neq4bc$时,$\lambda_1\neq\lambda_2$.不妨设
    \[D_n=C_1\lambda_1^n+C_2\lambda_2^n\]
    由$D_1=a,D_2=a^2-bc$可解得
    \[C_1=\dfrac{\lambda_1^2-(a^2-bc)}{\lambda_1^2-\lambda_2^2}=\dfrac{\lambda_1}{\lambda_1-\lambda_2}\ \ \ \ \ C_2=\dfrac{(a^2-bc)-\lambda_2^2}{\lambda_1^2-\lambda_2^2}=\dfrac{-\lambda_2}{\lambda_1-\lambda_2}\]
    于是
    \[D_n=\dfrac{\lambda_1^{n+1}-\lambda_2^{n+1}}{\lambda_1-\lambda_2}\]
\end{proof}
\subsection{Cramer法则}
\begin{theorem}
    $\F$上有$n$个方程的$n$元线性方程组有唯一解,当且仅当其系数行列式(即系数矩阵$\mat{A}$的行列式$\det\mat{A}$)不等于$0$.
\end{theorem}
\begin{proof}
    对于一个有$n$个方程的$n$元线性方程组,其无解当且仅当增广矩阵中出现主元在最后一列的情形.这样,其系数矩阵经初等行变换时出现全零行,这意味着系数行列式为零(初等行变换不会使得非零的行列式变为零).\\
    现在考虑有解时的情形.我们已经知道,如果增广矩阵的非零行数目$r$小于未知量数目$n$,那么方程组有无穷多解.此时,系数矩阵经初等行变换时也会出现零行,从而系数行列式为零.\\
    当且仅当增广矩阵的非零行数目$r$等于未知量数目$n$时,方程式有唯一解.由于方程式有$n$个主元,并且要求系数矩阵变换成的阶梯形矩阵中各主元不再同一列上,因此其必定形如
    \[\begin{pmatrix}
        c_{11}&c_{12}&\cdots&c_{1n}\\
        0&c_{22}&\cdots&c_{2n}\\
        \vdots&\vdots&\ddots&\vdots\\
        0&0&\cdots&c_{nn}
    \end{pmatrix}\]
    这是一个上三角矩阵,并且各$c_{ii}\neq0$,于是系数行列式不为零.\\
    这样就证明了前述定理.
\end{proof}
把上述定理应用到有$n$个方程的$n$元齐次线性方程组上可以得到下面的推论.
\begin{lemma}
    $\F$上有$n$个方程的$n$元齐次方程线性组只有零解,当且仅当其系数行列式不等于$0$,从而有非零解当且仅当其系数行列式等于$0$.
\end{lemma}
\begin{theorem}[$n$个方程的$n$元线性方程组的解]
    将$n$个方程的$n$元线性方程组的系数矩阵$\mat{A}$的第$j$列换成常数项对应的列后形成的矩阵记作$\mat{B}_j$,即
    \[\mat{B}_j=\begin{pmatrix}
        a_{11}&\cdots&a_{1(j-1)}&b_1&a_{1(j+1)}&\cdots&a_{1n}\\
        \vdots&\ddots&\vdots&\vdots&\ddots&\vdots\\
        a_{n1}&\cdots&a_{n(j-1)}&b_n&a_{n(j+1)}&\cdots&a_{nn}
    \end{pmatrix}\]
    当该方程组的系数行列式$\det\mat{A}\neq0$时,其唯一解为
    \[x_j=\dfrac{\det\mat{B}_j}{\det\mat{A}},\forall 1\leqslant j\leqslant n\]
\end{theorem}
\begin{proof}
    我们只需验证上述解满足每一个方程即可.为此,考虑第$i$行方程
    \[\sum_{j=1}^{n}a_{ij}x_i=b_i\]
    将解代入可得
    \[\begin{aligned}
        LHS
        &= \sum_{j=1}^{n}a_{ij}x_j = \sum_{j=1}^{n}a_{ij}\dfrac{\det\mat{B}_j}{\det\mat{A}} = \dfrac{1}{\det\mat{A}}\sum_{j=1}^{n}a_{ij}\det\mat{B}_j \\
        &= \dfrac{1}{\det\mat{A}}\sum_{j=1}^{n}a_{ij}\left(\sum_{k=1}^{n}b_kA_{kj}\right) = \dfrac{1}{\det\mat{A}}\sum_{k=1}^{n}b_k\left(\sum_{j=1}^{n}a_{ij}A_{kj}\right)
    \end{aligned}\]
    根据前面的推论,当且仅当$k=i$时
    \[\sum_{j=1}^{n}a_{ij}A_{kj}=\det\mat{A}\]
    否则
    \[\sum_{j=1}^{n}a_{ij}A_{kj}=0\]
    于是
    \[LHS=\dfrac{1}{\det\mat{A}}\sum_{k=1}^{n}b_k\left(\sum_{j=1}^{n}a_{ij}A_{kj}\right)=\dfrac{1}{\det\mat{A}}b_i\det\mat{A}=b_i=RHS\]
\end{proof}
\begin{theorem}[Cramer法则]
    $\F$上有$n$个方程的$n$元线性方程组有唯一解,则其系数行列式不等于$0$,并且解为
    \[x_j=\dfrac{\det\mat{B}_j}{\det\mat{A}},\forall 1\leqslant j\leqslant n\]
\end{theorem}
\subsection{行列式按$k$行(列展开)}
\begin{definition}[$k$阶子式]
    $n$级方阵$\mat{A}$中任意取定$k$行$k$列($1\leqslant k<n$),位于这些行和列交叉处的元素按原来的顺序排成的$k$级方阵称为$\mat{A}$的一个\tbf{$k$阶子式}.
\end{definition}
\begin{definition}[$k$阶子式的余子式和代数余子式]
    取定$n$级方阵$\mat{A}$的第$\li i,k$行和第$\li j,k$列形成的$k$阶子式记作
    \[\mat{A}\begin{pmatrix}
        \li i,k\\\li j,k
    \end{pmatrix}\]
    划去这些行和列形成的$n-k$阶子式称作上述方阵的\tbf{余子式}.余子式前乘以系数
    \[(-1)^{\li i+k+\li j+k}\]
    称作上述方阵的\tbf{代数余子式}.\\
    令
    \[\left\{i_1',\cdots,i_{n-k}'\right\}=\{1,\cdots,n\}\backslash\left\{\li i,k\right\}\]
    \[\left\{j_1',\cdots,j_{n-k}'\right\}=\{1,\cdots,n\}\backslash\left\{\li j,k\right\}\]
    并且$i_1'<\cdots<i_{n-k}'$,$j_1'<\cdots<j_{n-k}'$,那么前述$k$阶子式的代数余子式可以记作
    \[\mat{A}\begin{pmatrix}
        i_1',\cdots,i_{n-k}'\\j_1',\cdots,j_{n-k}'
    \end{pmatrix}\]
\end{definition}
\begin{theorem}[Laplace定理按$k$行展开的版本]
    在$n$阶方阵$\mat{A}$中取定$\li i,k$行,这$k$行形成的所有$k$阶子式与它们的代数余子式的乘积之和等于$\det\mat{A}$,即
    \[\det\mat{A}=\sum_{1\leqslant\li j\leqslant k\leqslant n}\mat{A}\begin{pmatrix}
        \li i,k\\\li j,k
    \end{pmatrix}(-1)^{\li i+k+\li j+k}\mat{A}\begin{pmatrix}
        i_1',\cdots,i_{n-k}'\\j_1',\cdots,j_{n-k}'
    \end{pmatrix}\]
\end{theorem}
\begin{proof}
    给定行指标的排列$i_1\cdots i_ki_1'\cdots i_{n-k}'$,则$\det\mat{A}$可以展写如下
    \[\det\mat{A}=\sum_{\alpha_1\cdots\alpha_k\beta_1\cdots\beta_{n-k}}(-1)^{\tau\left(i_1\cdots i_ki_1'\cdots i_{n-k}'\right)+\tau\left(\alpha_1\cdots\alpha_k\beta_1\cdots\beta_{n-k}\right)}a_{i_1\alpha_1}\cdots a_{i_k\alpha_k}a_{i_1'\beta_1}\cdots a_{i_{n-k}'\beta_{n-k}}\]
    现在将上述$n!$项按照下面的方式分成$C_n^k$组:任意取定$\li j,k$列,将$n$的全排列$\alpha_1\cdots\alpha_k\beta_1\cdots\beta_{n-k}$对应于
    \[j_1\cdots j_kj_1'\cdots j_{n-k}'\]
    在研究排列的性质时,我们已经知道
    \[(-1)^{\tau\left(i_1\cdots i_ki_1'\cdots i_{n-k}\right)}=(-1)^{\tau\left(i_1\cdots i_k\right)+\tau\left(i_1'\cdots i_{n-k}'\right)}\cdot(-1)^{\li i+k}\cdot(-1)^{\frac{n(n+1)}{2}}\]
    \[(-1)^{\tau\left(j_1\cdots j_kj_1'\cdots j_{n-k}\right)}=(-1)^{\tau\left(j_1\cdots j_k\right)+\tau\left(j_1'\cdots j_{n-k}'\right)}\cdot(-1)^{\li j+k}\cdot(-1)^{\frac{n(n+1)}{2}}\]
    于是
    \[\begin{aligned}
        \det\mat{A}
        &=\sum_{1\leqslant\li j\leqslant k\leqslant n}\begin{array}{l}
            (-1)^{\tau\left(i_1\cdots i_k\right)+\tau\left(j_1\cdots j_k\right)}\cdot(-1)^{\tau\left(i_1'\cdots i_{n-k}'\right)+\tau\left(j_1'\cdots j_{n-k}'\right)}\cdot(-1)^{n(n+1)}\\
            \displaystyle\cdot(-1)^{\li i+k+\li j+k}\prod_{p=1}^{k}a_{i_pj_p}\prod_{q=1}^{n-k}a_{i_q'j_q'}
        \end{array}\\
        &= \sum_{1\leqslant\li j\leqslant k\leqslant n}\begin{array}{l}
            \displaystyle(-1)^{\tau\left(i_1\cdots i_k\right)+\tau\left(j_1\cdots j_k\right)}\prod_{p=1}^{k}a_{i_pj_p}\\
            \cdot(-1)^{\li i+k+\li j+k}\\
            \displaystyle\cdot(-1)^{\tau\left(i_1'\cdots i_{n-k}'\right)+\tau\left(j_1'\cdots j_{n-k}'\right)}\prod_{q=1}^{n-k}a_{i_q'j_q'}
        \end{array} \\
        &= \sum_{1\leqslant\li j\leqslant k\leqslant n}\mat{A}\begin{pmatrix}
            \li i,k\\\li j,k
        \end{pmatrix}(-1)^{\li i+k+\li j+k}\mat{A}\begin{pmatrix}
            i_1',\cdots,i_{n-k}'\\j_1',\cdots,j_{n-k}'
        \end{pmatrix}
    \end{aligned}\]
    于是命题得证.显然,当$k=1$时,就是我们前面所述的Laplace定理.
\end{proof}
\begin{lemma}
    我们有
    \[\begin{pmatrix}
        \mat{A}&\tbf{0}\\
        \mat{C}&\mat{B}
    \end{pmatrix}=\det\mat{A}\cdot\det\mat{B}\]
    其中$\mat{A}$和$\mat{B}$分别为$k$级和$n-k$级方阵,$\mat{C}$为$(n-k)\times k$级的任意矩阵,$\tbf{0}$为零矩阵.
\end{lemma}
\begin{proof}
    将上述矩阵按前$k$行展开,仅有$\mat{A}$这一子式对应的求和项中不包含$0$,它对应的余子式恰为$\det\mat{B}$,并且指数项为$k(k+1)$,因此上述结论成立.
\end{proof}
\end{document}