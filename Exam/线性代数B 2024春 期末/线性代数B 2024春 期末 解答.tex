\documentclass{ctexart}
\usepackage{note}
\begin{document}\pagestyle{empty}
\begin{center}
    \tbf{\Large 北京大学数学科学学院2023-24学年第二学期线性代数B期末试题}
\end{center}
\begin{homework}[1(15')]
    设矩阵$\mat{A}=\begin{bmatrix}
        1&1&-1\\
        x&-2&0\\
        -1&-1&y
    \end{bmatrix}$与矩阵$\mat{B}=\begin{bmatrix}
        1&0&z\\
        0&-1&0\\
        z&0&0
    \end{bmatrix}$相似.
    \begin{enumerate}
        \item 求$x,y,z$的值.
        \item 求可逆矩阵$\mat{P}$使得$\mat{P}^{-1}\mat{A}\mat{P}=\mat{B}$.
        \item 求$\mat{A}^m$,其中$m\in\N^+$.
    \end{enumerate}
\end{homework}
\begin{solution}
\begin{enumerate}
    \item 相似的矩阵具有相同的特征多项式.对于$\mat{A}$有
    \[\det(\lambda\mat{I}-\mat{A})=\begin{vmatrix}
        \lambda-1&-1&1\\
        -x&\lambda+2&0\\
        1&1&\lambda-y
    \end{vmatrix}=\lambda^3+(1-y)\lambda^2-(x+y+3)\lambda+(x+2)(y-1)\]
    对于$\mat{B}$有
    \[\det(\lambda\mat{I}-\mat{B})=\begin{vmatrix}
        \lambda-1&0&-z\\
        0&\lambda+1&0\\
        -z&0&\lambda
    \end{vmatrix}=(\lambda+1)(\lambda^2-\lambda-z^2)=\lambda^3-(1+z^2)\lambda-z^2\]
    于是
    \[\left\{\begin{array}{l}
        1-y=0\\
        x+y+3=1+z^2\\
        (x+2)(y-1)=-z^2
    \end{array}\right.\]
    解得
    \[x=-3,\quad y=1,\quad z=0\]
    \item 此时$\mat{B}$为对角矩阵.\\
    考虑$\mat{A}$对应于特征值$1$的特征向量,齐次线性方程组$(1\mat{I}-\mat{A})$的一个解为
    \[\bs\alpha_1=\begin{bmatrix}
        -1&1&1
    \end{bmatrix}^\t\]
    同理, $\mat{A}$对应于特征值$-1$的一个特征向量为$\bs\alpha_2=\begin{bmatrix}
        -1&3&1
    \end{bmatrix}^\t$,对应于特征值$0$的一个特征向量为$\bs\alpha_3=\begin{bmatrix}
        -2&3&1
    \end{bmatrix}$.于是令
    \[\mat{P}=\begin{bmatrix}
        \bs\alpha_1&\bs\alpha_2&\bs\alpha_3
    \end{bmatrix}=\begin{bmatrix}
        -1&-1&-2\\
        1&3&3\\
        1&1&1
    \end{bmatrix}\]
    即可使得$\mat{P}^{-1}\mat{A}\mat{P}=\mat{B}$.
    \item 我们有
    \[\mat{A}^m=(\mat{P}\mat{B}\mat{P}^{-1})^m=\mat{P}\mat{B}^m\mat{P}^{-1}\]
    而$\mat{B}^3=\mat{B}$,于是对于任意奇数$m$都有$\mat{A}^m=\mat{A}$,对于任意偶数$m$都有$\mat{A}^m=\mat{A}^2=\begin{bmatrix}
        -1&0&2\\
        3&1&3\\
        1&0&2
    \end{bmatrix}$.
\end{enumerate}
\end{solution}
\begin{homework}[2(10')]
    设$t\in\R$,求实二次型
    \[q(x_1,x_2,x_3)=x_1^2+x_2^2+tx_1x_2+2tx_1x_3+4x_2x_3\]
    的秩和正惯性指数.
\end{homework}
\begin{solution}
    该二次型的矩阵为
    \[\mat{A}=\begin{bmatrix}
        1&\frac{t}{2}&t\\
        \frac{t}{2}&1&2\\
        t&2&0
    \end{bmatrix}\]
    顺序主子式依次为$1,1-\dfrac{t^2}{4},t^2-4$.\\
    当$t^2\neq4$时,矩阵的秩为$3$,考虑其特征多项式
    \[\det(\lambda\mat{I}-\mat{A})=\lambda^3-2\lambda^2-\left(\dfrac54t^2+3\right)\lambda+(4-t^2)\]
    根据韦达定理,设$\mat{A}$的特征值分别为$\lambda_1,\lambda_2,\lambda_3$,则有
    \[\lambda_1+\lambda_2+\lambda_3=2,\quad \lambda_1\lambda_2+\lambda_1\lambda_3+\lambda_2\lambda_3=-\left(\dfrac54t^2+3\right),\quad x_1x_2x_3=t^2-4\]
    于是当$t^2>4$时, $\mat{A}$的特征值两负一正,正惯性指数为$1$;当$t^2<4$时, $\mat{A}$的正惯性指数两正一负,正惯性指数为$2$.\\
    当$t^2=4$时,容易验证$\mat{A}$的秩为$2$,此时特征多项式为$\lambda^3-2\lambda^2-8\lambda$,正惯性指数为$1$.
\end{solution}
\begin{homework}[3(10')]
    求$a$满足的条件使得实二次型
    \[f(x_1,x_2,x_3)=a(x_1^2+x_2^2+x_3^2)+2(x_1x_2+x_1x_3+x_2x_3)\]
    正定.
\end{homework}
\begin{solution}
    该二次型的矩阵为
    \[\mat{A}=\begin{bmatrix}
        a&1&1\\
        1&a&1\\
        1&1&a
    \end{bmatrix}\]
    其顺序主子式依次为$a,(a+1)(a-1),(a+2)(a-1)^2$.由于上述二次型正定,因此所有顺序主子式均为正数,从而$a>1$.
\end{solution}
\begin{homework}[4(10')]
    设$V=\K^5$, $V_1=\text{span}(\bs\alpha_1,\bs\alpha_2,\bs\alpha_3)$, $V_2=\text{span}(\bs\beta_1,\bs\beta_2,\bs\beta_3)$,其中
    \[\bs\alpha_1=\begin{bmatrix}
        1\\0\\-1\\1\\2
    \end{bmatrix},\quad\bs\alpha_2=\begin{bmatrix}
        0\\1\\1\\1\\0
    \end{bmatrix},\quad\bs\alpha_3=\begin{bmatrix}
        2\\2\\1\\1\\-1
    \end{bmatrix},\quad\bs\beta_1=\begin{bmatrix}
        3\\1\\2\\4\\6
    \end{bmatrix},\quad\bs\beta_2=\begin{bmatrix}
        0\\3\\4\\6\\1
    \end{bmatrix},\quad\bs\beta_3=\begin{bmatrix}
        -1\\0\\-1\\-1\\2
    \end{bmatrix}\]
    分别求$V_1+V_2$, $V_1\cap V_2$的维数和一个基.
\end{homework}
\begin{solution}
    由于
    \[\begin{vmatrix}
        1&0&2\\0&1&2\\-1&1&1
    \end{vmatrix}=1\neq0,\quad\begin{bmatrix}
        3&0&-1\\1&3&0\\2&4&-1
    \end{bmatrix}=-7\neq0\]
    于是$\dim V_1=\dim V_2=3$.考虑
    \[\begin{vmatrix}
        \bs\alpha_1&\bs\alpha_2&\bs\alpha_3&\bs\alpha_4&\bs\alpha_5
    \end{vmatrix}=\begin{vmatrix}
        1&0&2&3&0\\
        0&1&2&1&3\\
        -1&1&1&2&4\\
        1&1&1&4&6\\
        2&0&-1&6&1
    \end{vmatrix}=\begin{vmatrix}
        1&2&1&3\\
        1&3&5&4\\
        1&-1&1&6\\
        0&-5&0&1
    \end{vmatrix}=\begin{vmatrix}
        1&4&1\\
        -3&0&3\\
        -5&0&1
    \end{vmatrix}=48\neq0\]
    于是$\dim(V_1+V_2)=5$,其一组基为$\bs\alpha_1,\bs\alpha_2,\bs\alpha_3,\bs\alpha_4,\bs\alpha_5$.而
    \[\dim(V_1\cap V_2)=\dim V_1+\dim V_2-\dim(V_1+V_2)=1\]
    考虑$x_1\bs\alpha_1+x_2\bs\alpha_2+x_3\bs\alpha_3+x_4\bs\beta_1+x_5\bs\beta_2+x_6\bs\beta_3=\mbf0$,这线性方程组的一组解为
    \[\begin{bmatrix}
        -1&-5&1&0&1&1
    \end{bmatrix}^\t\]
    于是$V_1\cap V_2$的一组基为
    \[\begin{bmatrix}
        -1&3&3&5&3
    \end{bmatrix}^\t\]
\end{solution}
\begin{homework}[5(15')]
    设$\mat{A},\mat{B}\in M_n(\K)$是$n$级矩阵,其中$\mat{B}=(b_{ij})$是严格上三角矩阵,满足$i\geq j$时$b_{ij}=0$.令
    \[f(\mat{X})=\mat{A}\mat{X}-\mat{X}\mat{B},\quad\forall\mat{X}\in M_n(\K)\]
    \begin{enumerate}
        \item 证明$f$是$M_n(\K)$上的一个线性变换.
        \item 求$f$在基$\mat{E}_{11},\mat{E}_{21},\cdots,\mat{E}_{n1},\mat{E}_{12},\mat{E}_{22},\cdots,\mat{E}_{n2},\cdots,\mat{E}_{nn}$下的矩阵,这里$\mat{E}_{ij}$是只有$(i,j)$元为$1$,其余元素为$0$的基本矩阵.
        \item 如果$\mat{A}$可逆,证明:对任意$\mat{C}\in M_n(\K)$,存在唯一的$\mat{X}$使得$f(\mat{X})=\mat{C}$.
    \end{enumerate}
\end{homework}
\begin{solution}
\begin{enumerate}
    \item 对于任意$\mat{X},\mat{Y}\in M_n(\K)$和$k\in\K$有
    \[f(\mat{X}+\mat{Y})=\mat{A}(\mat{X}+\mat{Y})-(\mat{X}+\mat{Y})\mat{B}=\mat{A}\mat{X}-\mat{X}\mat{B}+\mat{A}\mat{Y}-\mat{Y}\mat{B}=f(\mat{X})+f(\mat{Y})\]
    \[f(k\mat{X})=\mat{A}(k\mat{X})-(k\mat{X})\mat{B}=k\mat{A}\mat{X}-k\mat{X}\mat{B}=kf(\mat{X})\]
    于是$f$是$M_n(\K)$上的线性变换.
    \item 设$f$在这组基下的矩阵$\mat{M}=(m_{ij})$,设$\mat{A}=(a_{ij})$.于是有
    \[f(\mat{E}_{ij})=\mat{A}\mat{E}_{ij}-\mat{E}_{ij}\mat{B}=\sum_{k=1}^{n}a_{ki}\mat{E}_{kj}-\sum_{k=1}^{n}b_{jk}\mat{E}_{ik}=\sum_{k=1}^{n}a_{ki}\mat{E}_{kj}-\sum_{k=j+1}^{n}b_{ki}\mat{E}_{kj}\]
    \item 首先证明唯一性.如果这样的$\mat{X}$不唯一,那么总存在非零的$\mat{X}$使得
    \[\mat{A}\mat{X}=\mat{X}\mat{B}\]
    设$\mat{X}=\begin{bmatrix}
        \vec{x}_1&\cdots&\vec{x}_k
    \end{bmatrix}$,则有
    \[\mat{A}\vec{x}_i=\sum_{k=1}^{n}b_{ki}\vec{x}_k=\sum_{k<i}b_{ki}\vec{x}_k\]
    当$i=1$时可得$\mat{A}\vec{x}_1=\mbf0$,又因为$\mat{A}$可逆,于是$\vec{x}_1=\mbf0$.\\
    当$i=2$时可得$\mat{A}\vec{x}_2=b_{12}\vec{x}_1=\mbf0$,同理可得$\vec{x}_2=\mbf0$.于是如此推断可得$\vec{x}_1=\cdots=\vec{x}_n=\mbf0$,这与$\mat{X}$非零矛盾,于是满足条件的$\mat{X}$是唯一的.\\
    现在证明存在性.考虑$\mat{C}=\begin{bmatrix}
        \vec{c}_1&\cdots\vec{c}_n
    \end{bmatrix}$.依题意可得
    \[\mat{A}\vec{x}_i-\sum_{k<i}b_{ki}\vec{x}_k=\vec{c}_i\]
    于是令
    \[\vec{x}_{i}=\mat{A}^{-1}\left(\vec{c}_i+\sum_{k<i}b_{ki}\vec{x}_k\right)\]
    依次令$i=1,2,\cdots,n$即可构造出符合题意的$\mat{X}$.
\end{enumerate}
\end{solution}
\begin{homework}[6(14')]
    设$n$级矩阵$\mat{A}$和$\mat{B}$有相同的特征值.
    \begin{enumerate}
        \item 如果$\mat{A}$有$n$个互不相同的特征值,证明:存在$n$级可逆矩阵$\mat{P}$以及$n$级矩阵$\mat{Q}$使得$\mat{A}=\mat{P}\mat{Q}$, $\mat{B}=\mat{Q}\mat{P}$.
        \item 如果把前提条件$\mat{A}$有$n$个互不相同的特征值去掉,上一小题的结论是否成立?如果成立,请给出证明;如果不成立,请给出反例.
    \end{enumerate}
\end{homework}
\begin{solution}
\begin{enumerate}
    \item 由于$\mat{A},\mat{B}$均有$n$个互不相同的特征值,且两者的特征值相同,于是二者具有相同的特征多项式$f(\lambda)=\displaystyle\prod_{i=1}^{n}(\lambda-\lambda_i)$,从而二者相似.于是存在可逆矩阵$\mat{P}$使得$\mat{B}=\mat{P}^{-1}\mat{A}\mat{P}$.令$\mat{Q}=\mat{P}^{-1}\mat{A}$,则有
    \[\mat{A}=\mat{P}\mat{P}^{-1}\mat{A}=\mat{P}\mat{Q},\quad\mat{B}=\mat{P}^{-1}\mat{A}\mat{P}=\mat{Q}\mat{P}\]
    于是命题得证.
    \item 不成立.如果$\mat{A}$没有$n$个互不相同的特征值,那么$\mat{A}$与$\mat{B}$的特征多项式的根的重数可能不同,也即两者可能不相似,此时就无法推出题设的结论.
\end{enumerate}
\end{solution}
\begin{homework}[7(10')]
    设矩阵$\mat{A},\mat{B},\mat{C},\mat{D}\in M_n(\K)$两两可交换,且满足$\mat{A}\mat{C}+\mat{B}\mat{D}=\mat{I}$.设齐次线性方程组$\mat{A}\mat{B}\vec{x}=\mbf0$的解空间为$V$, $\mat{B}\vec{x}=\mbf0$的解空间为$V_1$, $\mat{A}\vec{x}=\mbf0$的解空间为$V_2$.证明: $V=V_1\oplus V_2$.
\end{homework}
\begin{proof}
    首先证明$V_1\cap V_2=\{\mbf0\}$.假定存在非零的$\vec{x}\in V_1\cap V_2$,则有
    \[\mat{A}\vec{x}=\mat{B}\vec{x}=\mbf0\]
    由于$\mat{A},\mat{B},\mat{C},\mat{D}$两两可交换,于是$\mat{C}\mat{A}+\mat{D}\mat{B}=\mat{I}$,于是则有
    \[\vec{x}=\mat{I}\vec{x}=(\mat{C}\mat{A}+\mat{D}\mat{B})\vec{x}=\mat{C}(\mat{A}\vec{x})+\mat{D}(\mat{B}\vec{x})=\mat{C}\mbf0+\mat{D}\mbf0=\mbf0\]
    这与$\vec{x}$非零矛盾,从而$V_1\cap V_2=\{\mbf0\}$.\\
    根据解空间的维数定理有
    \[\dim V_1=n-\rank\mat{B},\quad \dim V_2=n-\rank\mat{A},\quad\dim V=n-\rank\mat{A}\mat{B}\]
    根据Sylvester秩不等式,总有
    \[\rank\mat{A}\mat{B}\geq\rank\mat{A}+\rank\mat{B}-n\]
    即
    \[\dim V_1+\dim V_2\geq\dim V\]
    既然$V_1,V_2$都是$V$的子空间,并且$V_1\cap V_2=\{\mbf0\}$,于是$V_1\oplus V_2=V$.
\end{proof}
\begin{homework}[8(16')]
    考虑欧几里得空间$\R[x]_3=\{a+bx+cx^2:a,b,c\in\R\}$,其上内积的定义为
    \[\inprod{f}{g}=\int_{-1}^{1}f(x)g(x)\di x\]
    \begin{enumerate}
        \item 求$\alpha,\beta,\gamma$的值使得下述向量组是正交基:
        \[p_1(x)=1,\quad p_2(x)=\alpha+x,\quad p_3(x)=\beta+\gamma x+x^2\]
        \item 求二次首一多项式$r(x)=a+bx+x^2$的长度最小值.
    \end{enumerate}
\end{homework}
\end{document}