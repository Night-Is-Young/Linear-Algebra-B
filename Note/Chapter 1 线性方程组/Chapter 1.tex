\documentclass{ctexart}
\usepackage{note}
\begin{document}
\section{线性方程组}
\subsection{求解线性方程组}
\subsubsection{线性方程组的相关定义}
\begin{definition}[线性方程组]
    形如
    \[\left\{\begin{array}{c}
        a_{11}x_1+a_{12}x_2+\cdots+a_{1m}x_m=b_1\\
        a_{21}x_1+a_{22}x_2+\cdots+a_{2m}x_m=b_2\\
        \cdots\\
        a_{n1}x_1+a_{n2}x_2+\cdots+a_{nm}x_m=b_n
    \end{array}\right.\]
    的方程称作\tbf{$k$元线性方程组},其中$a_{11},a_{12},\cdots,a_{nk}$为系数,$\li b,n$为常数项.
\end{definition}
\begin{definition}[增广矩阵和系数矩阵]
    上述线性方程组的\tbf{增广矩阵}为
    \[\begin{pmatrix}
        a_{11}  &   a_{12}  &   \cdots  &   a_{1m}  & b_1\\
        a_{21}  &   a_{22}  &   \cdots  &   a_{2m}  & b_2\\
        \vdots&\vdots&\ddots&\vdots&\vdots\\
        a_{n1}  &   a_{n2}  &   \cdots  &   a_{nm}  & b_n
    \end{pmatrix}\]
    \tbf{系数矩阵}为
    \[\begin{pmatrix}
        a_{11}  &   a_{12}  &   \cdots  &   a_{1m}\\
        a_{21}  &   a_{22}  &   \cdots  &   a_{2m}\\
        \vdots&\vdots&\ddots&\vdots\\
        a_{n1}  &   a_{n2}  &   \cdots  &   a_{nm}
    \end{pmatrix}\]
\end{definition}
\subsubsection{线性方程组的解法}
\begin{theorem}[解线性方程组的操作]
    我们一般对线性方程组做如下变换:
    \begin{enumerate}[label=\arabic*.,topsep=0pt,parsep=0pt,itemsep=0pt,partopsep=0pt]
        \item 把一个方程的倍数加到另一个方程上.
        \item 互换两个方程的位置.
        \item 用一个非零的数乘某一个方程.
    \end{enumerate}
    上述操作被称为\tbf{线性方程组的初等变换},对应的在矩阵中对行的操作被称为\tbf{初等行变换}.
\end{theorem}
\begin{definition}[阶梯形矩阵和简化行阶梯形矩阵]
    \tbf{阶梯形矩阵}应当满足下述条件:
    \begin{enumerate}[label=\arabic*.,topsep=0pt,parsep=0pt,itemsep=0pt,partopsep=0pt]
        \item 元素全为$0$的行(即\tbf{零行})在下方.
        \item 元素不全为$0$的行(即\tbf{非零行}),从左起第一个不为$0$的元素(称\tbf{主元})的列指标随行指标的增大而严格增大.
    \end{enumerate}
    通俗而言,阶梯形矩阵的各行左起连续为$0$的元素数目是随行指标严格递增的.例如
    \[\begin{pmatrix}
        1&3&1&2\\0&1&-1&-3\\0&0&3&6\\0&0&0&0
    \end{pmatrix}\]
    即为阶梯形矩阵.阶梯形矩阵和上三角矩阵的定义是有些类似的,但上三角矩阵一定是方阵,而阶梯形矩阵不一定.\\
    一种特殊的阶梯形矩阵,即\tbf{简化行阶梯形矩阵},应当满足如下条件:
    \begin{enumerate}[label=\arabic*.,topsep=0pt,parsep=0pt,itemsep=0pt,partopsep=0pt]
        \item 是阶梯形矩阵.
        \item 每个非零行的主元均为$1$.
        \item 每个主元所在列的其余元素均为$0$.
    \end{enumerate}
    例如
    \[\begin{pmatrix}
        1&0&0&3\\0&1&0&-1\\0&0&1&2\\0&0&0&0
    \end{pmatrix}\]
    即为简化行阶梯形矩阵.不难看出,简化行阶梯形矩阵直接对应于线性方程组的解.
\end{definition}
于是,只要对线性方程组施以初等变换,即可得到解或者判断出解的存在性.
\subsection{线性方程组的解的情况与判别准则}
\subsubsection{阶梯形矩阵的必然存在性}
\begin{theorem}[阶梯形矩阵的必然存在性]
    任一矩阵都能经初等行变换为阶梯形矩阵.
\end{theorem}
\begin{proof}
    我们现在通过数学归纳法证明上述命题.\\
    零矩阵按定义是阶梯形矩阵.现在考虑非零矩阵,对行数$m$做归纳.\\
    当$m=1$时,该矩阵一定是阶梯形矩阵.\\
    当$m>1$时,假定$m-1$行矩阵可以经初等行变换为阶梯形矩阵.考虑$m$行的矩阵$\textit{\tbf{A}}$,其$(i,j)$元记为$a_{ij}$.\\
    如果$\textit{\tbf{A}}$的第一列不全为$0$,那么可以通过交换使得$a_{11}\neq0$,因此不妨直接假设$a_{11}\neq0$.对于任意$2\leqslant i\leqslant m$,将第一行的$-\dfrac{a_{i1}}{a_{11}}$倍加到第$i$行上.于是,变换后的矩阵$\textit{\tbf{J}}_1$的第一列除$a_{11}$外将变为$0$,即
    \[\textit{\tbf{J}}_1=\begin{pmatrix}
        a_{11}&a_{12}&\cdots&a_{1n}\\
        0&a_{22}-\dfrac{a_{21}}{a_{11}}a_{12}&\cdots&a_{2n}-\dfrac{a_{21}}{a_{11}}a_{1n}\\
        \vdots&\vdots&\ddots&\vdots\\
        0&a_{m2}-\dfrac{a_{m1}}{a_{11}}a_{12}&\cdots&a_{mn}-\dfrac{a_{m1}}{a_{11}}a_{1n}
    \end{pmatrix}\]
    注意到这一矩阵除去第一行和第一列之外即为$m-1$行的矩阵,按照归纳假设,它可以通过初等行变换为阶梯形矩阵$\textit{\tbf{K}}_1$.于是,$\textit{\tbf{A}}$可以经初等行变换为下面的矩阵:
    \[\textit{\tbf{A}}\rightarrow\begin{pmatrix}
        a_{11}&a_{12}&\cdots&a_{1n}\\
        0&&&\\
        \vdots&&\Large{\textit{\tbf{K}}_1}&\\
        0&&&
    \end{pmatrix}\]
    依定义,上述右边的矩阵也是阶梯形矩阵.\\
    如果$\textit{\tbf{A}}$的第一列全为$0$,那么就忽略首列直到出现不全为$0$的列为止,记此时的矩阵为$\textit{\tbf{B}}$.根据前面的讨论,$\textit{\tbf{B}}$作为$m$行矩阵也是可以变换为阶梯形矩阵的,记变换后的矩阵为$\textit{\tbf{K}}_2$.于是,$\textit{\tbf{A}}$可以经历如下变换:
    \[\textit{\tbf{A}}=\begin{pmatrix}
        0&\cdots&0&a_{1k}&\cdots&a_{1n}\\
        \vdots&\ddots&\vdots&\vdots&\ddots&\vdots\\
        0&\cdots&0&a_{mk}&\cdots&a_{mn}
    \end{pmatrix}=\begin{pmatrix}
        0&\cdots&0&\\
        \vdots&\ddots&\vdots&\Large{\textit{\tbf{B}}}\\
        0&\cdots&0
    \end{pmatrix}\rightarrow\begin{pmatrix}
        0&\cdots&0&\\
        \vdots&\ddots&\vdots&\Large{\textit{\tbf{K}}_2}\\
        0&\cdots&0
    \end{pmatrix}\]
    依定义,上述右边的矩阵也是阶梯形矩阵.\\
    于是$m$行的矩阵$\textit{\tbf{A}}$可以通过初等行变换成阶梯形矩阵.\\
    于是对上述结论归纳可得:所有矩阵都可以通过初等行变换成阶梯形矩阵.
\end{proof}
\begin{theorem}[简化阶梯形矩阵的必然存在性]
    任一矩阵都能经初等行变换为简化阶梯形矩阵.
\end{theorem}
\begin{proof}
    这是简单的.在变换成阶梯形矩阵的基础上,自下而上的对矩阵消元即可.
\end{proof}
\subsubsection{线性方程的解的情况及其判别准则}
\begin{theorem}[线性方程的解的情况及其判别准则]
    系数和常数项为有理数(或实数,或复数)的$n$元线性方程组的解的情况有且仅有三种:\tbf{无解},\tbf{有唯一解},\tbf{有无穷多解}.\\
    把$n$元线性方程组的增广矩阵经初等行变换为阶梯形矩阵,如果出现主元在最后一列(即出现$0=d$型的方程)则原方程无解;否则有解.\\当有解时,如果阶梯形矩阵的非零行数目$r$等于未知量数目$n$,那么原方程组有唯一解;如果$r<n$,那么原方程组有无穷多解.
\end{theorem}
上述解线性方程组的办法称作\tbf{Gauss-Jordan算法}.
\begin{definition}[齐次线性方程组]
    常数项全为$0$的线性方程组称作齐次线性方程组.
\end{definition}
\begin{lemma}[齐次线性方程组有解的充要条件]
    $n$元齐次线性方程组有非零解,当且仅当它的系数矩阵经初等行变换成的阶梯形矩阵中,非零行的数目$r<n$.
\end{lemma}
\begin{proof}
    $n$元齐次线性方程组必然存在$(0,\cdots,0)'$这一组解.因此,当且仅当方程组有无穷多解时才存在非零解.
\end{proof}
\begin{lemma}[齐次线性方程组有解的充分条件]
    当$n$元齐次线性方程组的方程数目$s$小于未知量的数目$n$,那么它有非零解.
\end{lemma}
\begin{proof}
    注意到有$r\leqslant s<n$,故得证.
\end{proof}
\subsection{数域}
\begin{definition}[数域]
    如果集合$K\subseteq\mathbb{C}$满足
    \begin{enumerate}[label=\arabic*.,topsep=0pt,parsep=0pt,itemsep=0pt,partopsep=0pt]
        \item $0,1\in K$.
        \item $\forall a,b\in K,a\pm b\in K$且$ab\in K$.
        \item $\forall a,b\in K$且$b\neq0,\dfrac{a}{b}\in K$.
    \end{enumerate}
    那么称$K$是一个\tbf{数域}.
\end{definition}
通俗而言,数域就是对四则运算封闭的集合.特别地,上面第二条和第三条的运算都是$\mathbb{C}$中的运算.下面是一个典型的反例.
\begin{problem}
    令$S=\{0,1\}$,并令$0+1=1+0=1,0+0=1+1=0,1\times1=1,0\times0=1\times1=0$.说明$S$不是数域.
\end{problem}
\begin{proof}
    尽管$S\subset\mathbb{C}$,但上述定义的加法与$\mathbb{C}$上定义的不同,于是$S$不是数域.事实上,$S$是域的一种,称作\tbf{有限域}.
\end{proof}
\begin{problem}
    令
    \[F=\left\{\dfrac{a_0+a_1\e+\cdots+a_n\e^n}{b_0+b_1\e+\cdots+b_m\e^m}:n,m\in\N,a_i,b_j\in\mathbb{Z},\text{其中}0\leqslant i\leqslant n,0\leqslant j\leqslant m\right\}\]
    试证明$F$是数域.
\end{problem}
\begin{proof}
    首先有
    \[0=\dfrac{0}{1}\in F\]
    \[1=\dfrac{1}{1}\in F\]
    现在设
    \[\alpha=\dfrac{a_0+a_1\e+\cdots+a_n\e^n}{b_0+b_1\e+\cdots+b_m\e^m}\]
    \[\beta=\dfrac{c_0+c_1\e+\cdots+c_p\e^p}{d_0+d_1\e+\cdots+d_q\e^q}\]
    于是
    \[\begin{aligned}
        \alpha\pm\beta
        &= \dfrac{\left(a_0+\cdots+a_n\e^n\right)\left(d_0+\cdots+d_q\e^q\right)\pm\left(b_0+\cdots+b_m\e^m\right)\left(c_0+\cdots+c_p\e^p\right)}{\left(b_0+b_1\e+\cdots+b_m\e^m\right)\left(d_0+d_1\e+\cdots+d_q\e^q\right)} \\
        &= \dfrac{\displaystyle\sum_{i=0}^{n+q}\left(\e^{i}\sum_{j=\max\{0,i-q\}}^{\min\{i,n\}}a_jd_{i-j}\right)\pm\sum_{i=0}^{m+p}\left(\e^{i}\sum_{j=\max\{0,i-p\}}^{\min\{i,m\}}b_jc_{i-j}\right)}{\displaystyle\sum_{i=0}^{m+q}\left(\e^{i}\sum_{j=\max\{0,i-q\}}^{\min\{i,m\}}b_jd_{i-j}\right)}\\
        &= \dfrac{\displaystyle\sum_{i=0}^{n+q}A_i\e^i\pm\sum_{i=0}^{m+p}B_i\e^i}{\displaystyle\sum_{i=0}^{m+q}C_i\e^i}
        = \dfrac{\displaystyle\sum_{i=0}^{\max\{n+q,m+p\}}\left(A_i\pm B_i\right)\e^i}{\displaystyle\sum_{i=0}^{m+q}C_i\e^i}\in F
    \end{aligned}\]
    当$\beta=0$时$\alpha\beta=0\in F$.当$\beta\neq0$时有
    \[\dfrac{1}{\beta}=\dfrac{d_0+d_1\e+\cdots+d_q\e^q}{c_0+c_1\e+\cdots+c_p\e^p}\in F\]
    因此只需讨论乘法即可.我们有
    \[\begin{aligned}
        \alpha\beta
        &= \dfrac{a_0+a_1\e+\cdots+a_n\e^n}{b_0+b_1\e+\cdots+b_m\e^m}\cdot\dfrac{c_0+c_1\e+\cdots+c_p\e^p}{d_0+d_1\e+\cdots+d_q\e^q} \\
        &= \dfrac{\displaystyle\sum_{i=0}^{n+p}\left(\e^{i}\sum_{j=\max\{0,i-p\}}^{\min\{i,n\}}a_jc_{i-j}\right)}{\displaystyle\sum_{i=0}^{m+q}\left(\e^{i}\sum_{j=\max\{0,i-q\}}^{\min\{i,m\}}b_jd_{i-j}\right)}
        =\dfrac{\displaystyle\sum_{i=0}^{n+p}D_i\e^i}{\displaystyle\sum_{i=0}^{m+q}C_i\e^i}\in F
    \end{aligned}\]
    综上,$F$是数域.
\end{proof}
\end{document}