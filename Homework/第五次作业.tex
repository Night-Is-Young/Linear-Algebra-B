\documentclass{ctexart}
\usepackage{note}
\title{线性代数B  第五次作业}
\author{蒋锦豪 2400011785}
\date{}
\begin{document}
\maketitle
4.1: 3, 4 (1)(12), 5, 8, 10; 4.2: 2, 3, 4, 7; 4.3: 3, 5, 7; 4.4: 1, 3 (2), 4, 5, 7, 8, 9 (1), 10 (2).
\section*{习题4.1}
\begin{homework}[3]
    设$\mat{I}$是$n$阶单位矩阵, $\mat{J}$是元素全为$1$的$n$阶矩阵,设
    \[\mat{M}=\begin{bmatrix}
        k&\lambda&\cdots&\lambda\\
        \lambda&k&\cdots&\lambda\\
        \vdots&\vdots&\ddots&\vdots\\
        \lambda&\lambda&\cdots&k
    \end{bmatrix}\]
    试把$\mat{M}$表示成$x\mat{I}+y\mat{J}$的形式.
\end{homework}
\begin{solution}
    不难发现
    \[\mat{M}=\begin{bmatrix}
        \lambda&\lambda&\cdots&\lambda\\
        \lambda&\lambda&\cdots&\lambda\\
        \vdots&\vdots&\ddots&\vdots\\
        \lambda&\lambda&\cdots&\lambda
    \end{bmatrix}+\begin{bmatrix}
        k-\lambda&&&\\
        &k-\lambda&&\\
        &&\ddots&\\
        &&&k-\lambda
    \end{bmatrix}=(k-\lambda)\mat{I}+\lambda\mat{J}\]
    于是可知
    \[x=k-\lambda,\quad y=\lambda\]
\end{solution}
\begin{homework}[4(1)]
    计算
    \[\begin{bmatrix}
        7&-1\\-2&5\\3&-4
    \end{bmatrix}\begin{bmatrix}
        1&4\\-5&2
    \end{bmatrix}\]
\end{homework}
\begin{solution}
    \[\text{原式}=\begin{bmatrix}
        7\cdot\begin{bmatrix}1&4\end{bmatrix}+(-1)\begin{bmatrix}-5&2\end{bmatrix}\\
        -2\cdot\begin{bmatrix}1&4\end{bmatrix}+5\begin{bmatrix}-5&2\end{bmatrix}\\
        3\cdot\begin{bmatrix}1&4\end{bmatrix}+(-4)\begin{bmatrix}-5&2\end{bmatrix}
    \end{bmatrix}=\begin{bmatrix}
        12&26\\-27&2\\23&4
    \end{bmatrix}\]
\end{solution}
\begin{homework}[4(12)]
    计算
    \[\begin{bmatrix}
        a_1&a_2&a_3\\
        b_1&b_2&b_3\\
        c_1&c_2&c_3
    \end{bmatrix}\begin{bmatrix}
        1&0&0\\
        k&1&0\\
        0&0&1
    \end{bmatrix}\]
\end{homework}
\begin{solution}
    \[\text{原式}
        =\begin{bmatrix}
        a_1&a_2&a_3\\
        b_1&b_2&b_3\\
        c_1&c_2&c_3
    \end{bmatrix}\left(\begin{bmatrix}
        1&0&0\\
        0&1&0\\
        0&0&1
    \end{bmatrix}+\begin{bmatrix}
        0&0&0\\
        k&0&0\\
        0&0&0
    \end{bmatrix}\right)\\
    =\begin{bmatrix}
        a_1&a_2&a_3\\
        b_1&b_2&b_3\\
        c_1&c_2&c_3
    \end{bmatrix}+\begin{bmatrix}
        ka_2&0&0\\
        kb_2&0&0\\
        kc_2&0&0
    \end{bmatrix}=\begin{bmatrix}
        a_1+ka_2&a_2&a_3\\
        b_1+kb_2&b_2&b_3\\
        c_1+kc_2&c_2&c_3
    \end{bmatrix}\]
\end{solution}
\begin{homework}[5]
    设矩阵
    \[\mat{A}=\begin{bmatrix}
        1&2\\3&4
    \end{bmatrix},\quad\mat{B}=\begin{bmatrix}
        5&6\\7&8
    \end{bmatrix}\]
    求$\mat{A}\mat{B}$, $\mat{B}\mat{A}$, $\mat{A}\mat{B}-\mat{B}\mat{A}$.
\end{homework}
\begin{solution}
    \[\mat{A}\mat{B}=\begin{bmatrix}
        1\cdot\begin{bmatrix}5&6\end{bmatrix}+2\cdot\begin{bmatrix}7&8\end{bmatrix}\\
        3\cdot\begin{bmatrix}5&6\end{bmatrix}+4\cdot\begin{bmatrix}7&8\end{bmatrix}
    \end{bmatrix}=\begin{bmatrix}
        19&22\\
        43&50
    \end{bmatrix}\]
    \[\mat{B}\mat{A}=\begin{bmatrix}
        5\cdot\begin{bmatrix}1&2\end{bmatrix}+6\cdot\begin{bmatrix}3&4\end{bmatrix}\\
        7\cdot\begin{bmatrix}1&2\end{bmatrix}+8\cdot\begin{bmatrix}3&4\end{bmatrix}
    \end{bmatrix}=\begin{bmatrix}
        23&34\\
        31&46
    \end{bmatrix}\]
    \[\mat{A}\mat{B}-\mat{B}\mat{A}=\begin{bmatrix}
        -4&-12\\
        12&4
    \end{bmatrix}\]
\end{solution}
\begin{homework}[8]
    如果$n$级方阵$\mat{B}$满足$\mat{B}^3=\mbf{0}$,求$(\mat{I}-\mat{B})(\mat{I}+\mat{B}+\mat{B}^2)$.
\end{homework}
\begin{solution}
    \[(\mat{I}-\mat{B})(\mat{I}+\mat{B}+\mat{B}^2)=\mat{I}^2+\mat{I}\mat{B}+\mat{I}\mat{B}^2-\mat{B}\mat{I}-\mat{B}^2-\mat{B}^3=\mat{I}-\mat{B}^3=\mat{I}\]
\end{solution}
\begin{homework}[10]
    证明:如果$\mat{A}=\dfrac12(\mat{B}+\mat{I})$,那么$\mat{A}^2=\mat{A}$当且仅当$\mat{B}^2=\mat{I}$.
\end{homework}
\begin{proof}
    $\Rightarrow$:由$\mat{A}^2=\mat{A}$可知
    \[\mbf{0}=\mat{A}^2-\mat{A}=\dfrac14(\mat{B}+\mat{I})(\mat{B}-\mat{I})=\dfrac14(\mat{B}^2-\mat{I})\]
    于是$\mat{B}^2-\mat{I}=\mbf{0}$,即$\mat{B}^2=\mat{I}$.\\
    $\Leftarrow$:由$\mat{B}^2=\mat{I}$可知
    \[\mat{A}^2-\mat{A}=\dfrac14(\mat{B}^2+2\mat{B}+\mat{I})-\dfrac12(\mat{B}+\mat{I})=\dfrac14(\mat{B}^2-\mat{I})=\mbf0\]
    于是$\mat{A}^2=\mat{A}$.
\end{proof}
\section*{习题4.2}
\begin{homework}[2]
    证明:两个$n$级下三角矩阵的乘积仍是下三角矩阵,并且乘积矩阵的主对角元等于因子矩阵相应主对角元的乘积.
\end{homework}
\begin{proof}
    设$\mat{A}=(a_{ij})$, $\mat{B}=(b_{ij})$均为$n$级下三角矩阵,则有
    \[\begin{aligned}
        \mat{A}\mat{B}
        &=\left(\sum_{i=1}^{n}\sum_{j=1}^{i}a_{ij}\mat{E}_{ij}\right)\left(\sum_{k=1}^{n}\sum_{l=1}^{k}b_{kl}\mat{E}_{kl}\right)\\
        &=\sum_{i=1}^{n}\sum_{j=1}^{i}\sum_{k=1}^{n}\sum_{l=1}^{k}a_{ij}b_{kl}\mat{E}_{ij}\mat{E}_{kl}\\
        &=\sum_{i=1}^{n}\sum_{j=1}^{i}\sum_{l=1}^{j}a_{ij}b_{jl}\mat{E}_{il}
    \end{aligned}\]
    于是总有$l\leq j\leq i$,因此$\mat{A}\mat{B}$是下三角矩阵.特别地,当$i=j=l$时有
    \[(\mat{A}\mat{B})_{ii}=a_{ij}b_{jl}=a_{ii}b_{ii}\]
    于是主对角元等于因子矩阵对应主对角元的乘积.
\end{proof}
\begin{homework}[3]
    证明:与所有$n$级矩阵可交换的矩阵一定是$n$级数量矩阵.
\end{homework}
\begin{proof}
    设待求矩阵为$\mat{A}=(a_{ij})$.特别的,$\mat{A}$与基本矩阵$\mat{E}_{ij}$可交换,于是总有
    \[\begin{bmatrix}
        0&\cdots&0\\
        \vdots&\ddots&\vdots\\
        a_{j1}&\cdots&a_{jn}\\
        \vdots&\ddots&\vdots\\
        0&\cdots&0
    \end{bmatrix}=\begin{bmatrix}
        0&\cdots&a_{1i}&\cdots&0\\
        \vdots&\ddots&\vdots&\ddots&\vdots\\
        0&\cdots&a_{ni}&\cdots&0\\
    \end{bmatrix}\]
    于是总有$a_{j1}=\cdots=a_{j(j-1)}=a_{j(j+1)}=\cdots=a_{jn}=0$, $a_{1i}=\cdots=a_{(i-1)i}=a_{(i+1)i}=\cdots=a_{ni}=0$, $a_{jj}=a_{ii}$对所有$1\leq i,j\leq n$成立.\\
    满足上述要求的矩阵除主对角元外均为$0$,并且主对角元均相等.于是$\mat{A}$是数量矩阵,得证.
\end{proof}
\begin{homework}[4]
    证明:设$\mat{A}$是任一$s\times n$矩阵,则$\mat{A}\mat{A}^\t$和$\mat{A}^\t\mat{A}$均为对称矩阵.
\end{homework}
\begin{proof}
    设$\mat{A}=(a_{ij})$, $\mat{A}^\t=(a'_{ij})$,则$a'_{ij}=a_{ji}$.考虑$\mat{A}\mat{A}^\t=(b_{ij})$的元素:
    \[b_{ij}=\sum_{k=1}^{n}a_{ik}a'_{kj}=\sum_{k=1}^{n}a_{ik}a_{jk}=\sum_{k=1}^{n}a'_{jk}a_{ki}=b_{ji}\]
    于是$\mat{A}\mat{A}^\t$是对称矩阵.将上述结论应用于$\mat{A}^\t$可得$\mat{A}^\t\mat{A}$也是对称矩阵.于是命题得证.
\end{proof}
\begin{homework}[7]
    证明:设$\mat{A}$是任一$n$级方阵,则$\mat{A}+\mat{A}^\t$是对称矩阵,$\mat{A}-\mat{A}^\t$是斜对称矩阵.
\end{homework}
\begin{proof}
    记$\mat{A}=(a_{ij})$, $\mat{B}=\mat{A}+\mat{A}^\t=(b_{ij})$, $\mat{C}=\mat{A}-\mat{A}^\t=(c_{ij})$,则
    \[b_{ij}=a_{ij}+a_{ji}=a_{ji}+a_{ij}=b_{ji}\]
    \[c_{ij}=a_{ij}-a_{ji}=-(a_{ji}-a_{ij})=-c_{ji}\]
    于是$\mat{B}$是对称矩阵, $\mat{C}$是斜对称矩阵,得证.
\end{proof}
\section*{习题4.3}
\begin{homework}[3]
    证明:设$\mat{A}$是$n$级方阵,则$\det\mat{A}\mat{A}^\t=(\det\mat{A})^2$.
\end{homework}
\begin{proof}
    有
    \[\det\mat{A}\mat{A}^\t=\det\mat{A}\cdot\det\mat{A}^\t=\det\mat{A}\cdot\det\mat{A}=(\det\mat{A})^2\]
    得证.
\end{proof}
\begin{homework}[5]
    证明:如果$\mat{A}$是数域$\K$上的$n$级方阵,且满足
    \[\mat{A}\mat{A}^\t=\mat{I},\quad\det\mat{A}=-1,\]
    则$\det(\mat{I}+\mat{A})=0$.
\end{homework}
\begin{proof}
    有
    \[\det(\mat{I}+\mat{A})=\det(\mat{A}\mat{A}^\t+\mat{A})=\det\mat{A}\det(\mat{A}^\t+\mat{I})\]
    而$(\mat{A}^\t+\mat{I})^\t=\mat{A}+\mat{I}$,于是
    \[\det(\mat{I}+\mat{A})=-\det(\mat{I}+\mat{A})\]
    于是
    \[\det(\mat{I}+\mat{A})=0\]
\end{proof}
\begin{homework}[7]
    设$s_k=x_1^k+x_2^k+x_3^k(k=0,1,2,3,4)$,矩阵
    \[\mat{A}=\begin{bmatrix}
        s_0&s_1&s_2\\
        s_1&s_2&s_3\\
        s_2&s_3&s_4
    \end{bmatrix}\]
    证明:
    \[\det\mat{A}=\prod_{1\leq j<i\leq 3}(x_i-x_j)^2\]
\end{homework}
\begin{proof}
    注意到$\mat{A}$的$(i,j)$元为$s_{i+j-2}$.设
    \[\mat{B}=\begin{bmatrix}
        1&1&1\\
        x_1&x_2&x_3\\
        x_1^2&x_2^2&x_3^2\\
    \end{bmatrix},\quad\mat{C}=\begin{bmatrix}
        1&x_1&x_1^2\\
        1&x_2&x_2^2\\
        1&x_3&x_3^2\\
    \end{bmatrix}\]
    于是$b_{ik}=x_k^{i-1}$, $c_{kj}=x_{k}^{j-1}$.于是有
    \[a_{ij}=x_1^{i+j-2}+x_2^{i+j-2}+x_3^{i+j-2}=\sum_{k=1}^{3}b_{ik}c_{kj}\]
    于是$\mat{A}=\mat{B}\mat{C}$,因而
    \[\det\mat{A}=\det\mat{B}\det\mat{C}=\prod_{1\leq j<i\leq 3}(x_i-x_j)\prod_{1\leq j<i\leq 3}(x_i-x_j)=\prod_{1\leq j<i\leq 3}(x_i-x_j)^2\]
\end{proof}
\section*{习题4.4}
\begin{homework}[1]
    数量矩阵$k\mat{I}$何时可逆?何时不可逆?当$k\mat{I}$可逆时,求其逆矩阵.
\end{homework}
\begin{solution}
    有
    \[\det k\mat{I}=k\det\mat{I}=k\]
    于是当且仅当$k\neq0$时$k\mat{I}$可逆,否则不可逆.当$k\neq0$时有
    \[(k\mat{I})^{-1}=\dfrac1k\mat{I}\]
\end{solution}
\begin{homework}[3(2)]
    判断下列矩阵是否可逆,如果可逆则求出其逆矩阵:
    \[\begin{bmatrix}
        0&1\\1&0
    \end{bmatrix}\]
\end{homework}
\begin{solution}
    \[\det\begin{bmatrix}
        0&1\\1&0
    \end{bmatrix}=0-1=-1\neq0\]
    于是$\mat{A}$可逆,并且
    \[\mat{A}^{-1}=\dfrac{1}{\det\mat{A}}\mat{A}^\ast=-\begin{bmatrix}
        0&-1\\-1&0
    \end{bmatrix}=\begin{bmatrix}
        0&1\\1&0
    \end{bmatrix}\]
\end{solution}
\begin{homework}[4]
    证明:如果$\mat{A}$可逆,那么$\mat{A}^\ast$也可逆,并求$(\mat{A}^\ast)^{-1}$.
\end{homework}
\begin{proof}
    由于$\mat{A}$可逆,于是
    \[\mat{A}^{-1}=\dfrac{1}{\det\mat{A}}\mat{A}^\ast\]
    令$\mat{B}=\dfrac{1}{\det\mat{A}}\mat{A}$,则有
    \[\mat{B}\mat{A}^\ast=\dfrac{1}{\det\mat{A}}\mat{A}(\det\mat{A})\mat{A}^{-1}=\mat{A}\mat{A}^{-1}=\mat{I}\]
    同理可证$\mat{A}^\ast\mat{B}=\mat{I}$.于是$\mat{A}^\ast$可逆,其逆矩阵为$\dfrac{1}{\det\mat{A}}\mat{A}$.
\end{proof}
\begin{homework}[5]
    证明:如果$n$级方阵$\mat{A}$满足$\mat{A}^3=\mbf0$,那么$\mat{I}-\mat{A}$可逆,并求出其逆矩阵.
\end{homework}
\begin{proof}
    由$\mat{A}^3=\mbf0$可知
    \[\mat{I}^3-\mat{A}^3=\mat{I}\]
    于是
    \[(\mat{I}-\mat{A})(\mat{I}+\mat{A}+\mat{A}^2)=\mat{I}\]
    于是$\mat{I}-\mat{A}$可逆,且$(\mat{I}-\mat{A})^{-1}=\mat{I}+\mat{A}+\mat{A}^2$.
\end{proof}
\begin{homework}[7]
    证明:如果$n$级矩阵$\mat{A}$满足$2\mat{A}^4-5\mat{A}^2+4\mat{A}+2\mat{I}=\mbf0$,那么$\mat{A}$可逆,并求出$\mat{A}^{-1}$.
\end{homework}
\begin{proof}
    由题意可知
    \[-2\mat{I}=2\mat{A}^4-5\mat{A}^2+4\mat{A}=\mat{A}(2\mat{A}^3-5\mat{A}+4\mat{I})\]
    于是
    \[\mat{A}\left(-\mat{A}^3+\dfrac52\mat{A}-2\mat{I}\right)=\mat{I}\]
    于是$\mat{A}$可逆,并且$\mat{A}^{-1}=-\mat{A}^3+\dfrac52\mat{A}-2\mat{I}$.
\end{proof}
\begin{homework}[8]
    证明:可逆的对称(斜对称)矩阵的逆依然是对称(斜对称)矩阵.
\end{homework}
\begin{proof}
    设$\mat{A}$为对称矩阵,则$\mat{A}=\mat{A}^\t$.于是
    \[\mat{A}^{-1}=(\mat{A}^\t)^{-1}=(\mat{A}^{-1})^\t\]
    于是$\mat{A}^{-1}$也是对称矩阵.\\
    设$\mat{B}$为斜对称矩阵,则$\mat{B}=-\mat{B}^\t$.于是
    \[\mat{B}^{-1}=(-\mat{B}^\t)^{-1}=(-\mat{B}^{-1})^\t\]
    于是$\mat{B}^{-1}$也是斜对称矩阵.
\end{proof}
\begin{homework}[9(1)]
    求下列矩阵的逆:
    \[\begin{bmatrix}
        1&0&-1\\
        -2&1&3\\
        3&-1&2
    \end{bmatrix}\]
\end{homework}
\begin{solution}
    有
    \[\begin{bmatrix}
        1&0&-1&1&0&0\\
        -2&1&3&0&1&0\\
        3&-1&2&0&0&1
    \end{bmatrix}\longrightarrow\begin{bmatrix}
        1&0&-1&1&0&0\\
        0&1&1&2&1&0\\
        0&-1&5&-3&0&1
    \end{bmatrix}\longrightarrow\begin{bmatrix}
        1&0&-1&1&0&0\\
        0&1&1&2&1&0\\
        0&0&6&-1&1&1
    \end{bmatrix}\]
    \[\longrightarrow\begin{bmatrix}
        1&0&-1&1&0&0\\
        0&1&1&2&1&0\\
        0&0&1&-\frac16&\frac16&\frac16
    \end{bmatrix}\longrightarrow\begin{bmatrix}
        1&0&0&\frac{5}{6}&\frac16&\frac16\\
        0&1&0&\frac{13}{6}&\frac{5}{6}&-\frac16\\
        0&0&1&-\frac16&\frac16&\frac16
    \end{bmatrix}\]
    于是题设矩阵的逆矩阵为
    \[\begin{bmatrix}
        \frac{5}{6}&\frac16&\frac16\\
        \frac{13}{6}&\frac{5}{6}&-\frac16\\
        -\frac16&\frac16&\frac16
    \end{bmatrix}\]
\end{solution}
\begin{homework}[10(2)]
    解下列矩阵方程:
    \[\mat{X}\begin{bmatrix}
        3&-1&2\\
        1&0&-1\\
        -2&1&4
    \end{bmatrix}=\begin{bmatrix}
        3&0&-2\\
        -1&4&1
    \end{bmatrix}\]
\end{homework}
\begin{solution}
    首先有
    \[\begin{bmatrix}
        3&-1&2&1&0&0\\
        1&0&-1&0&1&0\\
        -2&1&4&0&0&1
    \end{bmatrix}\longrightarrow\begin{bmatrix}
        1&0&-1&0&1&0\\
        0&-1&5&1&-3&0\\
        0&1&2&0&2&1
    \end{bmatrix}\longrightarrow\begin{bmatrix}
        1&0&-1&0&1&0\\
        0&1&2&0&2&1\\
        0&0&7&1&-1&1
    \end{bmatrix}\]
    \[\longrightarrow\begin{bmatrix}
        1&0&-1&0&1&0\\
        0&1&2&0&2&1\\
        0&0&1&\frac17&-\frac17&\frac17
    \end{bmatrix}\longrightarrow\begin{bmatrix}
        1&0&0&\frac17&\frac67&\frac17\\
        0&1&0&-\frac27&\frac{16}{7}&\frac57\\
        0&0&1&\frac17&-\frac17&\frac17
    \end{bmatrix}\]
    于是
    \[\begin{bmatrix}
        3&-1&2\\
        1&0&-1\\
        -2&1&4
    \end{bmatrix}^{-1}=\dfrac17\begin{bmatrix}
        1&6&1\\
        -2&16&5\\
        1&-1&1
    \end{bmatrix}\]
    于是
    \[\mat{X}=\begin{bmatrix}
        3&0&-2\\
        -1&4&1
    \end{bmatrix}\begin{bmatrix}
        3&-1&2\\
        1&0&-1\\
        -2&1&4
    \end{bmatrix}^{-1}=\dfrac17\begin{bmatrix}
        3&0&-2\\
        -1&4&1
    \end{bmatrix}\begin{bmatrix}
        1&6&1\\
        -2&16&5\\
        1&-1&1
    \end{bmatrix}=\begin{bmatrix}
        \frac17&\frac{20}{7}&\frac17\\
        -\frac87&\frac{57}{7}&\frac{20}{7}
    \end{bmatrix}\]
\end{solution}
\end{document}