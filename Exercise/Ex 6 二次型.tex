\documentclass{ctexart}
\usepackage{Note}
\begin{document}
\section{二次型}
\begin{problem}
    求实二次型
    \[f(x_1,x_2,x_3)=x_1^2+x_2^2+tx_1x_2+2tx_1x_3+4x_2x_3\]
    的秩和正惯性指数.
\end{problem}
\begin{solution}
    这二次型的矩阵为
    \[\mat{A}=\begin{bmatrix}
        1&\frac{t}{2}&t\\
        \frac{t}{2}&1&2\\
        t&2&0
    \end{bmatrix}\]
    它的顺序主子式分别为
    \[\begin{vmatrix}
        1
    \end{vmatrix}=1,\quad\begin{vmatrix}
        1&\frac{t}{2}\\
        \frac{t}{2}&1
    \end{vmatrix}=1-\dfrac{t^2}{4},\quad\begin{vmatrix}
        1&\frac{t}{2}&t\\
        \frac{t}{2}&1&2\\
        t&2&0
    \end{vmatrix}=t^2-4\]
    如果$t^2=4$,则$\mat{A}$不满秩.此时$\mat{A}$的$(1,1)$元的余子式非零,因此$\rank\mat{A}=2$,正惯性指数为$1$.\\
    如果$t^2\neq4$,则矩阵满秩,只需判断$\mat{A}$的特征值的正负. $\mat{A}$的特征多项式为
    \[\lambda^3-2\lambda^2+\left(-\dfrac{5t^2}{4}-3\right)\lambda+(4-t^2)\]
    设它的三个实根为$\lambda_1,\lambda_2,\lambda_3$,则有
    \[\lambda_1+\lambda_2+\lambda_3=2,\quad\lambda_1\lambda_2+\lambda_1\lambda_3+\lambda_2\lambda_3=-\dfrac{5t^2}{4}-3,\quad\lambda_1\lambda_2\lambda_3=t^2-4\]
    前两个式子表明三个根中有正数也有负数.如果$t^2>4$,则正惯性指数为$1$;如果$t^2<4$,则正惯性指数为$2$.
\end{solution}
\begin{problem}
    判断当参数$a$满足什么条件时,实二次型
    \[f(x_1,x_2,x_3)=a(x_1^2+x_2^2+x_3^2)+2(x_1x_2+x_1x_3+x_2x_3)\]
    是正定的.
\end{problem}
\begin{solution}
    这二次型的矩阵为
    \[\mat{A}=\begin{bmatrix}
        a&1&1\\
        1&a&1\\
        1&1&a
    \end{bmatrix}\]
    其特征多项式为
    \[\begin{aligned}
        \det(\lambda\mat{I}-\mat{A})
        &=\begin{vmatrix}
            \lambda-a&-1&-1\\
            -1&\lambda-a&-1\\
            -1&-1&\lambda-a
        \end{vmatrix}\\
        &=(\lambda-a)(\lambda-a-1)(\lambda-a+1)+(-\lambda+a-1)-(\lambda-a+1)\\
        &=(\lambda-a+1)(\lambda^2-(2a+1)\lambda+a^2+a-2)\\
        &=(\lambda-a+1)^2(\lambda-a-2)
    \end{aligned}\]
    上述二次型正定当且仅当$\mat{A}$的所有特征值大于零,即
    \[a-1>0,\quad a+2>0\]
    从而$a>1$.
\end{solution}
\begin{problem}
    设实二次型
    \[f(x_1,x_2,x_3)=x_1^2+x_2^2+x_3^2+2ax_1x_2+2ax_2x_3+2ax_1x_3\]
    可以做非退化线性变换成为
    \[g(y_1,y_2,y_3)=y_1^2+y_2^2+4y_3^2+2y_1y_2\]
    求$a$的值和相应的非退化线性变换.
\end{problem}
\begin{solution}
    二次型$f,g$的矩阵分别为
    \[\mat{A}=\begin{bmatrix}
        1&a&a\\
        a&1&a\\
        a&a&1
    \end{bmatrix},\quad\mat{B}=\begin{bmatrix}
        1&1&0\\
        1&1&0\\
        0&0&4
    \end{bmatrix}\]
    
\end{solution}
\end{document}