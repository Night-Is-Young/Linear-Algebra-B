\documentclass{ctexart}
\usepackage{note}
\begin{document}\pagestyle{empty}
\begin{center}
    \tbf{\Large 北京大学数学科学学院2025-26学年第一学期线性代数B期中试题}
\end{center}
\begin{homework}[1(16')]
    已知线性方程组
    \[\left\{\begin{array}{l}
        ax_1+x_2+x_3=2\\
        x_1+bx_2+x_3=1\\
        x_1+2bx_2+x_3=1
    \end{array}\right.\]
    回答下列问题:
    \begin{enumerate}
        \item $a,b$为何值时,上述方程组\tbf{a.}有唯一解; \tbf{b.}有无穷多解; \tbf{c.}无解.
        \item 当方程组有无穷多解时,求方程组的导出组的一个基础解系.
    \end{enumerate}
\end{homework}
\begin{homework}[2(16')]
    判断下列向量组线性相关还是线性无关.如果线性相关,试找出其中一个向量使得它可以由其余向量线性表出,并给出一种表出方式.
    \[\bs\alpha_1=\begin{bmatrix}
        -2\\1\\0\\3
    \end{bmatrix},\quad\bs\alpha_2=\begin{bmatrix}
        1\\-3\\2\\4
    \end{bmatrix},\quad\bs\alpha_3=\begin{bmatrix}
        3\\0\\2\\-1
    \end{bmatrix},\quad\bs\alpha_4=\begin{bmatrix}
        2\\-2\\4\\6
    \end{bmatrix}\]
\end{homework}
\begin{homework}[3(10')]
    计算下面的$n$阶行列式$(n\geq2)$:
    \[\begin{vmatrix}
        1&2&3&\cdots&n-1&n\\
        1&-1&0&\cdots&0&0\\
        0&2&-2&\cdots&0&0\\
        \vdots&\vdots&\vdots&\ddots&\vdots&\vdots\\
        0&0&0&\cdots&n-1&1-n
    \end{vmatrix}\]
\end{homework}
\begin{homework}[4(10')]
    计算下面的$n$阶行列式$(n\geq2)$:
    \[\begin{vmatrix}
        1&\alpha&\alpha^2&\cdots&\alpha^{n-2}&\alpha^{n-1}\\
        \beta&1&\alpha&\cdots&\alpha^{n-3}&\alpha^{n-2}\\
        \beta^2&\beta&1&\cdots&\alpha^{n-4}&\alpha^{n-3}\\
        \vdots&\vdots&\vdots&\ddots&\vdots&\vdots\\
        \beta^{n-2}&\beta^{n-3}&\beta^{n-4}&\cdots&1&\alpha\\
        \beta^{n-1}&\beta^{n-2}&\beta^{n-3}&\cdots&\beta&1
    \end{vmatrix}\]
\end{homework}
\begin{homework}[5(13')]
    设
    \[\mat{A}=\begin{bmatrix}
        1&1&7&a\\
        2&2&1&b\\
        1&-1&5&c
    \end{bmatrix},\quad\mat{B}=\begin{bmatrix}
        1&1&5&9\\
        0&2&-1&1\\
        0&0&3&3
    \end{bmatrix}\]
    如果$\mat{A}$可经初等行变换化为$\mat{B}$,求$a,b,c$的值.
\end{homework}
\begin{homework}[6(12')]
    设
    \[\bs\alpha_1=\begin{bmatrix}
        1\\3\\2
    \end{bmatrix},\quad\bs\alpha_2=\begin{bmatrix}
        4\\5\\1
    \end{bmatrix},\quad\bs\alpha_3=\begin{bmatrix}
        2\\1\\6
    \end{bmatrix}\]
    矩阵$\mat{A}$的列向量组为$\bs\alpha_1,\bs\alpha_2,\bs\alpha_3$.设
    \[\bs\beta_1=\begin{bmatrix}
        \det\mat{A}\\0\\0
    \end{bmatrix},\quad\bs\beta_2=\begin{bmatrix}
        0\\\det\mat{A}\\0
    \end{bmatrix},\quad\bs\beta_3=\begin{bmatrix}
        0\\0\\\det\mat{A}
    \end{bmatrix}\]
    求下面三个方程组的解:
    \[x_1\bs\alpha_1+x_2\bs\alpha_2+x_3\bs\alpha_3=\bs\beta_i,\quad i=1,2,3.\]
\end{homework}
\begin{homework}[7(10')]
    证明:
    \[\rank\begin{bmatrix}
        \mat{A}&\mat{B}\\
        \mat{C}&\mat{D}
    \end{bmatrix}\leq\rank\mat{A}+\rank\mat{B}+\rank\mat{C}+\rank\mat{D}\]
\end{homework}
\begin{homework}[8(8')]
    设如下方程组
    \[\left\{\begin{array}{c}
        a_{11}x_1+a_{12}x_2+\cdots+a_{1r}x_r=a_{1(r+1)}x_{r+1}+\cdots+a_{1n}x_n\\
        a_{21}x_1+a_{22}x_2+\cdots+a_{2r}x_r=a_{2(r+1)}x_{r+1}+\cdots+a_{2n}x_n\\
        \vdots\\
        a_{r1}x_1+a_{r2}x_2+\cdots+a_{rr}x_r=a_{r(r+1)}x_{r+1}+\cdots+a_{rn}x_n
    \end{array}\right.\]
    有解,其中行列式
    \[\begin{vmatrix}
        a_{11}&\cdots&a_{1r}\\
        \vdots&\ddots&\vdots\\
        a_{r1}&\cdots&a_{rr}
    \end{vmatrix}\neq0\]
    设
    \[\bs\beta_i=\begin{bmatrix}
        l_{i1}&l_{i2}&\cdots&l_{ir}&l_{i(r+1)}&\cdots&l_{in}
    \end{bmatrix},\quad i=1,2,\cdots s\]
    为上述方程组的解.令
    \[\bs\beta_i'=\begin{bmatrix}
        l_{i(r+1)}&\cdots&l_{in}
    \end{bmatrix},\quad i=1,2,\cdots s\]
    证明:
    \[\rank\left\{\li{\bs\beta},s\right\}=\rank\left\{\bs\beta_1',\cdots,\bs\beta_s'\right\}\]
\end{homework}
\begin{homework}[9(5')]
    设矩阵
    \[\mat{A}=\begin{bmatrix}
        a_{11}&a_{12}&\cdots&a_{1n}\\
        a_{21}&a_{22}&\cdots&a_{2n}\\
        \vdots&\vdots&\ddots&\vdots\\
        a_{s1}&a_{s2}&\cdots&a_{sn}
    \end{bmatrix}\]
    其中$s\leq n$.定义
    \[\det\mat{A}=\sum_{j_1j_2\cdots j_s}(-1)^{\tau\left(j_1j_2\cdots j_s\right)}a_{1j_1}a_{2j_2}\cdots a_{sj_s}\]
    其中$j_1j_2\cdots j_s$取遍$1,2,\cdots,n$种任意$s$个不同数的排列.当$\det\mat{A}\neq0$时,问$\rank\mat{A}$是否为$s$.若是,请给出证明;若不是,请给出反例.
\end{homework}
\end{document}