\documentclass{ctexart}
\usepackage{Note}
\begin{document}
\section{$n$维向量空间}
\begin{problem}
    设
    \[\bs{\alpha}_1=\begin{pmatrix}
        1\\3\\-5\\-9
    \end{pmatrix},\quad\bs\alpha_2=\begin{pmatrix}
        2\\-1\\-3\\-4
    \end{pmatrix},\quad\bs\alpha_3=\begin{pmatrix}
        -3\\5\\1\\-1
    \end{pmatrix},\quad\bs\alpha_4=\begin{pmatrix}
        -4\\6\\2\\0
    \end{pmatrix},\quad\bs\beta=\begin{pmatrix}
        -5\\-1\\11\\17
    \end{pmatrix}\]
    \begin{enumerate}[label=\tbf{\arabic*}.,topsep=0pt,parsep=0pt,itemsep=0pt,partopsep=0pt]
        \item 求向量组$\bs\alpha_1,\bs\alpha_2,\bs\alpha_3,\bs\alpha_4$的一个极大线性无关组.
        \item 判断$\bs\beta$是否可以由向量组$\bs\alpha_1,\bs\alpha_2,\bs\alpha_3,\bs\alpha_4$线性表出.如果可以,给出所有的线性表出方式.
    \end{enumerate}
\end{problem}
\begin{solution}
    考虑以$\bs\alpha_1,\bs\alpha_2,\bs\alpha_3,\bs\alpha_4$为列向量的矩阵
    \[\mat{A}=\begin{bmatrix}
        1 & 2 & -3 & -4\\
        3 & -1 & 5 & 6\\
        -5 & -3 & 1 & 2\\
        -9 & -4 & -1 & 0
    \end{bmatrix}\]
    对$\mat{A}$做初等行变换可得
    \[\mat{A}\longrightarrow\begin{bmatrix}
        1&2&-3&-4\\
        0&-7&14&18\\
        0&7&-14&-18\\
        0&-7&14&18
    \end{bmatrix}\longrightarrow\begin{bmatrix}
        1&2&-3&-4\\
        0&-7&14&18\\
        0&0&0&0\\
        0&0&0&0
    \end{bmatrix}\]
    于是上述向量组的一个极大线性无关组为$\bs\alpha_1,\bs\alpha_2$.\\
    \indent 考虑方程
    \[x_1\bs\alpha_1+x_2\bs\alpha_2+x_3\bs\alpha_3+x_4\bs\alpha_4=\bs\beta\]
    对应线性方程组的增广矩阵
    \[\mat{B}=\begin{bmatrix}
        1 & 2 & -3 & -4&-5\\
        3 & -1 & 5 & 6&-1\\
        -5 & -3 & 1 & 2&11\\
        -9 & -4 & -1 & 0&17
    \end{bmatrix}\]
    对$\mat{B}$做初等行变换可得
    \[\mat{B}\longrightarrow\begin{bmatrix}
        1&2&-3&-4&-5\\
        0&-7&14&18&14\\
        0&7&-14&-18&-14\\
        0&-7&14&18&14
    \end{bmatrix}\longrightarrow\begin{bmatrix}
        1&2&-3&-4&-5\\
        0&-7&14&18&14\\
        0&0&0&0&0\\
        0&0&0&0&0
    \end{bmatrix}\]
    于是原方程组的解为
    \[\left\{\begin{array}{l}
        x_1=-1-x_3-\dfrac{8}{7}x_4\\
        x_2=-2+2x_3+\dfrac{18}{7}x_4\\
    \end{array}\right.\]
    因此$\bs\beta$可以被$\bs\alpha_1,\bs\alpha_2,\bs\alpha_3,\bs\alpha_4$线性表出,且所有的线性表出方式为
    \[\bs\beta=-\left(1+a+\dfrac{8}{7}b\right)\bs\alpha_1+\left(-2+2a+\dfrac{18}{7}b\right)\bs\alpha_2+a\bs\alpha_3+b\bs\alpha_4,\quad a,b\in\mathbb{K}\]
\end{solution}
\begin{problem}
    设
    \[\mat{A}=\begin{bmatrix}
        1&1&2&x\\3&2&1&y\\1&-1&0&z
    \end{bmatrix},\quad\mat{B}=\begin{bmatrix}
        3&1&5&14\\0&2&-1&11\\0&0&3&3
    \end{bmatrix}\]
    求$x,y,z$使得$\mat{A}$的行向量组与$\mat{B}$的行向量组等价.
\end{problem}
\begin{solution}
    对$\mat{A}$做初等行变换可得
    \[\mat{A}\longrightarrow\begin{bmatrix}
        1&1&2&x\\
        0&-1&-5&y-3x\\
        0&-2&-2&z-x
    \end{bmatrix}\longrightarrow\begin{bmatrix}
        1&1&2&x\\
        0&-1&-5&y-3x\\
        0&0&8&5x-2y+z
    \end{bmatrix}\]
\end{solution}
\begin{problem}
    求$\lambda$的值使矩阵
    \[\mat{A}=\begin{bmatrix}
        3&1&1&4\\
        \lambda&4&10&1\\
        1&7&17&3\\
        2&2&4&3
    \end{bmatrix}\]
    的秩最小.
\end{problem}
\begin{solution}
    对$\mat{A}$做初等行变换可得
    \[\begin{aligned}
        \mat{A}
        &\longrightarrow\begin{bmatrix}
            3&1&1&4\\
            0&12-\lambda&30-\lambda&3-4\lambda\\
            0&20&50&5\\
            0&4&10&1
        \end{bmatrix}\longrightarrow\begin{bmatrix}
            3&1&1&4\\
            0&4&10&1\\
            0&12-\lambda&30-\lambda&3-4\lambda\\
            0&0&0&0
        \end{bmatrix}\\
        &\longrightarrow\begin{bmatrix}
            3&1&1&4\\
            0&4&10&1\\
            0&\lambda&\lambda&4\lambda\\
            0&0&0&0
        \end{bmatrix}\longrightarrow\begin{bmatrix}
            3&1&1&4\\
            0&4&10&1\\
            0&0&-2\lambda&5\lambda\\
            0&0&0&0
        \end{bmatrix}
    \end{aligned}\]
    当$\lambda=0$时,矩阵的秩为$2$,否则矩阵的秩为$3$.于是$\lambda=0$.
\end{solution}
\begin{problem}
    设向量组$\bs\alpha_1,\bs\alpha_2,\bs\alpha_3,\bs\alpha_4$线性无关.令
    \[\bs\beta_1=\bs\alpha_1+2\bs\alpha_2+3\bs\alpha_3+4\bs\alpha_4,\quad\bs\beta_2=2\bs\alpha_1+3\bs\alpha_2+4\bs\alpha_3+\bs\alpha_4,\]
    \[\bs\beta_3=3\bs\alpha_1+4\bs\alpha_2+1\bs\alpha_3+2\bs\alpha_4,\quad\bs\beta_4=4\bs\alpha_1+\bs\alpha_2+2\bs\alpha_3+3\bs\alpha_4.\]
    判断向量组$\bs\beta_1,\bs\beta_2,\bs\beta_3,\bs\beta_4$是否线性无关.
\end{problem}
\begin{solution}
    假定向量组$\bs\beta_1,\bs\beta_2,\bs\beta_3,\bs\beta_4$线性相关,于是存在不全为$0$的$k_1,k_2,k_3,k_4$使得
    \[k_1\bs\beta_1+k_2\bs\beta_2+k_3\bs\beta_3+k_4\bs\beta_4=\mbf0\]
    即
    \[\begin{aligned}
        &\left(k_1+2k_2+3k_3+4k_4\right)\bs\alpha_1+\left(2k_1+3k_2+4k_3+k_4\right)\bs\alpha_2\\
        +&\left(3k_1+4k_2+k_3+2k_4\right)\bs\alpha_3+\left(4k_1+k_2+2k_3+3k_4\right)\bs\alpha_4=\mbf0
    \end{aligned}\]
    由于向量组$\bs\alpha_1,\bs\alpha_2,\bs\alpha_3,\bs\alpha_4$线性无关,于是
    \[\left\{\begin{array}{l}
        k_1+2k_2+3k_3+4k_4=0\\
        2k_1+3k_2+4k_3+k_4=0\\
        3k_1+4k_2+k_3+2k_4=0\\
        4k_1+k_2+2k_3+3k_4=0
    \end{array}\right.\]
    这线性方程组的系数行列式为
    \[\begin{vmatrix}
        1&2&3&4\\
        2&3&4&1\\
        3&4&1&2\\
        4&1&2&3
    \end{vmatrix}=10\begin{vmatrix}
        1&2&3&4\\
        1&3&4&1\\
        1&4&1&2\\
        1&1&2&3
    \end{vmatrix}=10\begin{vmatrix}
        1&2&3&4\\
        0&1&1&-3\\
        0&2&-2&-2\\
        0&-1&-1&-1
    \end{vmatrix}=10\begin{vmatrix}
        1&1&-3\\
        2&-2&-2\\
        -1&-1&-1
    \end{vmatrix}=10\begin{vmatrix}
        1&1&-3\\
        0&-4&4\\
        0&0&-4
    \end{vmatrix}=160\neq0\]
    于是原方程仅有零解,即$k_1=k_2=k_3=k_4=0$,这与$k_1,k_2,k_3,k_4$不全为$0$矛盾.于是向量组$\bs\beta_1,\bs\beta_2,\bs\beta_3,\bs\beta_4$线性无关
\end{solution}
\begin{problem}
    设向量组
    \[\bs\alpha_1=\begin{pmatrix}
        3\\-1\\2\\-5
    \end{pmatrix},\quad\bs\alpha_2=\begin{pmatrix}
        4\\3\\7\\2
    \end{pmatrix},\quad\bs\alpha_3=\begin{pmatrix}
        1\\4\\5\\-3
    \end{pmatrix},\quad\bs\alpha_4=\begin{pmatrix}
        7\\-1\\2\\0
    \end{pmatrix},\quad\bs\alpha_5=\begin{pmatrix}
        1\\2\\5\\6
    \end{pmatrix}\]
    \begin{enumerate}[label=\tbf{\arabic*}.,topsep=0pt,parsep=0pt,itemsep=0pt,partopsep=0pt]
        \item 证明: $\bs\alpha_1,\bs\alpha_2$线性无关.
        \item 将$\bs\alpha_1,\bs\alpha_2$扩充为$\bs\alpha_1,\cdots,\bs\alpha_5$的一个极大线性无关组.
    \end{enumerate}
\end{problem}
\begin{solution}
    考虑向量
    \[\bs\beta_1=\begin{pmatrix}
        3\\-1
    \end{pmatrix},\quad\bs\beta_2=\begin{pmatrix}
        4\\3
    \end{pmatrix}\]
    以$\bs\beta_1,\bs\beta_2$为列向量的矩阵的行列式
    \[\begin{vmatrix}
        3&4\\
        -1&3
    \end{vmatrix}=13\neq0\]
    于是$\bs\beta_1,\bs\beta_2$线性无关,因而它们的延伸组$\bs\alpha_1,\bs\alpha_2$线性无关.\\
    \indent 注意到$\bs\alpha_3=\bs\alpha_2-\bs\alpha_1$.将$\bs\alpha_4$加入$\bs\alpha_1,\bs\alpha_2$中,考虑前三个分量有
    \[\begin{vmatrix}
        3&4&7\\
        -1&3&-1\\
        2&7&2
    \end{vmatrix}=\begin{vmatrix}
        3&4&7\\
        -1&3&-1\\
        0&0&-4
    \end{vmatrix}=-4\begin{vmatrix}
        3&4\\-1&3
    \end{vmatrix}=-52\neq0\]
    于是向量组$\bs\alpha_1,\bs\alpha_2,\bs\alpha_4$线性无关.将$\bs\alpha_5$加入$\bs\alpha_1,\bs\alpha_2,\bs\alpha_4$中,有
    \[\begin{vmatrix}
        3&4&7&1\\
        -1&3&-1&2\\
        2&7&2&5\\
        -5&2&0&6
    \end{vmatrix}=\begin{vmatrix}
        13&4&7&-11\\
        \frac{13}{2}&3&-1&-7\\
        \frac{39}{2}&7&2&-16\\
        0&2&0&0
    \end{vmatrix}=2\cdot\dfrac{13}{2}\cdot(-1)^{2+4}\begin{vmatrix}
        2&7&-11\\
        1&-1&-7\\
        3&2&-16
    \end{vmatrix}=13\begin{vmatrix}
        2&7&-11\\
        1&-1&-7\\
        0&-4&2
    \end{vmatrix}=780\neq0\]
    于是$\bs\alpha_1,\bs\alpha_2,\bs\alpha_4,\bs\alpha_5$线性无关.根据前面的讨论,向量组$\bs\alpha_1,\bs\alpha_2,\bs\alpha_4,\bs\alpha_5$即$\bs\alpha_1,\cdots,\bs\alpha_5$的一个极大线性无关组.
\end{solution}
\begin{problem}
    
\end{problem}
\end{document}