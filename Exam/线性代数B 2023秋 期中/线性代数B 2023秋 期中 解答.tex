\documentclass{ctexart}
\usepackage{note}
\begin{document}\pagestyle{empty}
\begin{center}
    \tbf{\Large 北京大学数学科学学院2023-24学年第一学期线性代数B期中试题}
\end{center}
\begin{homework}[1(10')]
    判断下面的方程组是否有解.如果有解,给出方程组的全部解;如果无解,请给出理由.
    \[\left\{\begin{array}{l}
        x_1+x_3=1\\
        3x_1+x_2+2x_3+x_4=2\\
        x_1+2x_2-x_3+2x_4=2\\
        -x_2+x_3-x_4=0
    \end{array}\right.\]
\end{homework}
\begin{solution}
    对该方程的增广矩阵做初等行变换可得
    \[\begin{bmatrix}
        1&0&1&0&1\\
        3&1&2&1&2\\
        1&2&-1&2&2\\
        0&-1&1&-1&0
    \end{bmatrix}\longrightarrow\begin{bmatrix}
        1&0&1&0&1\\
        0&1&-1&1&-1\\
        0&2&-2&2&1\\
        0&-1&1&-1&0
    \end{bmatrix}\longrightarrow\begin{bmatrix}
        1&0&1&0&1\\
        0&-1&1&-1&0\\
        0&0&0&0&-1\\
        0&0&0&0&0
    \end{bmatrix}\]
    增广矩阵中存在形如$0=d$的行,于是原方程组无解.
\end{solution}
\begin{homework}[2(10')]
    已知线性方程组
    \[\left\{\begin{array}{c}
        \left(a_1+b\right)x_1+a_2x_2+\cdots+a_nx_n=0\\
        a_1x_1+\left(a_2+b\right)x_2+\cdots+a_nx_n=0\\
        \cdots\\
        a_1x_1+a_2x_2+\cdots+\left(a_n+b\right)x_n=0
    \end{array}\right.\]
    其中$\displaystyle\sum_{i=1}^{n}a_i\neq0$.
    \begin{enumerate}[label=\tbf{(\arabic*)},topsep=0pt,parsep=0pt,itemsep=0pt,partopsep=0pt]
        \item 计算上述方程组的系数矩阵的行列式.
        \item 讨论$\li a,n,b$满足何种条件时\tbf{a.}方程仅有零解; \tbf{b.}方程组有非零解.
        \item 当方程组有非零解时,求出方程的一个基础解系.
    \end{enumerate}
\end{homework}
\begin{solution}
    \begin{enumerate}[label=\tbf{(\arabic*)},topsep=0pt,parsep=0pt,itemsep=0pt,partopsep=0pt]
        \item 设系数矩阵为$\mat{A}$.注意到$\mat{A}$的每一行的和均为$\displaystyle\sum_{i=1}^{n}a_i+b$,于是将第二列及以后的列加到第一列上即可得
        \[\det\mat{A}=\begin{vmatrix}
            \displaystyle\sum_{i=1}^{n}a_i+b&a_2&\cdots&a_n\\
            \displaystyle\sum_{i=1}^{n}a_i+b&a_2+b&\cdots&a_n\\
            \vdots&\vdots&\ddots&\vdots\\
            \displaystyle\sum_{i=1}^{n}a_i+b&a_2&\cdots&a_n+b\\
        \end{vmatrix}=\left(\sum_{i=1}^{n}a_i+b\right)\begin{vmatrix}
            1&a_2&\cdots&a_n\\
            1&a_2+b&\cdots&a_n\\
            \vdots&\vdots&\ddots&\vdots\\
            1&a_2&\cdots&a_n+b\\
        \end{vmatrix}\]
        注意到第一列均为$1$,于是将第二行及以后的行减去第一行可得
        \[\det\mat{A}=\left(\sum_{i=1}^{n}a_i+b\right)\begin{vmatrix}
            1&a_2&\cdots&a_n\\
            0&b&\cdots&0\\
            \vdots&\vdots&\ddots&\vdots\\
            0&0&\cdots&b\\
        \end{vmatrix}=\left(\sum_{i=1}^{n}a_i+b\right)\begin{vmatrix}
            b&\cdots&0\\
            \vdots&\ddots&\vdots\\
            0&\cdots&b\\
        \end{vmatrix}=\left(\displaystyle\sum_{i=1}^{n}a_i+b\right)b^{n-1}\]
        \item 齐次方程仅有零解要求$\det\mat{A}\neq0$,于是$\displaystyle\sum_{i=1}^{n}a_i+b\neq0$且$b\neq0$.于是$a_1+\cdots+a_n+b\neq0$且$b\neq0$时原方程仅有零解; $a_1+\cdots+a_n+b=0$或$b=0$时原方程有非零解.
        \item 当$b=0$时,对系数矩阵做初等行变换可得
        \[\mat{A}\longrightarrow\begin{bmatrix}
            a_1&a_2&\cdots&a_n\\
            0&0&\cdots&0\\
            \vdots&\vdots&\ddots&\vdots\\
            0&0&\cdots&0
        \end{bmatrix}\]
        设$a_i$是$\li a,n$中第一个不为零的元素,则方程组的解为
        \[x_i=-\dfrac{a_1}{a_i}x_1-\cdots-\dfrac{a_{i-1}}{a_i}x_{i-1}-\dfrac{a_{i+1}}{a_i}x_{i+1}-\cdots-\dfrac{a_n}{a_i}x_n\]
        其中除$x_i$外均为自由变量.于是方程组的一个基础解系为$\bs\eta_1,\cdots,\bs\eta_{i-1},\bs\eta_{i+1},\cdots,\bs\eta_n$,其中$\bs\eta_{k}$的第$k$个分量为$1$,第$i$个分量为$-\dfrac{a_1}{a_i}$.
        当$b\neq0$且$\displaystyle\sum_{i=1}^{n}a_i+b=0$时,对系数矩阵做初等行变换可得
        \[\mat{A}\longrightarrow\begin{bmatrix}
            b&0&0&\cdots&-b\\
            0&b&0&\cdots&-b\\
            0&0&b&\cdots&-b\\
            \vdots&\vdots&\vdots&\ddots&\vdots\\
            a_1&a_2&a_3&\cdots&a_n+b
        \end{bmatrix}\longrightarrow\begin{bmatrix}
            1&0&0&\cdots&-1\\
            0&1&0&\cdots&-1\\
            0&0&1&\cdots&-1\\
            \vdots&\vdots&\vdots&\ddots&\vdots\\
            0&0&0&\cdots&0
        \end{bmatrix}\]
        于是方程的解为$x_i=-x_n(i=1,\cdots,n-1)$,方程组的基础解系为$\bs\eta=\begin{bmatrix}
            1&1&\cdots&1
        \end{bmatrix}^{\text{t}}$.
    \end{enumerate}
\end{solution}
\begin{homework}[3(15')]
    计算下面的行列式:
    \[A_n=\begin{vmatrix}
        0&1&2&\cdots&n-2&n-1\\
        1&0&1&\cdots&n-3&n-2\\
        \vdots&\vdots&\vdots&\ddots&\vdots&\vdots\\
        n-2&n-3&n-4&\cdots&0&1\\
        n-1&n-2&n-3&\cdots&1&0
    \end{vmatrix}\]
\end{homework}
\begin{solution}
    将第二行及以后的行减去第一行可得
    \[A_n=\begin{vmatrix}
        0&1&2&\cdots&n-2&n-1\\
        1&-1&-1&\cdots&-1&-1\\
        \vdots&\vdots&\vdots&\ddots&\vdots&\vdots\\
        1&1&1&\cdots&-1&-1\\
        1&1&1&\cdots&1&-1
    \end{vmatrix}\]
    将第二列及以后的列减去第一列可得
    \[A_n=\begin{vmatrix}
        0&1&2&\cdots&n-2&n-1\\
        1&-2&-2&\cdots&-2&-2\\
        \vdots&\vdots&\vdots&\ddots&\vdots&\vdots\\
        1&0&0&\cdots&-2&-2\\
        1&0&0&\cdots&0&-2
    \end{vmatrix}=\begin{vmatrix}
        \dfrac{n-1}{2}&1&2&\cdots&n-2&n-1\\
        0&-2&-2&\cdots&-2&-2\\
        \vdots&\vdots&\vdots&\ddots&\vdots&\vdots\\
        0&0&0&\cdots&-2&-2\\
        0&0&0&\cdots&0&-2
    \end{vmatrix}=\dfrac{n-1}{2}(-2)^{n-1}\]
    于是
    \[A_n=(-1)^{n-1}(n-1)2^{n-2}\]
\end{solution}
\begin{homework}[4(13')]
    求下面行列式中所有元素的代数余子式之和:
    \[\begin{vmatrix}
        1&0&0&0&0\\
        1&1&0&0&0\\
        1&1&1&0&0\\
        1&1&1&1&0\\
        1&1&1&1&1
    \end{vmatrix}\]
\end{homework}
\begin{solution}
    记行列式对应的矩阵为$\mat{A}$,由于$\mat{A}$是下三角矩阵,于是$\det\mat{A}=1$.\\
    于是有
    \[\sum_{1\leqslant i,j\leqslant 5}A_{ij}=\sum_{j=1}^{5}a_{5j}A_{5j}+\sum_{i=1}^{4}\sum_{j=1}^{5}a_{5j}A_{ij}=\det\mat{A}+0=1\]
\end{solution}
\begin{homework}[5(10')]
    已知向量组
    \[\bs\alpha_1=\begin{bmatrix}
        3\\4\\-2
    \end{bmatrix},\quad\bs\alpha_2=\begin{bmatrix}
        2\\-5\\0
    \end{bmatrix},\quad\bs\alpha_3=\begin{bmatrix}
        5\\0\\-1
    \end{bmatrix},\quad\bs\alpha_2=\begin{bmatrix}
        3\\3\\-3
    \end{bmatrix}\]
    求该向量组的一个极大线性无关组.
\end{homework}
\begin{solution}
    首先有
    \[\begin{vmatrix}
        3&2\\4&-5
    \end{vmatrix}=-23\neq0\]
    于是$\bs\alpha_1,\bs\alpha_2$线性无关.然后有
    \[\begin{vmatrix}
        3&2&5\\4&-5&0\\-2&0&-1
    \end{vmatrix}=\begin{vmatrix}
        -7&2&5\\4&-5&0\\0&0&-1
    \end{vmatrix}=(-1)\cdot\begin{vmatrix}
        -7&2\\4&-5
    \end{vmatrix}=27\neq0\]
    于是$\bs\alpha_1,\bs\alpha_2,\bs\alpha_3$线性无关.考虑由$\bs\alpha_1,\bs\alpha_2,\bs\alpha_3,\bs\alpha_4$构成的矩阵,其行秩与列秩最大为$3$,于是原向量组的一个极大线性无关组为$\bs\alpha_1,\bs\alpha_2,\bs\alpha_3$.
\end{solution}
\begin{homework}[6(12')]
    求下列$n$级方阵的秩:
    \[\begin{bmatrix}
        x&a&\cdots&a\\
        a&x&\cdots&a\\
        \vdots&\vdots&\ddots&\vdots\\
        a&a&\cdots&x
    \end{bmatrix}\]
\end{homework}
\begin{solution}
    记上述矩阵为$\mat{A}_n(x)$,注意到$\mat{A}_n$的每一行元素之和均为$x+(n-1)a$,于是有
    \[\begin{aligned}
        \det\mat{A}(x)
    &=\begin{vmatrix}
        x+(n-1)a&a&\cdots&a\\
        x+(n-1)a&x&\cdots&a\\
        \vdots&\vdots&\ddots&\vdots\\
        x+(n-1)a&a&\cdots&x
    \end{vmatrix}
    =\left[x+(n-1)a\right]\begin{vmatrix}
        1&a&\cdots&a\\
        1&x&\cdots&a\\
        \vdots&\vdots&\ddots&\vdots\\
        1&a&\cdots&x
    \end{vmatrix}\\
    &=\left[x+(n-1)a\right]\begin{vmatrix}
        1&a&\cdots&a\\
        0&x-a&\cdots&0\\
        \vdots&\vdots&\ddots&\vdots\\
        0&0&\cdots&x-a
    \end{vmatrix}=\left[x+(n-1)a\right](x-a)^{n-1}
    \end{aligned}\]
    当$x\neq a$且$x+(n-1)a\neq0$时$\det\mat{A}_n(x)\neq0$,即$\rank\ \mat{A}_n(x)=n$.\\
    当$x=a$时注意到每一列均相同.此时如果$a=0$则$\rank\ \mat{A}_n(x)=0$,否则$\rank\ \mat{A}_n(x)=1$.\\
    当$x=(1-n)a$时(这里$x=a=0$已经讨论过)考虑$\mat{A}_n(x)$的$(1,1)$元的余子式,恰为$\mat{A}_{n-1}(x)$.于是
    \[A_{11}=\mat{A}_{n-1}(x)=\left[x-(n-2)a\right](x-a)^{n-2}=a(-na)^{n-2}\neq0\]
    于是$\rank\mat{A}_n(x)=n-1$.于是总结如下:
    \[\rank\ \mat{A}_n(x)=\left\{\begin{array}{l}
        n,\quad x\neq a\text{且}x\neq(1-n)a\\
        n-1, \quad x=(1-n)a\text{且}a\neq0\\
        1,\quad x=a\text{且}a\neq0\\
         0,\quad x=a=0 
    \end{array}\right.\]
\end{solution}
\begin{homework}[7(10')]
    给定$n$个两两不同的数$\li a,n$.设$\li b,n$为任意的$n$个数,证明:存在唯一的次数不超过$n-1$的多项式$f(x)$使得$f\left(a_i\right)=b_i$成立.
\end{homework}
\begin{proof}
    考虑线性方程组
    \[\left\{\begin{array}{c}
        c_0+c_1a_1+c_2a_1^2+\cdots+c_{n-1}a_1^{n-1}=b_1\\
        c_0+c_1a_2+c_2a_2^2+\cdots+c_{n-1} a_2^{n-1}=b_2\\
        \cdots\\
        c_0+c_1a_n+c_2a_n^2+\cdots+c_{n-1} a_n^{n-1}=b_n
    \end{array}\right.\]
    这一方程组的系数矩阵的行列式为
    \[\begin{vmatrix}
        1&a_1&a_1^2&\cdots&a_1^{n-1}\\
        1&a_2&a_2^2&\cdots&a_2^{n-1}\\
        \vdots&\vdots&\vdots&\ddots&\vdots\\
        1&a_n&a_n^2&\cdots&a_n^{n-1}
    \end{vmatrix}=\prod_{1\leqslant i<j\leqslant n}\left(a_j-a_i\right)\neq0\]
    于是原方程组有唯一解,因此存在唯一的多项式
    \[f(x)=c_0+c_1x+\cdots+c_nx^{n-1}\]
    使其满足题意.
\end{proof}
\begin{homework}[8(10')]
    设$5$级方阵$\mat{M}$满足$\det\mat{M}\neq0$.试证明:存在一个$5$级上三角矩阵$\mat{B}$满足$\det\mat{B}\neq0$使得$\mat{B}\mat{M}$有如下性质:对于任一$1\leqslant i\leqslant 5$,都有且仅有$\mat{B}\mat{M}$的一行满足该行的前$i-1$个位置为$0$,第$i$个位置不为$0$.
\end{homework}
\end{document}