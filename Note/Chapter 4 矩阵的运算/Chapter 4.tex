\documentclass{ctexart}
\usepackage{Note}
\begin{document}
\section{矩阵的运算}
\subsection{矩阵的加法,数乘和乘法}
\begin{definition}[矩阵的加法]
    设$\mat{A}=(a_{ij})_{m\times n}$, $\mat{B}=(b_{ij})_{m\times n}$,令
    \[\mat{C}=(a_{ij}+b_{ij})_{m\times n}\]
    则称$\mat{C}$是$\mat{A}$与$\mat{B}$的和,记作$\mat{C}=\mat{A}+\mat{B}$.
\end{definition}
\begin{definition}[矩阵的数乘]
    设$\mat{A}=(a_{ij})_{m\times n}$,对于$k\in K$,令
    \[\mat{A}=(k_ij)_{m\times n}\]
    则称$\mat{M}$是$k$与矩阵$\mat{A}$的数量积,记作$\mat{M}=k\mat{A}$.
\end{definition}
\begin{definition}[矩阵的乘法]
    设$\mat{A}=(a_{ij})_{s\times n}$, $\mat{B}=(b_{ij})_{n\times m}$,令
    \[\mat{C}=(c_{ij})_{m\times n}\]
    其中
    \[c_{ij}=\sum_{k=1}^{n}a_{ik}b_{kj}\]
    则称$\mat{C}$是$\mat{A}$与$\mat{B}$的积,记作$\mat{C}=\mat{A}\mat{B}$.
\end{definition}
矩阵乘法满足结合律,但不满足交换律.
\begin{definition}[恒等矩阵]
    对角线元素为$1$,其余元素均为$0$的$n\times n$级矩阵称作$n$阶恒等矩阵,记作$\mat{I}_n$.
\end{definition}
\begin{definition}[可交换矩阵]
    如果$n$级方阵$\mat{A}$和$\mat{B}$满足
    \[\mat{A}\mat{B}=\mat{B}\mat{A}\]
    则称$\mat{A}$和$\mat{B}$是可交换的.
\end{definition}
\subsection{特殊矩阵}
\subsubsection{对角矩阵}
\begin{definition}[对角矩阵]
    除主对角线上的元素以外,其它元素均为$0$的矩阵称作\tbf{对角矩阵},记作$\diag\{\li a,n\}$,其中$\li a,n$为对角线上的元素.
\end{definition}
\begin{theorem}[对角矩阵的乘法]
    用对角矩阵$\mat{D}$左乘矩阵$\mat{A}$,相当于用$\mat{D}$的主对角元乘$\mat{A}$的各行,即
    \[\begin{bmatrix}
        d_1&\cdots&0\\
        \vdots&\ddots&\vdots\\
        0&\cdots&d_n
    \end{bmatrix}\begin{bmatrix}
        \bs\gamma_1\\
        \vdots\\
        \bs\gamma_n
    \end{bmatrix}=\begin{bmatrix}
        d_1\bs\gamma_1\\
        \vdots\\
        d_n\bs\gamma_n
    \end{bmatrix}\]
    用对角矩阵$\mat{D}$右乘矩阵$\mat{A}$,相当于用$\mat{D}$的主对角元乘$\mat{A}$的各列,即
    \[\begin{bmatrix}
        \bs\alpha_1&\cdots&\bs\alpha_n
    \end{bmatrix}\begin{bmatrix}
        d_1&\cdots&0\\
        \vdots&\ddots&\vdots\\
        0&\cdots&d_n
    \end{bmatrix}=\begin{bmatrix}
        d_1\bs\alpha_1&\cdots&d_n\bs\alpha_n
    \end{bmatrix}\]
\end{theorem}
\subsubsection{基本矩阵}
\begin{definition}[基本矩阵]
    只有一个元素是$1$,其它元素均为$0$的矩阵称作基本矩阵. $(i,j)$元为$1$的基本矩阵记作$\mat{E}_{ij}$.
\end{definition}
\begin{theorem}[基本矩阵的乘法]
    用$\mat{E}_{ij}$左乘矩阵$\mat{A}$就相当于把$\mat{A}$的第$j$行移到第$i$行,其余行均为$\mbf{0}$;用$\mat{E}_{ij}$右乘矩阵$\mat{A}$就相当于把$\mat{A}$的第$i$列移到第$j$列,其余列均为$\mbf{0}$.
\end{theorem}
\subsubsection{上/下三角矩阵}
\begin{definition}[上/下三角矩阵]
    主对角线下/上方元素均为$0$的方阵称作\tbf{上/下三角矩阵}.
\end{definition}
\begin{theorem}[上/下三角矩阵的乘法]
    上/下三角矩阵的乘积仍为上/下三角矩阵,并且主对角线上的元素等于各矩阵主对角线上对应元素的乘积.
\end{theorem}
\subsubsection{初等矩阵}
\begin{definition}[初等矩阵]
    由单位矩阵经过一次初等行(列)变换所得到的矩阵称作\tbf{初等矩阵}.
\end{definition}
根据单位矩阵和基本矩阵的乘法即可推导出有关初等矩阵的乘法.事实上,初等矩阵与其对应的初等行(列)变换是等价的.
\subsubsection{对称矩阵与斜对称矩阵}
\begin{definition}[对称矩阵]
    如果矩阵$\mat{A}$满足$\mat{A}=\mat{A}^{\text{t}}$,则称$\mat{A}$为\tbf{对称矩阵}.
\end{definition}
\begin{theorem}[对称矩阵的乘法]
    设$\mat{A}$和$\mat{B}$都是数域$\K$上的$n$级对称矩阵,则$\mat{A}\mat{B}$为对称矩阵当且仅当$\mat{A}$与$\mat{B}$可交换.
\end{theorem}
\begin{definition}[斜对称矩阵]
    如果矩阵$\mat{A}$满足$\mat{A}=-\mat{A}^{\text{t}}$,则称$\mat{A}$为\tbf{斜对称矩阵}.
\end{definition}
\begin{theorem}[斜对称矩阵的行列式]
    设$\mat{A}$为数域$\K$上的$n$级斜对称矩阵,则当$n$为奇数时,$\det(\mat{A})=0$.
\end{theorem}
\begin{proof}
    因为$\mat{A}^{\t}=-\mat{A}$,于是
    \[\det\mat{A}=\det\mat{A}^\t=\det(-\mat{A})=(-1)^n\det\mat{A}=-\det\mat{A}\]
    于是
    \[\det\mat{A}=0\]
\end{proof}
\subsection{矩阵乘积的秩与行列式}
\begin{theorem}
    设$\mat{A}$和$\mat{B}$分别为数域$\K$上的$s\times n$矩阵和$n\times m$矩阵,则
    \[\rank\mat{A}\mat{B}\leq\min\{\rank\mat{A},\rank\mat{B}\}\]
\end{theorem}
\begin{proof}
    设$\mat{A}$的列向量组为$\li{\bs\alpha},n$,则有
    \[\begin{aligned}
        \mat{A}\mat{B}
        &=\begin{bmatrix}
            \bs\alpha_1&\cdots\bs\alpha_n
        \end{bmatrix}\begin{bmatrix}
            b_{11}&\cdots&b_{1m}\\
            \vdots&\ddots&\vdots\\
            b_{n1}&\cdots&b_{nm}
        \end{bmatrix}\\
        &=\begin{bmatrix}
            b_{11}\bs\alpha_1+\cdots+b_{n1}\bs\alpha_n&\cdots&b_{1m}\bs\alpha_1+\cdots+b_{nm}\bs\alpha_n
        \end{bmatrix}
    \end{aligned}\]
    于是$\mat{AB}$的列向量组能被$\mat{A}$的列向量组线性表出,从而
    \[\rank\mat{A}\mat{B}\leq\rank\mat{A}\]
    利用这一结论有
    \[\rank\mat{A}\mat{B}=\rank(\mat{A}\mat{B})^\t=\rank(\mat{B}^\t\mat{A}^\t)\leq\rank\mat{B}^\t=\rank\mat{B}\]
    因此
    \[\rank\mat{A}\mat{B}\leq\min\{\rank\mat{A},\rank\mat{B}\}\]
    命题得证.
\end{proof}
\begin{theorem}[方阵乘积的行列式]
    设$\mat{A}$, $\mat{B}$均为数域$\K$上的$n$级方阵,则
    \[\det\mat{A}\mat{B}=\det\mat{A}\det\mat{B}\]
\end{theorem}
\begin{proof}
    考虑矩阵
    \[\begin{bmatrix}
        \mat{A}&\mbf0\\
        -\mat{I}&\mat{B}
    \end{bmatrix}=\begin{bmatrix}
        a_{11}&\cdots&a_{1n}&0&\cdots&0\\
        \vdots&\ddots&\vdots&\vdots&\ddots&\vdots\\
        a_{n1}&\cdots&a_{nn}&0&\cdots&0\\
        -1&\cdots&0&b_{11}&\cdots&b_{1n}\\
        \vdots&\ddots&\vdots&\vdots&\ddots&\vdots\\
        0&\cdots&-1&b_{n1}&\cdots&b_{nn}
    \end{bmatrix}\]
    将第$i$行$(1\leq i\leq n)$加上第$n+j$行$(1\leq j\leq n)$的$a_{ij}$倍可得
    \[\begin{vmatrix}
        \mat{A}&\mbf0\\
        -\mat{I}&\mat{B}
    \end{vmatrix}=\begin{vmatrix}
        0&\cdots&0&\displaystyle\sum_{k=1}^{n}a_{1k}b_{k1}&\cdots&\displaystyle\sum_{k=1}^{n}a_{1k}b_{kn}\\
        \vdots&\ddots&\vdots&\vdots&\ddots&\vdots\\
        0&\cdots&0&\displaystyle\sum_{k=1}^{n}a_{nk}b_{k1}&\cdots&\displaystyle\sum_{k=1}^{n}a_{nk}b_{kn}\\
        -1&\cdots&0&b_{11}&\cdots&b_{1n}\\
        \vdots&\ddots&\vdots&\vdots&\ddots&\vdots\\
        0&\cdots&-1&b_{n1}&\cdots&b_{nn}
    \end{vmatrix}=\begin{vmatrix}
        \mat{0}&\mat{A}\mat{B}\\
        -\mat{I}&\mat{B}
    \end{vmatrix}\]
    而
    \[\begin{vmatrix}
        \mat{0}&\mat{A}\mat{B}\\
        -\mat{I}&\mat{B}
    \end{vmatrix}=\det\mat{A}\mat{B}\det\mat{I}(-1)^{1+\cdots+(2n)+n}=(-1)^{2n^2+2n}\det\mat{A}\mat{B}=\det\mat{A}\mat{B}\]
    于是
    \[\det\mat{A}\mat{B}=\det\mat{A}\det\mat{B}\]
\end{proof}
现在我们把上述结论推广到非方阵的情形.
\begin{theorem}[Binet-Cauchy公式]\\
    设$\mat{A}=(a_{ij})_{s\times n}$, $\mat{B}=(b_{ij})_{n\times s}$.\\
    如果$s>n$,那么$\det\mat{A}\mat{B}=0$;
    如果$s\leq n$,那么
    \[\det\mat{A}\mat{B}=\sum_{1\leq v_1<\cdots<v_s\leq n}\mat{A}\left(\begin{array}{c}
        1,2,\cdots,s\\
        v_1,v_2,\cdots,v_s
    \end{array}\right)\mat{B}\left(\begin{array}{c}
        v_1,v_2,\cdots,v_s\\
        1,2,\cdots,s
    \end{array}\right)\]
    即$\det\mat{A}\mat{B}$为$\mat{A}$的所有$s$阶子式与$\mat{B}$的相应的$s$阶子式的乘积.
\end{theorem}
\subsection{可逆矩阵}
\begin{definition}[可逆矩阵]
    对于$\K$上的$n$级方阵$\mat{A}$,如果存在$n$级方阵$\mat{B}$使得
    \[\mat{A}\mat{B}=\mat{B}\mat{A}=\mat{I}_n\]
    则称$\mat{A}$为\tbf{可逆矩阵},并称$\mat{B}$为$\mat{A}$的\tbf{逆矩阵},记作$\mat{B}=\mat{A}^{-1}$.
\end{definition}
\begin{definition}[伴随矩阵]
    对于$\K$上的$n$级方阵$\mat{A}$,记
    \[\mat{A}^\ast=\begin{bmatrix}
        A_{11}&\cdots&A_{1n}\\
        \vdots&\ddots&\vdots\\
        A_{n1}&\cdots&A_{nn}
    \end{bmatrix}\]
    为$\mat{A}$的\tbf{伴随矩阵},其中$A_{ij}$为$\mat{A}$的$(i,j)$元的代数余子式.
\end{definition}
\begin{theorem}[可逆的条件]
    $\K$上的$n$级方阵$\mat{A}$可逆当且仅当$\det\mat{A}\neq0$,并且此时
    \[\mat{A}^{-1}=\dfrac{\mat{A}^\ast}{\det\mat{A}}\]
\end{theorem}
\subsection{正交矩阵与欧几里得空间}
\begin{definition}
    设$\mat{A}$为$\R$上的$n$级方阵,如果$\mat{A}$满足
    \[\mat{A}^\t\mat{A}=\mat{A}\mat{A}^\t=\mat{I}_n\]
    则称$\mat{A}$为\tbf{正交矩阵}.
\end{definition}
\begin{theorem}[正交矩阵的性质]
    设$\mat{A}$为$\R$上的$n$级正交矩阵,则有以下性质:
    \begin{enumerate}[label=\tbf{\arabic*.}]
        \item $\det\mat{A}=\pm1$.
        \item $\mat{A}$可逆,并且$\mat{A}^{-1}=\mat{A}^\t$.
    \end{enumerate}
\end{theorem}
\begin{definition}[内积]
    
\end{definition}
\end{document}