\documentclass{ctexart}
\usepackage{note}
\title{线性代数B  第二次作业 订正}
\author{蒋锦豪 2400011785}
\date{}
\begin{document}
\maketitle
\section*{习题2.5}
\begin{homework}[3]
    求$\lambda$使得下面的齐次线性方程组有非零解:
    \[\left\{\begin{array}{l}
        (\lambda-2)x_1-3x_2-2x_3=0\\
        -x_1+(\lambda-8)x_2-2x_3=0\\
        2x_1+14x_2+(\lambda+3)x_3=0
    \end{array}\right.\]
\end{homework}
\begin{solution}
    该齐次方程组有非零解当且仅当系数行列式为零,即
    \[\begin{vmatrix}
        \lambda-2&-3&-2\\
        -1&\lambda-8&-2\\
        2&14&\lambda+3
    \end{vmatrix}=0\]
    而
    \[\begin{aligned}
        \begin{vmatrix}
            \lambda-2&-3&-2\\
            -1&\lambda-8&-2\\
            2&14&\lambda+3
        \end{vmatrix}
        &= \begin{vmatrix}
                \lambda-2&-3&-2\\
                -1&\lambda-8&-2\\
                0&2\lambda-2&\lambda-1
            \end{vmatrix}=
            \begin{vmatrix}
                \lambda-2&-3&-2\\
                -1&\lambda-8&-2\\
                0&2\lambda-2&\lambda-1
            \end{vmatrix} \\
        &= \begin{vmatrix}
                \lambda-2&1&-2\\
                -1&\lambda-4&-2\\
                0&0&\lambda-1
            \end{vmatrix} = 
            {\color{red}(\lambda-1)\begin{vmatrix}
                \lambda-2&1\\
                -1&\lambda-4
            \end{vmatrix}} \\
        &{\color{red}=} {\color{red}(\lambda-1)(\lambda^2-6\lambda+9)}
    \end{aligned}\]
    当且仅当{\color{red}$\lambda=1$或$\lambda=3$}时系数行列式为$0$,此时原方程有非零解.
\end{solution}
\begin{homework}[6]
    对于本节\tbf{5}中的方程,求$a,b$使得方程组无解/有无穷多解.
\end{homework}
\begin{solution}
    我们只需讨论\tbf{5}结论以外的情形.当$b=0$时,方程组的增广矩阵可以变换如下:
    \[\begin{bmatrix}
        a&1&1&2\\
        1&b&1&1\\
        1&2b&1&2
    \end{bmatrix}=\begin{bmatrix}
        a&1&1&2\\
        1&0&1&1\\
        1&0&1&2
    \end{bmatrix}\longrightarrow\begin{bmatrix}
        a&1&1&2\\
        0&1&0&1\\
        0&0&0&1
    \end{bmatrix}\]
    出现$0=d$类型的行,因此原方程无解.\\当$a=1$时,对方程组的增广矩阵进行初等行变换可得:
    \[\begin{bmatrix}
        a&1&1&2\\
        1&b&1&1\\
        1&2b&1&2
    \end{bmatrix}=\begin{bmatrix}
        1&1&1&2\\
        1&b&1&1\\
        1&2b&1&2
    \end{bmatrix}\longrightarrow\begin{bmatrix}
        1&1&1&2\\
        0&b-1&0&-1\\
        0&2b-1&0&0
    \end{bmatrix}\]
    如果$b=\dfrac12$,那么非零行数目为$2$,小于变量数,原方程组有无穷多解.否则继续行变换:
    \[\begin{bmatrix}
        1&1&1&2\\
        0&b-1&0&-1\\
        0&2b-1&0&0
    \end{bmatrix}\longrightarrow
    \begin{vmatrix}
        1&1&1&2\\
        0&b-1&0&-1\\
        0&0&0&\frac{2b-1}{b-1}
    \end{vmatrix}\]
    无论$b$是否为$1$,都会出现$0=d$的情形,因此此时方程无解.\\
    综上所述,{\color{red}当$b=0$或$a=1,b\neq1/2$}时原方程组无解;当$b\neq0$且$a=1$时原方程组有无穷多解
\end{solution}
\end{document}