\documentclass{ctexart}
\usepackage{note}
\begin{document}\pagestyle{empty}
\begin{center}
    \tbf{\Large 北京大学数学科学学院2024-25学年第一学期线性代数A期中试题}
\end{center}

\begin{homework}[1(20')]
    计算下列行列式的值.
    \begin{enumerate}[label=\tbf{(\arabic*)},topsep=0pt,parsep=0pt,itemsep=0pt,partopsep=0pt]
        \item \[\begin{vmatrix}
            1&1&1&1\\
            1&1&-1&-1\\
            1&-1&1&-1\\
            1&-1&-1&1
        \end{vmatrix}\]
        \item \[\begin{vmatrix}
            5&3&0&\cdots&0&0\\
            2&5&3&\cdots&0&0\\
            0&2&5&\cdots&0&0\\
            \vdots&\vdots&\vdots&\ddots&\vdots&\vdots\\
            0&0&0&\cdots&5&3\\
            0&0&0&\cdots&2&5
        \end{vmatrix}\]
        \item \[\begin{vmatrix}
            \frac{1-a_1^nb_1^n}{1-a_1b_1}&\frac{1-a_1^nb_2^n}{1-a_1b_2}&\cdots&\frac{1-a_1^nb_n^n}{1-a_1b_n}\\
            \frac{1-a_2^nb_1^n}{1-a_2b_1}&\frac{1-a_2^nb_2^n}{1-a_2b_2}&\cdots&\frac{1-a_2^nb_n^n}{1-a_2b_n}\\
            \vdots&\vdots&\ddots&\vdots\\
            \frac{1-a_n^nb_1^n}{1-a_nb_1}&\frac{1-a_n^nb_2^n}{1-a_nb_2}&\cdots&\frac{1-a_n^nb_n^n}{1-a_nb_n}
        \end{vmatrix}\]
    \end{enumerate}
\end{homework}
\begin{solution}
\begin{enumerate}[label=\tbf{(\arabic*)},topsep=0pt,parsep=0pt,itemsep=0pt,partopsep=0pt]
    \item 有
    \[\text{原行列式}=\begin{vmatrix}
        1&1&1&1\\
        0&0&-2&-2\\
        0&-2&0&-2\\
        0&-2&-2&0
    \end{vmatrix}=1\begin{vmatrix}
        0&-2&-2\\
        -2&0&-2\\
        -2&-2&0
    \end{vmatrix}=2\begin{vmatrix}
        -2&-2\\-2&0
    \end{vmatrix}-2\begin{vmatrix}
        -2&0\\-2&-2
    \end{vmatrix}=-16\]
    \item 记原行列式为$D_n$,其中$n$为行列式的级数.将行列式按第一行展开可得
    \[D_n=5D_{n-1}-3\begin{vmatrix}
        2&3&\cdots&0&0\\
        0&5&\cdots&0&0\\
        \vdots&\vdots&\ddots&\vdots&\vdots\\
        0&0&\cdots&5&3\\
        0&0&\cdots&2&5
    \end{vmatrix}=5D_{n-1}-6D_{n-2}\]
    递推式的特征方程$\lambda^2-5\lambda+6=0$的特征根为$\lambda_1=2,\lambda_2=3$.于是设$D_n=A2^n+B3^n$,注意到$D_1=5,D_2=19$,于是解得$A=-2,B=3$,即
    \[D_n=3^{n+1}-2^{n+1}\]
    \item 注意到
    \[1-x^n=(1-x)(1+x+\cdots+x^{n-1})\]
    于是
    \[\text{原行列式}=\begin{vmatrix}
        1+a_1b_1+\cdots+a_1^nb_1^n&\cdots&1+a_1b_n+\cdots+a_1^nb_n^n\\
        \vdots&\ddots&\vdots\\
        1+a_nb_1+\cdots+a_n^nb_1^n&\cdots&1+a_nb_n+\cdots+a_n^nb_n^n
    \end{vmatrix}\]
    将每一列拆分为$n$项,则原行列式可以拆分为$n^n$个行列式之和.当某一列选择$\begin{bmatrix}
        a_1^jb_i^j&\cdots&a_n^jb_i^j
    \end{bmatrix}^{\text{t}}$时,其余列均不能选择含有$a_k^j$项的列,否则会使两列成倍数而使该项为$0$.如此,非零的项必然有
    \[\begin{vmatrix}
        a_1^{i_1}b_1^{i_1}&\cdots&a_1^{i_n}b_n^{i_n}\\
        \vdots&\ddots&\vdots\\
        a_n^{i_1}b_1^{i_1}&\cdots&a_n^{i_n}b_n^{i_n}
    \end{vmatrix}\]
    的形式,其中$i_1\cdots i_n$是$0$到$n-1$的一个排列.而
    \[\begin{vmatrix}
        a_1^{i_1}b_1^{i_1}&\cdots&a_1^{i_n}b_n^{i_n}\\
        \vdots&\ddots&\vdots\\
        a_n^{i_1}b_1^{i_1}&\cdots&a_n^{i_n}b_n^{i_n}
    \end{vmatrix}=\prod_{j=1}^{n}b_j^{i_j}(-1)^{\tau\left(i_1\cdots i_n\right)}\begin{vmatrix}
        1&a_1&\cdots&a_1^{n-1}\\
        \vdots&\vdots&\ddots&\vdots\\
        1&a_n&\cdots&a_n^{n-1}
    \end{vmatrix}
    =\prod_{j=1}^{n}b_j^{i_j}(-1)^{\tau\left(i_1\cdots i_n\right)}\prod_{1\leqslant i<j\leqslant n}\left(a_j-a_i\right)\]
    于是
    \[\begin{aligned}
        \text{原行列式}
        &=\sum_{i_1\cdots i_n}\left(\prod_{j=1}^{n}b_j^{i_j}(-1)^{\tau\left(i_1\cdots i_n\right)}\prod_{1\leqslant i<j\leqslant n}\left(a_j-a_i\right)\right)\\
        &=\begin{vmatrix}
        1&\cdots&1\\
        b_1&\cdots&b_n\\
        \vdots&\ddots&\vdots\\
        b_1^{n-1}&\cdots&b_n^{n-1}
    \end{vmatrix}\prod_{1\leqslant i<j\leqslant n}\left(a_j-a_i\right)\\
    &= \prod_{1\leqslant i<j\leqslant n}\left(a_j-a_i\right)\prod_{1\leqslant i<j\leqslant n}\left(b_j-b_i\right)
    \end{aligned}\]
\end{enumerate}
\end{solution}
\begin{homework}[2(18')]
    设
    \[\bs\alpha_1=\begin{bmatrix}
        1\\3\\-5\\-9
    \end{bmatrix},\quad\bs\alpha_2=\begin{bmatrix}
        2\\-1\\-3\\-4
    \end{bmatrix},\quad\bs\alpha_3=\begin{bmatrix}
        -3\\5\\1\\-1
    \end{bmatrix},\quad\bs\alpha_4=\begin{bmatrix}
        -4\\6\\2\\0
    \end{bmatrix},\quad\bs\beta=\begin{bmatrix}
        -5\\-1\\11\\17
    \end{bmatrix}\]
    \begin{enumerate}[label=\tbf{(\arabic*)},topsep=0pt,parsep=0pt,itemsep=0pt,partopsep=0pt]
        \item 求向量组$\bs\alpha_1,\bs\alpha_2,\bs\alpha_3,\bs\alpha_4$的一个极大线性无关组.
        \item 判断$\bs\beta$是否可以被$\bs\alpha_1,\bs\alpha_2,\bs\alpha_3,\bs\alpha_4$线性表出.如果可以,请给出所有的表出方式.
    \end{enumerate}
\end{homework}
\begin{solution}
\begin{enumerate}[label=\tbf{(\arabic*)},topsep=0pt,parsep=0pt,itemsep=0pt,partopsep=0pt]
    \item 考虑$\bs\alpha_1,\bs\alpha_2$的前两个分量构成的矩阵的行列式:
    \[\begin{vmatrix}
        1&2\\3&-1
    \end{vmatrix}=-7\neq0\]
    于是$\bs\alpha_1,\bs\alpha_2$线性无关.观察到$\bs\alpha_3=\bs\alpha_1-2\bs\alpha_2$.于是$\bs\alpha_3$不在组中.考虑$\bs\alpha_1,\bs\alpha_2,\bs\alpha_4$的前三个分量构成的矩阵的行列式:
    \[\begin{vmatrix}
        1&2&-4\\
        3&-1&6\\
        -5&1&2
    \end{vmatrix}=\begin{vmatrix}
        -9&4&0\\
        18&-4&0\\
        -5&1&2
    \end{vmatrix}=2\begin{vmatrix}
        -9&4\\18&-4
    \end{vmatrix}=-72\neq0\]
    于是$\bs\alpha_1,\bs\alpha_2,\bs\alpha_4$线性无关.\\
    于是$\bs\alpha_1,\bs\alpha_2,\bs\alpha_4$是原向量组的一个极大线性无关组.
    \item 考虑线性方程组
    \[\bs\beta=x_1\bs\alpha_1+x_2\bs\alpha_2+x_3\bs\alpha_3+x_4\bs\alpha_4\]
    对其增广矩阵做初等行变换可得
    \[\begin{vmatrix}
        1&2&-3&-4&-5\\
        3&-1&5&6&-1\\
        -5&-3&1&2&11\\
        -9&-4&-1&0&17
    \end{vmatrix}\longrightarrow\begin{vmatrix}
        1&2&-3&-4&-5\\
        0&-7&14&18&14\\
        0&7&-14&-18&-14\\
        0&14&-28&-36&-28
    \end{vmatrix}\longrightarrow\begin{vmatrix}
        1&2&-3&-4&-5\\
        0&-7&14&18&14\\
        0&0&0&0&0\\
        0&0&0&0&0\\
    \end{vmatrix}\]
    于是原方程组的解为
    \[\left\{\begin{array}{l}
        x_1=1-x_3-\frac{8}{7}x_4\\
        x_2=2+2x_3+\frac{18}{7}x_4
    \end{array}\right.\]
    于是所有表出方式可以写做
    \[\bs\beta=\left(1-a-\dfrac87b\right)\bs\alpha_1+\left(2+2a+\dfrac{18}{7}b\right)\bs\alpha_2+a\bs\alpha_3+b\bs\alpha_4,\quad a,b\in K\]
\end{enumerate}
\end{solution}
\begin{homework}[3(10')]
    求$\lambda$使得矩阵
    \[\mat{A}=\begin{bmatrix}
        3&1&1&4\\
        \lambda&4&10&1\\
        1&7&17&3\\
        2&2&4&3
    \end{bmatrix}\]
    的秩最小.
\end{homework}
\begin{solution}
    对$\mat{A}$做初等行变换可得
    \[\mat{A}\longrightarrow\begin{bmatrix}
        1&7&17&3\\
        0&-20&-50&-5\\
        0&4-7\lambda&10-17\lambda&1-3\lambda\\
        0&-12&-30&-3
    \end{bmatrix}\longrightarrow\begin{bmatrix}
        1&7&17&3\\
        0&4&10&1\\
        0&7\lambda&17\lambda&3\lambda\\
        0&0&0&0
    \end{bmatrix}\]
    当$\lambda=0$时, $\rank\ \mat{A}=2$,否则$\rank\ \mat{A}=3$.于是所求$\lambda=0$.
\end{solution}
\begin{homework}[4(20')]
    设$\mat{A}$是$3\times4$矩阵,齐次线性方程组$\mat{A}\vec{x}=\mbf{0}$的解空间由向量$\begin{bmatrix}
        1&1&1&0
    \end{bmatrix}^{\text{t}}$和$\begin{bmatrix}
        -2&-1&0&1
    \end{bmatrix}^{\text{t}}$生成.
    \begin{enumerate}[label=\tbf{(\arabic*)},topsep=0pt,parsep=0pt,itemsep=0pt,partopsep=0pt]
        \item 求$\mat{A}$对应的简化阶梯形矩阵.
        \item 求以下空间的维数: \tbf{(a)} $\mat{A}$的列空间$C(\mat{A})$; \tbf{(b)} $\mat{A}^{\text{t}}$的列空间$C\left(\mat{A}^{\text{t}}\right)$; \tbf{(c)}齐次线性方程组$\mat{A}^{\text{t}}\vec{x}=\mbf{0}$的解集$N\left(\mat{A}^{\text{t}}\right)$.
        \item 写出\tbf{(2)}中所有可以写出基的空间的一组基.
    \end{enumerate}
\end{homework}
\begin{solution}
\begin{enumerate}[label=\tbf{(\arabic*)},topsep=0pt,parsep=0pt,itemsep=0pt,partopsep=0pt]
    \item 观察基础解系的结构可以写出方程的解为
    \[\left\{\begin{array}{l}
        x_1=x_3-2x_4\\
        x_2=x_3-x_4
    \end{array}\right.\]
    于是$\mat{A}$对应的简化阶梯形矩阵为
    \[\begin{bmatrix}
        1&0&-1&2\\
        0&1&-1&1\\
        0&0&0&0
    \end{bmatrix}\]
    \item 初等行变换不改变列空间的维数.于是$\dim C(\mat{A})=\rank\ \mat{A}=2$.\\
    $\mat{A}^{\text{t}}$的列空间也即$\mat{A}$的行空间,于是$\dim C\left(\mat{A}^{\text{t}}\right)=\rank\ \mat{A}=2$.\\
    $\mat{A}^{\text{t}}\vec{x}=\mbf{0}$的解集$N\left(\mat{A}^{\text{t}}\right)$的维数$\dim N\left(\mat{A}^{\text{t}}\right)=3-\rank\ \mat{A}^{\text{t}}=3-\rank\ \mat{A}=1$.
    \item $C(\mat{A})$的一组基为
    \[\bs\eta_1=\begin{bmatrix}
        1\\0\\0
    \end{bmatrix},\quad\bs\eta_2=\begin{bmatrix}
        0\\1\\0
    \end{bmatrix}\]
    $C(\mat{A}^{\text{t}})$的一组基为
    \[\bs\eta_1=\begin{bmatrix}
        1\\0\\-1&2
    \end{bmatrix},\quad\bs\eta_2=\begin{bmatrix}
        0\\1\\-1&1
    \end{bmatrix}\]
\end{enumerate}
\end{solution}
\begin{homework}[5(10')]
    设$m\times n$矩阵$\mat{A}$满足$\rank\ \mat{A}=r$.如果$\mat{A}$的前$r$行线性无关,前$r$列也线性无关,证明: $\mat{A}$的前$r$行和前$r$列构成的$r$阶子式非零.
\end{homework}
\begin{proof}
    由于$\mat{A}$的秩为$r$并且前$r$行线性无关,于是可以通过初等行变换将其第$r$行以后的行全部消去得到矩阵$\mat{B}$.\\
    由于初等行变换不改变列向量的线性无关性,因此$\mat{B}$的前$r$列仍然线性无关,并且有$\rank\ \mat{A}=\rank\ \mat{B}=r$.于是可以通过初等列变换将$\mat{B}$的第$r$列以后的列全部消去得到矩阵$\mat{C}$.现在$\mat{C}$仅有前$r$行前$r$列的元素非零,并且仍有$\rank\ \mat{C}=r$,于是其前$r$行前$r$列构成的子式非零,从而$\mat{A}$的前$r$行前$r$列元素构成的子式非零,命题得证.
\end{proof}
\begin{homework}[6(10')]
    设$m\times n$矩阵$\mat{A}$和$r\times n$矩阵$\mat{B}$,且齐次线性方程组$\mat{A}\vec{x}=\mbf{0}$和$\mat{B}\vec{x}=\mbf{0}$同解.证明: $\mat{A}$和$\mat{B}$的行向量组等价.
\end{homework}
\begin{proof}
    设$\mathcal{A}=\left\{\bs\alpha_1,\cdots,\bs\alpha_m\right\}$为$\mat{A}$的行向量组, $\mathcal{B}=\left\{\bs\beta_1,\cdots,\bs\beta_m\right\}$为$\mat{B}$的行向量组.记方程组$\mat{A}\vec{x}=\mbf{0}$和$\mat{B}\vec{x}=\mbf{0}$的解集为$W$.\\
    设$\mathcal{A}$的一个极大线性无关组为$\bs\alpha_1,\cdots,\bs\alpha_s$.对于任意$\bs\beta_i\in\mathcal{B}$,考虑以$\bs\alpha_1,\cdots,\bs\alpha_s,\bs\beta_i$为行向量的系数矩阵对应的齐次线性方程组,其解集也应当为$W$.于是
    \[\dim W=n-\rank\ \mat{A}=n-\rank\left\{\bs\alpha_1,\cdots,\bs\alpha_s,\bs\beta_i\right\}\]
    而由于$\bs\alpha_1,\cdots,\bs\alpha_s$是$\mat{A}$的极大线性无关组,因此
    \[\rank\ \mat{A}=\rank\left\{\bs\alpha_1,\cdots,\bs\alpha_s\right\}\]
    于是
    \[\rank\left\{\bs\alpha_1,\cdots,\bs\alpha_s\right\}=\rank\left\{\bs\alpha_1,\cdots,\bs\alpha_s,\bs\beta_i\right\}\]
    由于$\bs\alpha_1,\cdots,\bs\alpha_s$线性无关,于是$\bs\beta_i$能由$\bs\alpha_1,\cdots,\bs\alpha_s$线性表出,进而$\bs\beta_i$能由$\mathcal{A}$线性表出.对所有$1\leqslant i\leqslant r$使用以上结论可知$\mat{B}$能被$\mat{A}$线性表出.\\
    同理可以证得$\mat{A}$能被$\mat{B}$线性表出.于是$\mat{A}$和$\mat{B}$的行向量组等价.
\end{proof}
\begin{homework}[7(12')]
    设$r\times n$矩阵$\mat{A}$, $1\times r$向量$\bs\alpha$和$n\times 1$向量$\bs\beta$满足$\bs\alpha\mat{A}\bs\beta=\begin{bmatrix}a\end{bmatrix}$且$a\neq0$.令
    \[\mat{B}=\mat{A}-a^{-1}\left(\mat{A}\bs\beta\bs\alpha\mat{A}\right)\]
    并记$N(\mat{A}),N(\mat{B})$分别为齐次线性方程组$\mat{A}\vec{x}=\mbf{0}$和$\mat{B}\vec{x}=\mbf{0}$的解空间.证明: $N(\mat{B})$可以由$N(\mat{A})$和$\bs\beta$生成,并进而证明$\rank\ \mat{B}=\rank\ \mat{A}-1$.
\end{homework}
\begin{proof}
    考虑$N(\mat{A})$的基础解系$\li{\bs\eta},s$.于是对任意$1\leqslant i\leqslant s$有
    \[\begin{aligned}
        \mat{B}\bs\eta_i
        &=\left(\mat{A}-a^{-1}\left(\mat{A}\bs\beta\bs\alpha\mat{A}\right)\right)\bs\eta_i\\
        &=\mat{A}\bs\eta_i-a^{-1}\mat{A}\bs\beta\bs\alpha\left(\mat{A}\bs\eta_i\right)\\
        &=\mbf0-a^{-1}\mat{A}\bs\beta\bs\alpha\mbf0\\
        &=\mbf0
    \end{aligned}\]
    于是$\bs\eta_i$为$\mat{B}\vec{x}=0$的解.\\
    又有
    \[\begin{aligned}
        \mat{B}\bs\beta
        &=\left(\mat{A}-a^{-1}\left(\mat{A}\bs\beta\bs\alpha\mat{A}\right)\right)\bs\beta\\
        &=\mat{A}\bs\beta-a^{-1}\mat{A}\bs\beta\left(\bs\alpha\mat{A}\bs\beta\right)\\
        &=\mat{A}\bs\beta-a^{-1}\mat{A}\bs\beta a\\
        &=\mat{A}\bs\beta-\mat{A}\bs\beta\\
        &=\mbf0
    \end{aligned}\]
    于是$\bs\beta$也是$\mat{B}\vec{x}=0$的解.又因为$\bs\alpha\mat{A}\bs\beta=\begin{bmatrix}a\end{bmatrix}$并且$a\neq0$,于是$\mat{A}\bs\beta\neq\mbf0$,因此$\bs\beta$与$\li{\bs\eta},s$线性无关.\\
    于是$\bs\beta$与$\li{\bs\eta},s$是$N(\mat{B})$中的线性无关组.为证明它们是$N(\mat{B})$的基,考虑$\bs\gamma\in N(\mat{B})$,设$\bs\alpha\mat{A}\bs\gamma=b$,则有
    \[\mbf0=\mat{B}\bs\gamma=\mat{A}\bs\gamma+a^{-1}\mat{A}\bs\beta c=\mat{A}\left(\bs\gamma-\dfrac{c}{a}\bs\beta\right)\]
    于是$\bs\gamma-\dfrac{c}{a}\bs\beta\in N(\mat{A})$.于是任意$\bs\gamma\in N(\mat{B})$都能由向量组$\bs\beta,\li{\bs\eta},s$线性表出,而该向量组本身线性无关.于是该向量组是$N(\mat{B})$的基.于是$N(\mat{B})$能由$N(\mat{A})$和$\bs\beta$生成.\\
    现在有
    \[\dim\ N(\mat{B})=n-\rank\ \mat{B}=s+1\]
    \[\dim\ N(\mat{A})=n-\rank\ \mat{A}=s\]
    于是
    \[\rank\ \mat{B}=\rank\ \mat{A}-1\]
\end{proof}
\end{document}