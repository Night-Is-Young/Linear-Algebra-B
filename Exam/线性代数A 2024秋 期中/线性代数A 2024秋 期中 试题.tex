\documentclass{ctexart}
\usepackage{note}
\begin{document}\pagestyle{empty}
\begin{center}
    \tbf{\Large 北京大学数学科学学院2024-25学年第一学期线性代数A期中试题}
\end{center}
\begin{homework}[1(20')]
    计算下列行列式的值.
    \begin{enumerate}[label=\tbf{(\arabic*)},topsep=0pt,parsep=0pt,itemsep=0pt,partopsep=0pt]
        \item \[\begin{vmatrix}
            1&1&1&1\\
            1&1&-1&-1\\
            1&-1&1&-1\\
            1&-1&-1&1
        \end{vmatrix}\]
        \item \[\begin{vmatrix}
            5&3&0&\cdots&0&0\\
            2&5&3&\cdots&0&0\\
            0&2&5&\cdots&0&0\\
            \vdots&\vdots&\vdots&\ddots&\vdots&\vdots\\
            0&0&0&\cdots&5&3\\
            0&0&0&\cdots&2&5
        \end{vmatrix}\]
        \item \[\begin{vmatrix}
            \frac{1-a_1^nb_1^n}{1-a_1b_1}&\frac{1-a_1^nb_2^n}{1-a_1b_2}&\cdots&\frac{1-a_1^nb_n^n}{1-a_1b_n}\\
            \frac{1-a_2^nb_1^n}{1-a_2b_1}&\frac{1-a_2^nb_2^n}{1-a_2b_2}&\cdots&\frac{1-a_2^nb_n^n}{1-a_2b_n}\\
            \vdots&\vdots&\ddots&\vdots\\
            \frac{1-a_n^nb_1^n}{1-a_nb_1}&\frac{1-a_n^nb_2^n}{1-a_nb_2}&\cdots&\frac{1-a_n^nb_n^n}{1-a_nb_n}
        \end{vmatrix}\]
    \end{enumerate}
\end{homework}
\begin{homework}[2(18')]
    设
    \[\bs\alpha_1=\begin{bmatrix}
        1\\3\\-5\\-9
    \end{bmatrix},\quad\bs\alpha_2=\begin{bmatrix}
        2\\-1\\-3\\-4
    \end{bmatrix},\quad\bs\alpha_3=\begin{bmatrix}
        -3\\5\\1\\-1
    \end{bmatrix},\quad\bs\alpha_4=\begin{bmatrix}
        -4\\6\\2\\0
    \end{bmatrix},\quad\bs\beta=\begin{bmatrix}
        -5\\-1\\11\\17
    \end{bmatrix}\]
    \begin{enumerate}[label=\tbf{(\arabic*)},topsep=0pt,parsep=0pt,itemsep=0pt,partopsep=0pt]
        \item 求向量组$\bs\alpha_1,\bs\alpha_2,\bs\alpha_3,\bs\alpha_4$的一个极大线性无关组.
        \item 判断$\bs\beta$是否可以被$\bs\alpha_1,\bs\alpha_2,\bs\alpha_3,\bs\alpha_4$线性表出.如果可以,请给出所有的表出方式.
    \end{enumerate}
\end{homework}
\begin{homework}[3(10')]
    求$\lambda$使得矩阵
    \[\mat{A}=\begin{bmatrix}
        3&1&1&4\\
        \lambda&4&10&1\\
        1&7&17&3\\
        2&2&4&3
    \end{bmatrix}\]
    的秩最小.
\end{homework}
\begin{homework}[4(20')]
    设$\mat{A}$是$3\times4$矩阵,齐次线性方程组$\mat{A}\vec{x}=\mbf{0}$的解空间由向量$\begin{bmatrix}
        1&1&1&0
    \end{bmatrix}^{\text{t}}$和$\begin{bmatrix}
        -2&-1&0&1
    \end{bmatrix}^{\text{t}}$生成.
    \begin{enumerate}[label=\tbf{(\arabic*)},topsep=0pt,parsep=0pt,itemsep=0pt,partopsep=0pt]
        \item 求$\mat{A}$对应的简化阶梯形矩阵.
        \item 求以下空间的维数: \tbf{(a)} $\mat{A}$的列空间$C(\mat{A})$; \tbf{(b)} $\mat{A}^{\text{t}}$的列空间$C\left(\mat{A}^{\text{t}}\right)$; \tbf{(c)}齐次线性方程组$\mat{A}^{\text{t}}\vec{x}=\mbf{0}$的解集$N\left(\mat{A}^{\text{t}}\right)$.
        \item 写出\tbf{(2)}中所有可以写出基的空间的一组基.
    \end{enumerate}
\end{homework}
\begin{homework}[5(10')]
    设$m\times n$矩阵$\mat{A}$满足$\rank\ \mat{A}=r$.如果$\mat{A}$的前$r$行线性无关,前$r$列也线性无关,证明: $\mat{A}$的前$r$行和前$r$列构成的$r$阶子式非零.
\end{homework}
\begin{homework}[6(10')]
    设$m\times n$矩阵$\mat{A}$和$r\times n$矩阵$\mat{B}$,且齐次线性方程组$\mat{A}\vec{x}=\mbf{0}$和$\mat{B}\vec{x}=\mbf{0}$同解.证明: $\mat{A}$和$\mat{B}$的行向量组等价.
\end{homework}
\begin{homework}[7(12')]
    设$r\times n$矩阵$\mat{A}$, $1\times r$向量$\bs\alpha$和$n\times 1$向量$\bs\beta$满足$\bs\alpha\mat{A}\bs\beta=a$且$a\neq0$.令
    \[\mat{B}=\mat{A}-a^{-1}\left(\mat{A}\bs\beta\bs\alpha\mat{A}\right)\]
    并记$N(\mat{A}),N(\mat{B})$分别为齐次线性方程组$\mat{A}\vec{x}=\mbf{0}$和$\mat{B}\vec{x}=\mbf{0}$的解空间.证明: $N(\mat{B})$可以由$N(\mat{A})$和$\bs\beta$生成,并进而证明$\rank\ \mat{B}=\rank\ \mat{A}-1$.
\end{homework}
\end{document}