\documentclass{ctexart}
\usepackage{Note}
\begin{document}
\section{二次型}
\begin{problem}
    求实二次型
    \[f(x_1,x_2,x_3)=x_1^2+x_2^2+tx_1x_2+2tx_1x_3+4x_2x_3\]
    的秩和正惯性指数.
\end{problem}
\begin{solution}
    这二次型的矩阵为
    \[\mat{A}=\begin{bmatrix}
        1&\frac{t}{2}&t\\
        \frac{t}{2}&1&2\\
        t&2&0
    \end{bmatrix}\]
    它的顺序主子式分别为
    \[\begin{vmatrix}
        1
    \end{vmatrix}=1,\quad\begin{vmatrix}
        1&\frac{t}{2}\\
        \frac{t}{2}&1
    \end{vmatrix}=1-\dfrac{t^2}{4},\quad\begin{vmatrix}
        1&\frac{t}{2}&t\\
        \frac{t}{2}&1&2\\
        t&2&0
    \end{vmatrix}=t^2-4\]
    如果$t^2=4$,则$\mat{A}$不满秩.此时$\mat{A}$的$(1,1)$元的余子式非零,因此$\rank\mat{A}=2$,正惯性指数为$1$.\\
    如果$t^2\neq4$,则矩阵满秩,只需判断$\mat{A}$的特征值的正负. $\mat{A}$的特征多项式为
    \[\lambda^3-2\lambda^2+\left(-\dfrac{5t^2}{4}-3\right)\lambda+(4-t^2)\]
    设它的三个实根为$\lambda_1,\lambda_2,\lambda_3$,则有
    \[\lambda_1+\lambda_2+\lambda_3=2,\quad\lambda_1\lambda_2+\lambda_1\lambda_3+\lambda_2\lambda_3=-\dfrac{5t^2}{4}-3,\quad\lambda_1\lambda_2\lambda_3=t^2-4\]
    前两个式子表明三个根中有正数也有负数.如果$t^2>4$,则正惯性指数为$1$;如果$t^2<4$,则正惯性指数为$2$.
\end{solution}
\begin{problem}
    判断当参数$a$满足什么条件时,实二次型
    \[f(x_1,x_2,x_3)=a(x_1^2+x_2^2+x_3^2)+2(x_1x_2+x_1x_3+x_2x_3)\]
    是正定的.
\end{problem}
\begin{solution}
    这二次型的矩阵为
    \[\mat{A}=\begin{bmatrix}
        a&1&1\\
        1&a&1\\
        1&1&a
    \end{bmatrix}\]
    其特征多项式为
    \[\begin{aligned}
        \det(\lambda\mat{I}-\mat{A})
        &=\begin{vmatrix}
            \lambda-a&-1&-1\\
            -1&\lambda-a&-1\\
            -1&-1&\lambda-a
        \end{vmatrix}\\
        &=(\lambda-a)(\lambda-a-1)(\lambda-a+1)+(-\lambda+a-1)-(\lambda-a+1)\\
        &=(\lambda-a+1)(\lambda^2-(2a+1)\lambda+a^2+a-2)\\
        &=(\lambda-a+1)^2(\lambda-a-2)
    \end{aligned}\]
    上述二次型正定当且仅当$\mat{A}$的所有特征值大于零,即
    \[a-1>0,\quad a+2>0\]
    从而$a>1$.
\end{solution}
\begin{problem}
    设实二次型
    \[f(x_1,x_2,x_3)=x_1^2+x_2^2+x_3^2+2ax_1x_2+2ax_2x_3+2ax_1x_3\]
    可以做非退化线性变换成为
    \[g(y_1,y_2,y_3)=y_1^2+y_2^2+4y_3^2+2y_1y_2\]
    求$a$的值和相应的非退化线性变换.
\end{problem}
\begin{solution}
    二次型$g=(y_1+y_2)^2+4y_3^2$,其秩和正惯性指数均为$2$.考虑二次型$f$对应的矩阵:
    \[\mat{A}=\begin{bmatrix}
        1&a&a\\
        a&1&a\\
        a&a&1
    \end{bmatrix}\]
    于是
    \[\det\mat{A}=(1-a^2)-a(a-a^2)+a(a^2-a)=(a-1)^2(2a+1)\]
    当$a=1$时, $\rank\mat{A}=1$,不存在相应的变换,于是只能有$a=-\dfrac12$.现在求相应的非退化线性变换:
    \[\begin{bmatrix}
        1&-\frac12&-\frac12\\
        -\frac12&1&-\frac12\\
        -\frac12&-\frac12&1\\
        1&0&0\\
        0&1&0\\
        0&0&1
    \end{bmatrix}\longrightarrow\begin{bmatrix}
        1&0&-\frac12\\
        0&\frac34&-\frac34\\
        -\frac12&-\frac34&1\\
        1&\frac12&0\\
        0&1&0\\
        0&0&1
    \end{bmatrix}\longrightarrow\begin{bmatrix}
        1&0&0\\
        0&\frac34&-\frac34\\
        0&-\frac34&\frac34\\
        1&\frac12&\frac12\\
        0&1&0\\
        0&0&1
    \end{bmatrix}\longrightarrow\begin{bmatrix}
        1&0&0\\
        0&\frac34&0\\
        0&0&0\\
        1&\frac12&1\\
        0&1&1\\
        0&0&1
    \end{bmatrix}\]
    于是令
    \[\mat{C}=\begin{bmatrix}
        1&\frac12&1\\
        0&1&1\\
        0&0&1
    \end{bmatrix}\]
    然后令$\vec{x}=\mat{C}\vec{z}$,就有$f(x_1,x_2,x_3)=z_1^2+\dfrac34z_2^2$,再令
    \[\vec{z}=\begin{bmatrix}
        z_1\\z_2\\z_3
    \end{bmatrix}=\begin{bmatrix}
        y_1+y_2\\
        \frac{16}{\sqrt3}y_3\\
        0
    \end{bmatrix}\]
    即可.
\end{solution}
\begin{problem}
    设$\mat{A}$是正定矩阵, $\mat{B}$是与$\mat{A}$阶数相同的实对称矩阵.证明:存在可逆矩阵$\mat{C}$使得$\mat{C}^\t\mat{A}\mat{C}$与$\mat{C}^\t\mat{B}\mat{C}$都是对角矩阵.
\end{problem}
\begin{proof}
    由于$\mat{A}$是正定矩阵,因此它合同于$\mat{I}_n$,即存在可逆矩阵$\mat{P}$使得
    \[\mat{P}^\t\mat{A}\mat{P}=\mat{I}\]
    由于$\mat{P}^\t\mat{B}\mat{P}$也是实对称矩阵,因此存在可逆矩阵$\mat{Q}$使得
    \[\mat{Q}^\t(\mat{P}^\t\mat{B}\mat{P})\mat{Q}\]
    也为对角矩阵.令$\mat{C}=\mat{Q}^\t\mat{P}^\t$为可逆矩阵,则有
    \[\mat{C}^\t\mat{A}\mat{C}=\mat{Q}^\t\mat{P}^\t\mat{A}\mat{P}\mat{Q}=\mat{Q}^\t\mat{I}\mat{Q}=\mat{I}\]
    于是存在可逆的$\mat{C}$使得$\mat{C}^\t\mat{A}\mat{C}$和$\mat{C}^\t\mat{B}\mat{C}$均为对角矩阵.
\end{proof}
\begin{problem}
    设$\mat{A},\mat{B}$是两个半正定矩阵.证明: $\mat{A}\mat{B}=\mbf0$当且仅当$\tr(\mat{A}\mat{B})=0$.
\end{problem}
\begin{proof}
    $\Rightarrow$:显然.\\
    $\Leftarrow$:现在假设$\tr(\mat{A}\mat{B})=0$.由于$\mat{A}$是半正定的,因此存在可逆矩阵$\mat{P}$使得
    \[\mat{A}=\mat{P}^\t\diag\{\li\lambda,s,0,\cdots,0\}\mat{P}\]
    令
    \[\mat{S}=\mat{P}^\t\diag\{\sqrt{\lambda_1},\cdots,\sqrt{\lambda_s},0,\cdots,0\}\mat{P}\]
    则
    \[\mat{S}^2=\mat{A}\]
    同理构造$\mat{T}$使得$\mat{T}^2=\mat{B}$.并且$\mat{S},\mat{T}$都是半正定矩阵.于是
    \[\tr(\mat{A}\mat{B})=\tr(\mat{S}\mat{S}\mat{B})=\tr(\mat{S}\mat{B}\mat{S})=\tr(\mat{S}^\t\mat{B}\mat{S})\]
    $\mat{S}^\t\mat{B}\mat{S}$是一个半正定矩阵,它的迹为$0$说明它的特征值全部为$0$;它又可以对角化,于是$\mat{S}\mat{B}\mat{S}=\mat{S}\mat{T}\mat{T}\mat{S}=\mbf0$.而$\mat{S}\mat{T}(\mat{S}\mat{T})^\t=\mat{S}\mat{T}\mat{T}\mat{S}=\mbf0$,于是$\mat{S}\mat{T}=\mbf0$,于是$\mat{A}\mat{B}=\mat{S}\mat{S}\mat{T}\mat{T}=\mbf0$.
\end{proof}
\begin{problem}
    设$\mat{A}$是正定实矩阵.证明:存在对角元均为正数的上三角矩阵$\mat{L}$使得$\mat{A}=\mat{L}^\t\mat{L}$.
\end{problem}
\begin{proof}
    对$\mat{A}$的阶数$n$做归纳.\\
    当$n=1$时,命题显然成立.\\
    当$n>1$时,根据归纳假设,命题对所有阶数小于$n$的矩阵均成立.将$\mat{A}$做如下分块:
    \[\mat{A}=\begin{bmatrix}
        \mat{A}_1&\bs\alpha\\
        \bs\alpha^\t&a_{nn}
    \end{bmatrix}\]
    由于$\mat{A}_1$的顺序主子式也是$\mat{A}$的顺序主子式,因此$\mat{A}_1$也是正定矩阵,于是存在对角元均为正数的上三角矩阵$\mat{L}_1$使得$\mat{L}_1^\t\mat{L}_1=\mat{A}_1$.由于$\mat{L}_1$的对角元为正数,于是$\det\mat{L}_1>0$,因此$\mat{L}_1$可逆.\\
    考虑行列变换
    \[\begin{bmatrix}
        \mat{I}_{n-1}&\mbf0\\
        -\bs\alpha^t\mat{A}_1^{-1}&1
    \end{bmatrix}\begin{bmatrix}
        \mat{A}_1&\bs\alpha\\
        \bs\alpha^\t&a_{nn}
    \end{bmatrix}\begin{bmatrix}
        \mat{I}_{n-1}&-\mat{A}_1^{-1}\bs\alpha\\
        \mbf0&1
    \end{bmatrix}=\begin{bmatrix}
        \mat{A}_1&\mbf0\\
        \mbf0&a_{nn}-\bs\alpha^t\mat{A}_1^{-1}\bs\alpha
    \end{bmatrix}\]
    两边取行列式可得
    \[\det\mat{A}=\det\mat{A}_1(a_{nn}-\bs\alpha^\t\mat{A}_1^{-1}\bs\alpha)\]
    由于$\det\mat{A}>0,\det\mat{A}_1>0$,于是
    \[a_{nn}-\bs\alpha^\t\mat{A}_1^{-1}\bs\alpha>0\]
    记上式为$b$.而
    \[\begin{bmatrix}
        \mat{A}_1&\mbf0\\
        \mbf0&b
    \end{bmatrix}=\begin{bmatrix}
        \mat{L}_1^\t&\mbf0\\
        \mbf0&\sqrt{b}
    \end{bmatrix}\begin{bmatrix}
        \mat{L}_1&\mbf0\\
        \mbf0&\sqrt{b}
    \end{bmatrix}\]
    于是令
    \[\mat{L}=\begin{bmatrix}
        \mat{L}_1&\mbf0\\
        \mbf0&\sqrt{b}
    \end{bmatrix}\begin{bmatrix}
        \mat{I}_{n-1}&-\mat{A}_1^{-1}\bs\alpha\\
        \mbf0&1
    \end{bmatrix}^{-1}\]
    即有$\mat{A}=\mat{L}^\t\mat{L}$,归纳可知命题对所有正定矩阵$\mat{A}$都成立.
\end{proof}
\begin{problem}
    设$\mat{A}$是$n$阶正定矩阵,证明:二次型
    \[f(x_1,x_2,\cdots,x_n)=\det\begin{bmatrix}
        \mat{A}&\vec{x}\\
        \vec{x}^\t&\mbf0
    \end{bmatrix}\]
    是负定的.
\end{problem}
\begin{proof}
    只需证明对任意$\vec{x}$有$f(\vec{x})<0$即可.根据前一题的变换方式,注意到
    \[\begin{bmatrix}
        \mat{I}_{n}&\mbf0\\
        -\vec{x}^\t\mat{A}^{-1}&1
    \end{bmatrix}\begin{bmatrix}
        \mat{A}&\vec{x}\\
        \vec{x}^\t&0
    \end{bmatrix}\begin{bmatrix}
        \mat{I}_{n}&-\mat{A}^{-1}\vec{x}\\
        \mbf0&1
    \end{bmatrix}=\begin{bmatrix}
        \mat{A}&\mbf0\\
        \mbf0&-\vec{x}^t\mat{A}^{-1}\vec{x}
    \end{bmatrix}\]
    两边取行列式可得
    \[f(\vec{x})=-\det\mat{A}(\vec{x}^\t\mat{A}^{-1}\vec{x})\]
    由于$\mat{A}$正定,因此$\det\mat{A}>0,\vec{x}^\t\mat{A}^{-1}\vec{x}>0$,因此$f(\vec{x})<0$,于是题设二次型负定.
\end{proof}
\begin{problem}
    设$\mat{A}=(a_{ij})$是$n$级正定矩阵.证明:
    \[\det\mat{A}\leq\prod_{i=1}^{n}a_{ii}\]
\end{problem}
\begin{proof}
    对$\mat{A}$的阶数$n$采用数学归纳法.\\
    当$n=1$时命题显然成立.\\
    当$n>1$时对$\mat{A}$分块如下:
    \[\mat{A}=\begin{bmatrix}
        \mat{A}_1&\bs\alpha\\
        \bs\alpha^\t&a_{nn}
    \end{bmatrix}\]
    考虑行列变换
    \[\begin{bmatrix}
        \mat{I}_{n-1}&\mbf0\\
        -\bs\alpha^t\mat{A}_1^{-1}&1
    \end{bmatrix}\begin{bmatrix}
        \mat{A}_1&\bs\alpha\\
        \bs\alpha^\t&a_{nn}
    \end{bmatrix}\begin{bmatrix}
        \mat{I}_{n-1}&-\mat{A}_1^{-1}\bs\alpha\\
        \mbf0&1
    \end{bmatrix}=\begin{bmatrix}
        \mat{A}_1&\mbf0\\
        \mbf0&a_{nn}-\bs\alpha^t\mat{A}_1^{-1}\bs\alpha
    \end{bmatrix}\]
    两边取行列式可得
    \[\det\mat{A}=\det\mat{A}_1(a_{nn}-\bs\alpha^\t\mat{A}_1^{-1}\bs\alpha)\leq a_{nn}\det\mat{A}_1\leq a_{nn}\prod_{i=1}^{n-1}a_{ii}=\prod_{i=1}^{n}a_{ii}\]
\end{proof}
\begin{problem}
    设$\mat{C}=(c_{ij})$是$n$级实可逆矩阵.证明:
    \[(\det\mat{C})^2\leq\prod_{i=1}^{n}\left(\sum_{j=1}^{n}c_{ji}^2\right)\]
\end{problem}
\begin{proof}
    首先注意到$\mat{C}^\t\mat{C}$是正定矩阵,于是利用上一问的结论即可得证.
\end{proof}
\begin{problem}
    设$\mat{A},\mat{B}$是阶数相同的半正定矩阵.证明:
    \[\det(\mat{A}+\mat{B})\geq\max\{\det\mat{A},\det\mat{B}\}\]
\end{problem}
\begin{proof}
    如果$\mat{A},\mat{B}$都不是正定矩阵,那么$\det\mat{A}=\det\mat{B}=0$,而$\mat{A}+\mat{B}$也是半正定矩阵,因此其行列式非负,于是命题得证.\\
    现在不妨假设$\mat{A}$是正定矩阵.由前面题目的结论可知存在可逆矩阵$\mat{C}$使得$\mat{C}^\t\mat{A}\mat{C}$和$\mat{C}^\t\mat{B}\mat{C}$都是对角矩阵,不妨分别设为$\mat{S},\mat{T}$,并且对角元都非负.于是
    \[\det\mat{S}=\prod_{i=1}^{n}s_{ii}=(\det\mat{C})^2\det\mat{A}\]
    \[\det\mat{T}=\prod_{i=1}^{n}t_{ii}=(\det\mat{C})^2\det\mat{B}\]
    \[\det(\mat{S}+\mat{T})=\prod_{i=1}^{n}(s_{ii}+t_{ii})=(\det\mat{C})^2\det(\mat{A}+\det\mat{B})\]
    而
    \[\prod_{i=1}^{n}(s_{ii}+t_{ii})^2\geq\max\left\{\prod_{i=1}^{n}s_{ii},\prod_{i=1}^{n}t_{ii}\right\}\]
    并且$(\det\mat{C})^2>0$.于是
    \[\det(\mat{A}+\mat{B})\geq\max\{\det\mat{A},\det\mat{B}\}\]
    命题得证.
\end{proof}
\end{document}