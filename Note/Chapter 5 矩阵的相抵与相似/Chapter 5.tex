\documentclass{ctexart}
\usepackage{Note}
\begin{document}
\section{矩阵的相抵与相似}
\subsection{矩阵的相抵}
\begin{definition}[矩阵的相抵]
    对于$\K$上的$s\times n$矩阵$\mat{A}$和$\mat{B}$,如果$\mat{A}$经过一系列初等行列变换能变为$\mat{B}$,则称$\mat{A}$与$\mat{B}$是\tbf{相抵}的,记作$\mat{A}\sim\mat{B}$.
\end{definition}
考虑相抵的矩阵$\mat{A}$与$\mat{B}$,这意味着存在一系列$s$级初等矩阵$\mat{P}_1,\cdots,\mat{P}_t$和$n$级初等矩阵$\mat{Q}_1,\cdots,\mat{Q}_r$,使得
\[\mat{B}=\mat{P}_t\cdots\mat{P}_1\mat{A}\mat{Q}_1\cdots\mat{Q}_r\]
即存在$s$级可逆矩阵$\mat{P}$和$n$级可逆矩阵$\mat{Q}$,使得
\[\mat{B}=\mat{P}\mat{A}\mat{Q}\]
\begin{theorem}[相抵的判断条件I]
    $\K$上的$s\times n$矩阵$\mat{A}$和$\mat{B}$相抵当且仅当存在$s$级可逆矩阵$\mat{P}$和$n$级可逆矩阵$\mat{Q}$使得$\mat{B}=\mat{P}\mat{A}\mat{Q}$.
\end{theorem}
\begin{definition}[相抵标准型]
    设$\K$上的$s\times n$矩阵$\mat{A}$与矩阵
    \[\mat{H}_r=\begin{bmatrix}
        \mat{I}_r&\mbf{0}\\
        \mbf{0}&\mbf{0}
    \end{bmatrix}_{s\times n}\]
    相抵,其中$r=\rank\mat{A}$,则称$\mat{H}_r$为$\mat{A}$的\tbf{相抵标准型}.
\end{definition}
不难看出,任意矩阵的相抵标准型是唯一的.于是立即可以得到下面的推论:
\begin{theorem}[相抵的判断条件]
    $\K$上的$s\times n$矩阵$\mat{A}$和$\mat{B}$相抵当且仅当$\rank\mat{A}=\rank\mat{B}$.
\end{theorem}
于是可以得到下面的推论:
\begin{theorem}[矩阵的相抵标准型分解]
    设$\K$上的$s\times n$矩阵$\mat{A}$的秩为$r$,则存在$s$级可逆矩阵$\mat{P}$和$n$级可逆矩阵$\mat{Q}$,使得
    \[\mat{A}=\mat{P}\begin{bmatrix}
        \mat{I}_r&\mbf{0}\\
        \mbf{0}&\mbf{0}
    \end{bmatrix}\mat{Q}\]
\end{theorem}
\subsection{矩阵的相似}
\begin{definition}[矩阵的相似]
    设$\mat{A}$与$\mat{B}$都是$\K$上的$n$级矩阵,如果存在$\K$上的$n$级可逆矩阵$\mat{P}$,使得
    \[\mat{B}=\mat{P}^{-1}\mat{A}\mat{P}\]
    则称$\mat{A}$与$\mat{B}$是\tbf{相似}的,记作$\mat{A}\sim\mat{B}$.
\end{definition}
\subsection{特征值与特征向量}
\begin{definition}[特征值与特征向量]
    设$\mat{A}$是$\K$上的$n$级矩阵,如果$\K^n$中存在非零向量$\bs\alpha$使得
    \[\mat{A}\bs\alpha=\lambda\bs\alpha,\quad\lambda\in\K\]
    则称$\lambda$是$\mat{A}$的一个\tbf{特征值},称$\bs\alpha$是$\mat{A}$对应于特征值$\lambda$的一个\tbf{特征向量}.
\end{definition}
\begin{definition}[特征多项式]
    设$\mat{A}$是$\K$上的$n$级矩阵,则称$f(\lambda)=\det(\lambda\mat{I}-\mat{A})$为$\mat{A}$的\tbf{特征多项式}.
\end{definition}
\begin{theorem}
    设$\mat{A}$是$\K$上的$n$级矩阵,则$\lambda_0\in\K$是$\mat{A}$的特征值当且仅当$f(\lambda_0)=0$,其中$f$是$\mat{A}$的特征多项式.
\end{theorem}
\begin{proof}
    根据线性方程组解与其行列式的关系有
    \[\begin{aligned}
        \lambda\text{是}\mat{A}\text{的一个特征值}
        &\Leftrightarrow\exists\bs\alpha\in\K^n\text{且}\bs\alpha\neq\mbf0\text{使得}\mat{A}\bs\alpha=\lambda\bs\alpha\\
        &\Leftrightarrow\exists\bs\alpha\in\K^n\text{且}\bs\alpha\neq\mbf0\text{使得}\bs\alpha\text{是线性方程组}(\lambda\mat{I}-\mat{A})\vec{x}=\mbf0\text{的一个非零解}\\
        &\Leftrightarrow\text{线性方程组}(\lambda\mat{I}-\mat{A})\vec{x}=\mbf0\text{有非零解}\\
        &\Leftrightarrow\det(\lambda\mat{I}-\mat{A})=0\\
        &\Leftrightarrow \lambda\text{是特征多项式}f(\lambda)=\det(\lambda\mat{I}-\mat{A})\text{的一个根}
    \end{aligned}\]
\end{proof}
\begin{theorem}
    相似的矩阵具有相同的特征多项式.
\end{theorem}
\begin{proof}
    假定矩阵$\mat{A}\sim\mat{B}$,并且有可逆矩阵$\mat{P}$使得$\mat{P}^{-1}\mat{A}\mat{P}$,
\end{proof}
\subsection{矩阵可对角化的条件}
\begin{definition}[可对角化]
    如果$\K$上的$n$级矩阵$\mat{A}$与某个对角矩阵$\mat{D}$相似,则称$\mat{A}$是\tbf{可对角化}的.
\end{definition}
\begin{theorem}[矩阵可对角化的条件]
    $\K$上的$n$级矩阵$\mat{A}$可对角化的充分必要条件是$\mat{A}$有$n$个线性无关的特征向量$\li{\bs\alpha},n$,并且此时令
    \[\mat{P}=\begin{bmatrix}
        \bs\alpha_1&\cdots&\bs\alpha_n
    \end{bmatrix},\quad\mat{D}=\diag\{\lambda_1,\cdots,\lambda_n\}\]
    其中$\lambda_i$是$\bs\alpha_i$对应的特征值,则有$\mat{P}^{-1}\mat{A}\mat{P}=\mat{D}$.
\end{theorem}
\begin{proof}
    \[\begin{aligned}
        \mat{P}^{-1}\mat{A}\mat{P}=\mat{D}
        &\Leftrightarrow\mat{A}\mat{P}=\mat{P}\mat{D}\\
        &\Leftrightarrow\mat{A}\begin{bmatrix}
            \bs\alpha_1&\cdots&\bs\alpha_n
        \end{bmatrix}=\begin{bmatrix}
            \bs\alpha_1&\cdots&\bs\alpha_n
        \end{bmatrix}\mat{D}\\
        &\Leftrightarrow\begin{bmatrix}
            \mat{A}\bs\alpha_1&\cdots&\mat{A}\bs\alpha_n
        \end{bmatrix}=\begin{bmatrix}
            \lambda_1\bs\alpha_1&\cdots&\lambda_n\bs\alpha_n
        \end{bmatrix}\\
        &\Leftrightarrow\mat{A}\bs\alpha_i=\lambda_i\bs\alpha_i,\quad i=1,2,\cdots,n
    \end{aligned}\]
    并且由于$\mat{P}$是可逆矩阵,于是其列向量组$\li{\bs\alpha},n$线性无关.于是就证明了上述命题.
\end{proof}
\end{document}