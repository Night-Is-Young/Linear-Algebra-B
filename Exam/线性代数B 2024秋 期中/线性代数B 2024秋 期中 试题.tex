\documentclass{ctexart}
\usepackage{note}
\begin{document}\pagestyle{empty}
\begin{center}
    \tbf{\Large 北京大学数学科学学院2024-25学年第一学期线性代数B期中试题}
\end{center}
\begin{homework}[1]
    求下列行向量构成的向量组的秩和一个极大线性无关组:
    \[\bs\alpha_1=\begin{bmatrix}
        -1\\5\\3\\2
    \end{bmatrix},\quad\bs\alpha_2=\begin{bmatrix}
        4\\1\\-2\\9
    \end{bmatrix},\quad\bs\alpha_3=\begin{bmatrix}
        2\\0\\1\\4
    \end{bmatrix},\quad\bs\alpha_4=\begin{bmatrix}
        0\\3\\4\\-5
    \end{bmatrix}\]
\end{homework}
\begin{homework}[2]
    下述齐次线性方程组何时有非零解?何时只有零解?
    \[\left\{\begin{array}{l}
        x_1-3x_2-5x_3=0\\
        2x_1-7x_2-4x_3=0\\
        4x_1-9x_2+ax_3=0\\
        5x_1+bx_2-55x_3=0
    \end{array}\right.\]
\end{homework}
\begin{homework}[3]
    计算下面的行列式:
    \[A_n=\begin{vmatrix}
        1&1&1&\cdots&1\\
        1&2&0&\cdots&0\\
        1&0&3&\cdots&0\\
        \vdots&\vdots&\vdots&\ddots&\vdots\\
        1&0&0&\cdots&n
    \end{vmatrix}\]
\end{homework}
\begin{homework}[4]
    已知
    \[\begin{vmatrix}
        x&y&z&x+y+z\\
        3&0&2&0\\
        1&1&1&1\\
        2&2&2&1
    \end{vmatrix}=\dfrac12\]
    求
    \[\begin{vmatrix}
        x-y&y&z-x&x+y+z\\
        3&0&-1&0\\
        y-x&2-y&x-z&2-x-y-z\\
        3&2&-1&1
    \end{vmatrix}\]
\end{homework}
\begin{homework}[5]
    证明:
    \[\rank\begin{bmatrix}
        \mat{A}&\mbf{0}\\
        \mat{C}&\mat{B}
    \end{bmatrix}\geqslant\rank\ \mat{A}+\rank\ \mat{B}\]
\end{homework}
\begin{homework}[6]
    已知向量$\bs\beta$能由向量组$\li{\bs\alpha},s$线性表出,但不能由向量组$\li{\bs\alpha},{s-1}$线性表出.证明:
    \[\rank\left\{\li{\bs\alpha},s\right\}=\rank\left\{\li{\bs\alpha},{s-1},\bs\beta\right\}\]
\end{homework}
\begin{homework}[7]
    矩阵
    \[\begin{bmatrix}
        1&1&2&x\\
        3&2&1&y\\
        1&-1&0&z
    \end{bmatrix}\]
    的行向量组何时与矩阵
    \[\begin{bmatrix}
        3&1&5&14\\
        0&2&-1&11\\
        0&0&3&3
    \end{bmatrix}\]
    的行向量组等价?
\end{homework}
\begin{homework}[8]
    设$n$个方程的$n$元齐次线性方程组的系数矩阵$\mat{A}$的行列式等于$0$,并且$\mat{A}$的$(k,l)$元的代数余子式$A_{kl}\neq0$.试证明:
    \[\bs\eta=\begin{bmatrix}
        A_{k1}\\A_{k2}\\\vdots\\A_{kn}
    \end{bmatrix}\]
    是该齐次线性方程组的一个基础解系.
\end{homework}
\begin{homework}[9]
    已知方程组
    \[\left\{\begin{array}{c}
        a_{11}x_1+a_{12}x_2+\cdots+a_{1n}x_n=0\\
        a_{21}x_1+a_{22}x_2+\cdots+a_{2n}x_n=0\\
        \vdots\\
        a_{n1}x_1+a_{n2}x_2+\cdots+a_{nn}x_n=0
    \end{array}\right.\]
    的解都是方程
    \[b_1x_1+\cdots+b_nx_n=0\]
    的解.证明:$\bs\beta=\begin{bmatrix}
        b_1&b_2&\cdots&b_n
    \end{bmatrix}^{\text{t}}$可以由向量组$\li{\bs\alpha},n$线性表出,其中$\bs\alpha_i$是上述方程组系数矩阵$\mat{A}$的行向量.
\end{homework}
\end{document}