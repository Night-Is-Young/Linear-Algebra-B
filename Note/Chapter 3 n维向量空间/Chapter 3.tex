\documentclass{ctexart}
\usepackage{Note}
\begin{document}
\section{$n$维向量空间$K^n$}
\subsection{向量空间及其子空间}
\begin{definition}[$n$维向量空间]
    设$K$为数域,则所有$n$维向量组成的集合
    \[K^n=\left\{\begin{pmatrix}
        a_1&a_2&\cdots&a_n
    \end{pmatrix}\mid a_i\in K, i=1,2,\ldots,n\right\}\]
    称为$n$维向量空间.
\end{definition}
\begin{definition}[子空间]
    如果$U\subseteq K^n$满足
    \begin{enumerate}[label=\tbf{\arabic*}.,topsep=0pt,parsep=0pt,itemsep=0pt,partopsep=0pt]
        \item $\forall\boldsymbol{\alpha},\boldsymbol{\beta}\in U,\ \ \boldsymbol{\alpha}+\boldsymbol{\beta}\in U$.
        \item $\forall\boldsymbol{\alpha}\in U,k\in K,\ \ k\boldsymbol{\alpha}\in U$.
    \end{enumerate}
    则称$U$为$K^n$的一个子空间.
\end{definition}
\begin{definition}[张成空间]
    $K^n$中的向量组$\boldsymbol{\alpha}_1,\cdots,\boldsymbol{\alpha}_m$的所有线性组合构成的集合$W$是$K^n$的一个子空间,称为由$\boldsymbol{\alpha}_1,\cdots,\boldsymbol{\alpha}_m$张成的空间,记为
    \[W=\langle\boldsymbol{\alpha}_1,\boldsymbol{\alpha}_2,\cdots,\boldsymbol{\alpha}_m\rangle\]
\end{definition}
上述证明均略.
\subsection{线性相关与线性无关}
\begin{definition}[线性相关]
    称$K^n$中的向量组$\boldsymbol{\alpha}_1,\boldsymbol{\alpha}_2,\cdots,\boldsymbol{\alpha}_m$线性相关,如果存在不全为零的数$k_1,k_2,\cdots,k_m\in K$,使得
    \[k_1\boldsymbol{\alpha}_1+k_2\boldsymbol{\alpha}_2+\cdots+k_m\boldsymbol{\alpha}_m=\boldsymbol{0}\]
\end{definition}
\begin{definition}[线性无关]
    称$K^n$中的向量组$\boldsymbol{\alpha}_1,\boldsymbol{\alpha}_2,\cdots,\boldsymbol{\alpha}_m$线性无关,如果
    \[k_1\boldsymbol{\alpha}_1+k_2\boldsymbol{\alpha}_2+\cdots+k_m\boldsymbol{\alpha}_m=\boldsymbol{0}\]
    当且仅当$k_1=k_2=\cdots=k_m=0$.
\end{definition}
显然, $K^n$中的向量组要么线性相关,要么线性无关.
\subsection{极大线性无关组与向量组的秩}
\begin{definition}[极大线性无关组]
    设$K^n$中的向量组$\boldsymbol{\alpha}_1,\boldsymbol{\alpha}_2,\cdots,\boldsymbol{\alpha}_s$线性相关,则其中的一个线性无关子组,且任一向量加入该子组后都变成线性相关,称为该向量组的一个极大线性无关组.
\end{definition}
\begin{definition}[等价的向量组]
    设向量组$\mathcal{A}=\left\{\li{\bs\alpha},s\right\},\mathcal{B}=\left\{\li{\bs\beta},r\right\}$.如果$\mathcal{A}$中的每个向量都能由$\mathcal{B}$线性表出,且$\mathcal{B}$中的每个向量也都能由$\mathcal{A}$线性表出,则称向量组$\mathcal{A},\mathcal{B}$\tbf{等价},记为$\mathcal{A}\simeq\mathcal{B}$.
\end{definition}
\begin{theorem}
    向量组和它的极大线性无关组等价.
\end{theorem}
\begin{proof}
    考虑向量组$\mathcal{A}=\li{\bs\alpha},s$,假定它的一个极大线性无关组为$\mathcal{B}=\li{\bs\alpha},r(r\leqslant s)$.对于任意$1\leqslant i\leqslant s$有
    \[\bs\alpha_i=0\bs\alpha_1+\cdots+1\bs\alpha_i+\cdots+0\bs\alpha_s\]
    于是向量组$\mathcal{B}$的每个向量都能由$\mathcal{A}$线性表出.同样,对于任意$1\leqslant i\leqslant r$, $\mathcal{A}$中的$\bs\alpha_i$也能由$\mathcal{B}$线性表出.现在考虑$r<j\leqslant s$.根据极大线性无关组的定义,$\bs\alpha_j$总是能由$\mathcal{B}$线性表出.因此$\mathcal{A}$中的每个向量都能由$\mathcal{B}$线性表出.综上, $\mathcal{A}\simeq\mathcal{B}$.
\end{proof}
从上面的定理可以很容易地得出下面的推论.
\begin{lemma}
    向量组的任意两个极大线性无关组等价.
\end{lemma}
\begin{theorem}
    设向量组$\li{\bs\beta},r$可以由向量组$\li{\bs\alpha},s$线性表出.如果$r>s$,那么向量组$\li{\bs\beta},r$线性相关.
\end{theorem}
\begin{proof}
    考虑$\bs\beta_i$由$\li{\bs\alpha},s$线性表出的表达式:
    \[\bs\beta_i=a_{i1}\bs\alpha_1+\cdots+a_{is}\bs\alpha_s\]
    其中$1\leqslant i\leqslant r$.考虑$\li k,r$使得
    \[k_1\bs\beta_1+\cdots+k_r\bs\beta_r=\mbf0\]
    即
    \[\left(k_1a_{11}+\cdots+k_ra_{r1}\right)\bs\alpha_1+\cdots+\left(k_1a_{1s}+\cdots+k_ra_{rs}\right)\bs\alpha_s=\mbf0\]
    为使得上式成立,考虑齐次线性方程组
    \[\left\{\begin{array}{c}
        k_1a_{11}+k_2a_{21}+\cdots+k_ra_{r1}=0\\
        k_1a_{12}+k_2a_{22}+\cdots+k_ra_{r2}=0\\
        \vdots\\
        k_1a_{1s}+k_2a_{2s}+\cdots+k_ra_{rs}=0
    \end{array}\right.\]
    这是一个有$r$个未知数和$s$个方程的齐次线性方程组,由于$r>s$,所以它有非零解.取一组非零解,即可证得$\li{\bs\beta},r$线性相关.
\end{proof}
上述命题的逆否命题如下.
\begin{lemma}
    设向量组$\li{\bs\beta},r$可以由向量组$\li{\bs\alpha},s$线性表出.如果向量组$\li{\bs\beta},r$线性无关,那么$r\leqslant s$.
\end{lemma}
从上面的推论容易得到下面的定理.
\begin{theorem}
    向量组的两个极大线性无关组含有向量的数目相等.
\end{theorem}
这就引出了秩的概念.
\begin{definition}[秩]
    设$K^n$中的向量组$\boldsymbol{\alpha}_1,\boldsymbol{\alpha}_2,\cdots,\boldsymbol{\alpha}_s$的一个极大线性无关组含$r$个向量,则称$r$为该向量组的秩,记为$r(\boldsymbol{\alpha}_1,\boldsymbol{\alpha}_2,\cdots,\boldsymbol{\alpha}_s)$.\\
    此外,规定由零向量构成的向量组的秩为$0$.
\end{definition}
有关向量组的秩有一些重要的性质和定理.
\begin{theorem}
    向量组$\li{\bs\alpha},s$线性无关,当且仅当$\rank\left\{\li{\bs\alpha},s\right\}=s$.
\end{theorem}
\begin{proof}
    易得.
\end{proof}
\begin{theorem}
    设向量组$\li{\bs\beta},r$可以由向量组$\li{\bs\alpha},s$线性表出,则
    \[\rank\left\{\li{\bs\beta},r\right\}\leqslant\rank\left\{\li{\bs\alpha},s\right\}\]
\end{theorem}
\begin{proof}
    取它们的极大线性无关组$\li{\bs\beta},t$和$\li{\bs\alpha},m$.由前面的推论知$t\leqslant m$.所以
    \[\rank\left\{\li{\bs\beta},r\right\}= t\leqslant m=\rank\left\{\li{\bs\alpha},s\right\}\]
\end{proof}
\begin{theorem}
    设向量组$\mathcal{A},\mathcal{B}$满足$\rank\ \mathcal{A}=\rank\ \mathcal{B}$,且$\mathcal{A}$可以由$\mathcal{B}$线性表出,则$\mathcal{A}\simeq\mathcal{B}$.
\end{theorem}
\begin{proof}
    设$\li{\bs\alpha},r$和$\li{\bs\beta},r$分别为$\mathcal{A}$和$\mathcal{B}$的极大线性无关组.由线性表出的传递性可得$\li{\bs\alpha},r$可以由$\li{\bs\beta},r$线性表出.任取$1\leqslant i\leqslant r$,则向量组$\li{\bs\alpha},r,\bs\beta_i$可以由向量组$\li{\bs\beta},r$线性表出.根据前面的引理可知向量组$\li{\bs\alpha},r,\bs\beta_i$线性相关,并且由于$\li{\bs\alpha},r$线性无关,所以$\bs\beta_i$可以由$\li{\bs\alpha},r$线性表出.因此$\mathcal{B}$中的每个向量都能由$\mathcal{A}$线性表出.综上, $\mathcal{A}\simeq\mathcal{B}$.
\end{proof}
\end{document}