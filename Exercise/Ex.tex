\documentclass{ctexart}
\usepackage{Note}
\begin{document}
\section{线性方程组}
\section{行列式}
\begin{problem}
    设$n\geqslant 2$,$n$级矩阵$\mat{A}$的元素都是$1$或者$-1$.
    \begin{enumerate}[label=\tbf{\arabic*}.,topsep=0pt,parsep=0pt,itemsep=0pt,partopsep=0pt]
        \item $\mat{A}$的行列式$\det\mat{A}$一定是偶数.
        \item 在$n=3$的时候,求$\det\mat{A}$的最大值.
        \item 当$n\geqslant 3$时,证明:
            \[\left|\det\mat{A}\right|\leqslant(n-1)!(n-1)\]
    \end{enumerate}
\end{problem}
\begin{proof}\ 
    \begin{enumerate}[label=\tbf{\arabic*}.,topsep=0pt,parsep=0pt,itemsep=0pt,partopsep=0pt]
        \item 首先证明$\det\mat{A}$为偶数.\\
            \tbf{方法一}:考虑行列式的定义:
            \[\det\mat{A}=\sum_{j_1\cdots j_n}(-1)^{\tau\left(j_1\cdots j_n\right)}\prod_{i=1}^{n}a_{ij_{i}}\]
            由于$a_{ij_i}=\pm1$,于是
            \[(-1)^{\tau\left(j_1\cdots j_n\right)}\prod_{i=1}^{n}a_{ij_{i}}=\pm1\]
            又因为$j_1\cdots j_n$的排列数目共有$n!$种,当$n\geqslant2$时$n!$为偶数.于是$\det\mat{A}$是偶数个$1$或$-1$相加的结果,于是一定为偶数.\\
            \tbf{方法二}:采用数学归纳法.当$n=2$时,不难有
            \[\det\mat{A}=a_{11}a_{22}-a_{12}a_{21}\]
            由于$a_{ij}=\pm1$,于是$a_{11}a_{22}=\pm1,a_{12}a_{21}=\pm1$.无论何种情况,都有$\det\mat{A}=-2,0,2$,为偶数.\\
            当$n\geqslant3$时,将行列式按照第一行展开,有
            \[\det\mat{A}=\sum_{i=1}^{n}a_{1i}A_{1}\]
            余子式$A_{1i}$也是由$-1,1$组成的行列式,按照归纳假设,$A_{1i}$均为偶数.又因为$a_{1i}=\pm1$,于是$\det\mat{A}$为$n$个偶数相加减的结果,当然也是偶数.于是命题得证.
        \item 当$n=3$时,$\det\mat{A}$一共有$6$项,并且它们只能取$\pm1$.因此,$\det\mat{A}$能取到的最大值为$6$,并且此时
            \[a_{11}a_{22}a_{33}=a_{12}a_{23}a_{31}=a_{13}a_{21}a_{32}=1\]
            \[a_{11}a_{23}a_{32}=a_{12}a_{21}a_{33}=a_{13}a_{22}a_{31}=-1\]
            分别将各行的三项相乘可得
            \[\prod_{1\leqslant i,j\leqslant3}a_{ij}=1,\ \ \prod_{1\leqslant i,j\leqslant3}a_{ij}=-1\]
            这是矛盾的.因此$\det\mat{A}$无法取到$6$,又因为
            \[\begin{vmatrix}
                1&1&1\\-1&1&1\\1&-1&1
            \end{vmatrix}=4\]
            于是$\det\mat{A}$的最大值为$4$.
        \item 采用数学归纳法.当$n=3$时,有
            \[\det\mat{A}\leqslant4=(3-1)!(3-1)\]
            将$\det\mat{A}$按照第一行展开可得
            \[\det\mat{A}=\sum_{j=1}^{n}a_{1j}A_{1j}\leqslant\sum_{j=1}^{n}\left|a_{1j}\right|\left|A_{1j}\right|\leqslant n(n-2)!(n-2)<(n-2)!\left(n^2-2n+1\right)=(n-1)!(n-1)\]
            于是命题得证.
    \end{enumerate}
\end{proof}
\begin{problem}
    计算以下行列式的值.
    \begin{enumerate}[label=\tbf{\arabic*}.,topsep=0pt,parsep=0pt,itemsep=0pt,partopsep=0pt]
        \item \[\begin{vmatrix}
                a_1+b_1&a_1+b_2&\cdots&a_1+b_n\\
                a_2+b_1&a_2+b_2&\cdots&a_2+b_n\\
                \vdots&\vdots&\ddots&\vdots\\
                a_n+b_1&a_n+b_2&\cdots&a_n+b_n
            \end{vmatrix}\]
        \item \[\begin{vmatrix}
                a_{1n}+a_{11}&a_{11}+a_{12}&\cdots&a_{1(n-1)}+a_{1n}\\
                a_{2n}+a_{21}&a_{21}+a_{22}&\cdots&a_{2(n-1)}+a_{2n}\\
                \vdots&\vdots&\ddots&\vdots\\
                a_{nn}+a_{n1}&a_{n1}+a_{n2}&\cdots&a_{n(n-1)}+a_{nn}
            \end{vmatrix}\]
        \item \[\begin{vmatrix}
                1&2&3&\cdots&n-2&n-1&n\\
                2&3&4&\cdots&n-1&n&n\\
                3&4&5&\cdots&n&n&n\\
                \vdots&\vdots&\vdots&\ddots&\vdots&\vdots&\vdots\\
                n&n&n&\cdots&n&n&n
            \end{vmatrix}\]
    \end{enumerate}
\end{problem}
\begin{solution}
    \begin{enumerate}[label=\tbf{\arabic*}.,topsep=0pt,parsep=0pt,itemsep=0pt,partopsep=0pt]
        \item 将第一列减去第二列可得
            \[\begin{aligned}
                \begin{vmatrix}
                    a_1+b_1&a_1+b_2&\cdots&a_1+b_n\\
                    a_2+b_1&a_2+b_2&\cdots&a_2+b_n\\
                    \vdots&\vdots&\ddots&\vdots\\
                    a_n+b_1&a_n+b_2&\cdots&a_n+b_n
                \end{vmatrix}
                &=
                \begin{vmatrix}
                    b_1-b_2&a_1+b_2&\cdots&a_1+b_n\\
                    b_1-b_2&a_2+b_2&\cdots&a_2+b_n\\
                    \vdots&\vdots&\ddots&\vdots\\
                    b_1-b_2&a_n+b_2&\cdots&a_n+b_n
                \end{vmatrix}\\
                &=\left(b_1-b_2\right)
                \begin{vmatrix}
                    1&a_1+b_2&\cdots&a_1+b_n\\
                    1&a_2+b_2&\cdots&a_2+b_n\\
                    \vdots&\vdots&\ddots&\vdots\\
                    1&a_n+b_2&\cdots&a_n+b_n
                \end{vmatrix}
            \end{aligned}\]
            同样地,将第$2$列减去第$3$列,可以得到
            \[LHS=\left(b_1-b_2\right)\left(b_2-b_3\right)\begin{vmatrix}
                1&1&\cdots&a_1+b_n\\
                1&1&\cdots&a_2+b_n\\
                \vdots&\vdots&\ddots&\vdots\\
                1&1&\cdots&a_n+b_n
            \end{vmatrix}\]
            既然第一列与第二列相同,因此原行列式的值为$0$.
        \item 将行列式每列拆成两列的和.如果第一列保留$a_{i1}$,那么第二列只能保留$a_{i2}$(否则两列相同,行列式值为$0$).依次类推,可得第$j$列只能保留$a_{ij}$;如果第一列保留$a_{in}$,那么最后一列只能保留$a_{i(n-1)}$,依次类推,可得第$j$列只能保留$a_{i(j-1)}$.因此,原行列式可拆成两个行列式之和:
            \[\begin{vmatrix}
                a_{1n}+a_{11}&a_{11}+a_{12}&\cdots&a_{1(n-1)}+a_{1n}\\
                a_{2n}+a_{21}&a_{21}+a_{22}&\cdots&a_{2(n-1)}+a_{2n}\\
                \vdots&\vdots&\ddots&\vdots\\
                a_{nn}+a_{n1}&a_{n1}+a_{n2}&\cdots&a_{n(n-1)}+a_{nn}
            \end{vmatrix}=\begin{vmatrix}
                a_{11}&a_{12}&\cdots&a_{1n}\\
                a_{21}&a_{22}&\cdots&a_{2n}\\
                \vdots&\vdots&\ddots&\vdots\\
                a_{n1}&a_{n2}&\cdots&a_{nn}
            \end{vmatrix}+\begin{vmatrix}
                a_{1n}&a_{11}&\cdots&a_{1(n-1)}\\
                a_{2n}&a_{21}&\cdots&a_{2(n-1)}\\
                \vdots&\vdots&\ddots&\vdots\\
                a_{nn}&a_{n1}&\cdots&a_{n(n-1)}
            \end{vmatrix}\]
            将第二个行列式的第一列移到最后一列共需$n-1$次对换,因此
            \[\begin{vmatrix}
                a_{1n}+a_{11}&a_{11}+a_{12}&\cdots&a_{1(n-1)}+a_{1n}\\
                a_{2n}+a_{21}&a_{21}+a_{22}&\cdots&a_{2(n-1)}+a_{2n}\\
                \vdots&\vdots&\ddots&\vdots\\
                a_{nn}+a_{n1}&a_{n1}+a_{n2}&\cdots&a_{n(n-1)}+a_{nn}
            \end{vmatrix}=\left(1+(-1)^{n-1}\right)\begin{vmatrix}
                a_{11}&a_{12}&\cdots&a_{1n}\\
                a_{21}&a_{22}&\cdots&a_{2n}\\
                \vdots&\vdots&\ddots&\vdots\\
                a_{n1}&a_{n2}&\cdots&a_{nn}
            \end{vmatrix}\]
        \item \tbf{方法一}:将第$n$列减去第$n-1$列可得
            \[\begin{vmatrix}
                1&2&3&\cdots&n-2&n-1&n\\
                2&3&4&\cdots&n-1&n&n\\
                3&4&5&\cdots&n&n&n\\
                \vdots&\vdots&\vdots&\ddots&\vdots&\vdots&\vdots\\
                n&n&n&\cdots&n&n&n
            \end{vmatrix}=\begin{vmatrix}
                1&2&3&\cdots&n-2&n-1&1\\
                2&3&4&\cdots&n-1&n&0\\
                3&4&5&\cdots&n&n&0\\
                \vdots&\vdots&\vdots&\ddots&\vdots&\vdots&\vdots\\
                n&n&n&\cdots&n&n&0
            \end{vmatrix}\]
            按照最后一列展开,仍然把第$n-1$列减去第$n-2$列可得
            \[LHS=(-1)^{n+1}\begin{vmatrix}
                2&3&4&\cdots&n-1&n\\
                3&4&5&\cdots&n&n\\
                \vdots&\vdots&\vdots&\ddots&\vdots&\vdots\\
                n&n&n&\cdots&n&n
            \end{vmatrix}=(-1)^{n+1}\begin{vmatrix}
                2&3&4&\cdots&n-1&1\\
                3&4&5&\cdots&n&0\\
                \vdots&\vdots&\vdots&\ddots&\vdots&\vdots\\
                n&n&n&\cdots&n&0
            \end{vmatrix}\]
            重复上面的操作,可得
            \[LHS=(-1)^{(n+1)+n+\cdots+3}\left|n\right|=(-1)^{\frac{(n+4)(n-1)}{2}}n\]
            \tbf{方法二}:将第$j$列减去第$j-1$列,于是原行列式的右下部分全部为$0$.按照行列式的定义有
            \[LHS=(-1)^{\frac{n(n-1)}{2}}n\]
    \end{enumerate}
\end{solution}
\begin{problem}
    设$n\geqslant2$,求下面行列式的值:
    \[D_n=\begin{vmatrix}
        1&1&\cdots&1&1\\
        x_1&x_2&\cdots&x_{n-1}&x_n\\
        x_1^2&x_2^2&\cdots&x_{n-1}^2&x_n^2\\
        \vdots&\vdots&\ddots&\vdots&\vdots\\
        x_1^{n-2}&x_2^{n-2}&\cdots&x_{n-1}^{n-2}&x_n^{n-2}\\
        x_1^n&x_2^n&\cdots&x_{n-1}^n&x_n^n
    \end{vmatrix}\]
\end{problem}
\begin{solution}
    考虑$n+1$阶行列式
    \[\tilde{D}_{n+1}(y)=\begin{vmatrix}
        1&1&\cdots&1&1\\
        x_1&x_2&\cdots&x_n&y\\
        x_1^2&x_2^2&\cdots&x_n^2&y^2\\
        \vdots&\vdots&\ddots&\vdots&\vdots\\
        x_1^n&x_2^n&\cdots&x_n^n&y^n
    \end{vmatrix}\]
    这行列式的$(n,n+1)$元的余子式即为$D_n$,于是只需考虑$y^{n-1}$的系数即可.又$\tilde{D}_{n+1}$本身为范德蒙德行列式,于是
    \[\tilde{D}_{n+1}(y)=\left(y-x_1\right)\cdots\left(y-x_n\right)\prod_{1\leqslant i<j\leqslant n}\left(x_j-x_i\right)\]
    于是
    \[(-1)^{n+(n+1)}D_n=-\left(x_1+x_2+\cdots+x_n\right)\prod_{1\leqslant i<j\leqslant n}\left(x_j-x_i\right)\]
    从而
    \[D_n=\left(\sum_{i=1}^{n}x_i\right)\left(\prod_{1\leqslant i<j\leqslant n}\left(x_j-x_i\right)\right)\]
\end{solution}
\begin{problem}
    求下面行列式的值:
    \[D_n=\begin{vmatrix}
        1+x_1&1+x_2&\cdots&1+x_n\\
        1+x_1^2&1+x_2^2&\cdots&1+x_n^2\\
        \vdots&\vdots&\ddots&\vdots\\
        1+x_1^n&1+x_2^n&\cdots&1+x_n^n
    \end{vmatrix}\]
\end{problem}
\begin{solution}
    考虑行列式
    \[\tilde{D}_{n}=\begin{vmatrix}
        1&1&1&\cdots&1\\
        0&1+x_1&1+x_2&\cdots&1+x_n\\
        0&1+x_1^2&1+x_2^2&\cdots&1+x_n^2\\
        \vdots&\vdots&\vdots&\ddots&\vdots\\
        0&1+x_1^n&1+x_2^n&\cdots&1+x_n^n
    \end{vmatrix}\]
    将$\tilde{D}_n$按第一行展开即可得$\tilde{D}_n=D_n$.现在将$\tilde{D}_n$的第$j$行$(1<j\leqslant n+1)$减去第$1$行可得
    \[\tilde{D}_n=\begin{vmatrix}
        1&1&1&\cdots&1\\
        -1&x_1&x_2&\cdots&x_n\\
        -1&x_1^2&x_2^2&\cdots&x_n^2\\
        \vdots&\vdots&\vdots&\ddots&\vdots\\
        -1&x_1^n&x_2^n&\cdots&x_n^n
    \end{vmatrix}=\begin{vmatrix}
        2&1&1&\cdots&1\\
        0&x_1&x_2&\cdots&x_n\\
        0&x_1^2&x_2^2&\cdots&x_n^2\\
        \vdots&\vdots&\vdots&\ddots&\vdots\\
        0&x_1^n&x_2^n&\cdots&x_n^n
    \end{vmatrix}+\begin{vmatrix}
        -1&1&1&\cdots&1\\
        -1&x_1&x_2&\cdots&x_n\\
        -1&x_1^2&x_2^2&\cdots&x_n^2\\
        \vdots&\vdots&\vdots&\ddots&\vdots\\
        -1&x_1^n&x_2^n&\cdots&x_n^n
    \end{vmatrix}\]
    右边两个行列式都可以拆解成范德蒙德行列式的形式,于是有
    \[\begin{aligned}
        D_n=\tilde{D}_n
        &= 2\prod_{k=1}^{n}x_k\prod_{1\leqslant i<j\leqslant n}\left(x_j-x_i\right)-\prod_{k=1}^{n}\left(x_k-1\right)\prod_{1\leqslant i<j\leqslant n}\left(x_j-x_i\right) \\
        &= \prod_{1\leqslant i<j\leqslant n}\left(x_j-x_i\right)\left(2\prod_{k=1}^{n}x_k-\prod_{k=1}^{n}\left(x_k-1\right)\right)
    \end{aligned}\]
    \textcolor{blue}{如果遇到一行(列)中只有一个元素与其它的不同,就可以考虑将这个元素进行拆项.}
\end{solution}
\begin{problem}
    设$n\geqslant2$,$a_i\neq0,i=1,\cdots,n$.计算以下行列式的值:
    \[A_n=\begin{vmatrix}
        x_1-a_1&x_2&\cdots&x_n\\
        x_1&x_2-a_2&\cdots&x_n\\
        \vdots&\vdots&\ddots&\vdots\\
        x_1&x_2&\cdots&x_n-a_n
    \end{vmatrix}\]
\end{problem}
\begin{solution}
    考虑行列式
    \[\tilde{A}_n=\begin{vmatrix}
        1&x_1&x_2&\cdots&x_n\\
        0&x_1-a_1&x_2&\cdots&x_n\\
        0&x_1&x_2-a_2&\cdots&x_n\\
        \vdots&\vdots&\vdots&\ddots&\vdots\\
        0&x_1&x_2&\cdots&x_n-a_n
    \end{vmatrix}\]
    将$\tilde{A_n}$按第一列展开可得$\tilde{A_n}=A_n$.现在将$\tilde{A_n}$的第$i$行$(1<j\leqslant n+1)$减去第$1$行可得
    \[\tilde{A}_n=\begin{vmatrix}
        1&x_1&x_2&\cdots&x_n\\
        -1&-a_1&0&\cdots&0\\
        -1&0&-a_2&\cdots&0\\
        \vdots&\vdots&\vdots&\ddots&\vdots\\
        -1&0&0&\cdots&-a_n
    \end{vmatrix}=\begin{vmatrix}
        \displaystyle1-\sum_{k=1}^{n}\dfrac{x_k}{a_k}&x_1&x_2&\cdots&x_n\\
        0&-a_1&0&\cdots&0\\
        0&0&-a_2&\cdots&0\\
        \vdots&\vdots&\vdots&\ddots&\vdots\\
        0&0&0&\cdots&-a_n
    \end{vmatrix}\]
    这是一个上三角行列式,于是
    \[A_n=\tilde{A}_n=(-1)^{n}\left(1-\sum_{k=1}^{n}\dfrac{x_k}{a_k}\right)\left(\prod_{k=1}^{n}a_k\right)\]
\end{solution}
\begin{problem}
    求下面行列式的值:
    \[A_n=\begin{vmatrix}
        0&1&2&\cdots&n-2&n-1\\
        1&0&1&\cdots&n-3&n-2\\
        \vdots&\vdots&\vdots&\ddots&\vdots&\vdots\\
        n-2&n-3&n-4&\cdots&0&1\\
        n-1&n-2&n-3&\cdots&1&0
    \end{vmatrix}\]
\end{problem}
\begin{solution}
    将第$i$行减去第$i-1$行$(i>1)$可得
    \[A_n=\begin{vmatrix}
        0&1&\cdots&n-2&n-1\\
        1&-1&\cdots&-1&-1\\
        \vdots&\vdots&\ddots&\vdots&\vdots\\
        1&1&\cdots&-1&-1\\
        1&1&\cdots&1&-1
    \end{vmatrix}\]
    将第$i$行减去第$i-1$行$(i>2)$可得
    \[A_n=\begin{vmatrix}
        0&1&\cdots&n-2&n-1\\
        1&-1&\cdots&-1&-1\\
        0&2&\cdots&0&0\\
        \vdots&\vdots&\ddots&\vdots&\vdots\\
        0&0&\cdots&0&0\\
        0&0&\cdots&2&0
    \end{vmatrix}\]
    按照行列式的定义,只有第一行选择$n-1$,第二行选择$1$,后面所有行选择$2$才能使得求和的项不为$0$.于是
    \[A_n=(-1)^{n-1}(n-1)2^{n-2}\]
\end{solution}
\begin{problem}
    求下面行列式的值:
    \[A_n=\begin{vmatrix}
        1&2&\cdots&n-1&n\\
        2&3&\cdots&n&1\\
        3&4&\cdots&1&2\\
        \vdots&\vdots&\ddots&\vdots&\vdots\\
        n&1&\cdots&n-2&n-1
    \end{vmatrix}\]
\end{problem}
\begin{solution}
    将第$i$行减去第$i-1$行有
    \[A_n=\begin{vmatrix}
        1&2&\cdots&n-1&n\\
        1&1&\cdots&1&1-n\\
        1&1&\cdots&1-n&1\\
        \vdots&\vdots&\ddots&\vdots&\vdots\\
        1&1-n&\cdots&1&1
    \end{vmatrix}\]
    将第$i$列减去第$1$列有
    \[A_n=\]
\end{solution}
\end{document}