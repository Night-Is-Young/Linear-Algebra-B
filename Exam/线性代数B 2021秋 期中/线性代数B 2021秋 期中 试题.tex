\documentclass{ctexart}
\usepackage{note}
\begin{document}\pagestyle{empty}
\begin{center}
    \tbf{\Large 北京大学数学科学学院2021-22学年第一学期线性代数B期中试题}
\end{center}
\begin{homework}[1(20')]
    求$a$为何值时,下述线性方程组有\tbf{a.}唯一解; \tbf{b.}无解; \tbf{c.}无穷多解?在有无穷多解的情况下写出解集的结构.
    \[\left\{\begin{array}{l}
        x_1-ax_2-2x_3=-1\\
        x_1-x_2+ax_3=2\\
        5x_1-5x_2-4x_3=1
    \end{array}\right.\]
\end{homework}
\begin{homework}[2(10')]
    判断$\R^3$中下列子集是否为$\R^3$的子空间,并说明理由.
    \begin{enumerate}[label=\tbf{(\arabic*)},topsep=0pt,parsep=0pt,itemsep=0pt,partopsep=0pt]
        \item $\left\{\left(x_1,x_2,x_3\right)\in\R^3:x_1+2x_2+3x_3=0\right\}$.
        \item $\left\{\left(x_1,x_2,x_3\right)\in\R^3:x_1+2x_2+3x_3=4\right\}$.
        \item $\left\{\left(x_1,x_2,x_3\right)\in\R^3:x_1x_2x_3=0\right\}$.
        \item $\left\{\left(x_1,x_2,x_3\right)\in\R^3:\left(x_1+x_2\right)^2+\left(x_1+5x_3\right)^2=0\right\}$.
    \end{enumerate}
\end{homework}
\begin{homework}[3(10')]
    找出一个非零的$3\times3$矩阵$\mat{P}$使得$\mat{P}\mat{A}$为简化行阶梯形矩阵,其中
    \[\mat{A}=\begin{bmatrix}
        1&-2&-3\\
        0&2&2\\
        3&-2&0
    \end{bmatrix}\]
\end{homework}
\begin{homework}[4(20')]
    向量组$\bs\alpha_1,\bs\alpha_2,\bs\alpha_3,\bs\alpha_4$和线性无关的向量组$\bs\beta_1,\bs\beta_2,\bs\beta_3$满足如下关系:
    \[\left\{\begin{array}{l}
        \bs\beta_1-2\bs\beta_2-\bs\beta_3=\bs\alpha_1\\
        \bs\beta_1-2\bs\beta_2-\bs\beta_3=\bs\alpha_1\\
        5\bs\beta_1-3\bs\beta_2+9\bs\beta_3=\bs\alpha_3\\
        -2\bs\beta_1+\bs\beta_2-4\bs\beta_3=\bs\alpha_4\\
    \end{array}\right.\]
    求出所有满足$l_1\bs\alpha_1+l_2\bs\alpha_2+l_3\bs\alpha_3+l_4\bs\alpha_4$的向量$\begin{bmatrix}
        l_1&l_2&l_3&l_4
    \end{bmatrix}^{\text{t}}$.
\end{homework}
\begin{homework}[5(10')]
    设$\mat{E}_{i,i+1}(i=1,\cdots,n-1)$是$n\times n$的基本矩阵.证明:
    \begin{enumerate}[label=\tbf{(\arabic*)},topsep=0pt,parsep=0pt,itemsep=0pt,partopsep=0pt]
        \item 如果$\left|i-j\right|>1$,那么$\mat{E}_{i,i+1}\mat{E}_{j,j+1}=\mat{E}_{j,j+1}\mat{E}_{i,i+1}$.
        \item 如果$\left|i-j\right|=1$,那么$\mat{E}_{i,i+1}^2\mat{E}_{j,j+1}-2\mat{E}_{i,i+1}\mat{E}_{j,j+1}\mat{E}_{i,i+1}+\mat{E}_{i,i+1}\mat{E}_{j,j+1}^2=\mbf{0}$.
    \end{enumerate}
\end{homework}
\begin{homework}[6(10')]
    设$\mat{A}=\left(a_{ij}\right)_{1\leqslant i,h\leqslant n}$是$n$级方阵,$A_{ij}$是$a_{ij}$的代数余子式.证明:
    \[\begin{vmatrix}
        a_{11}+x&a_{12}+x&\cdots&a_{1n}+x\\
        a_{21}+x&a_{22}+x&\cdots&a_{2n}+x\\
        \vdots&\vdots&\ddots&\vdots\\
        a_{n1}+x&a_{n2}+x&\cdots&a_{nn}+x
    \end{vmatrix}=\det\mat{A}+x\sum_{1\leqslant i,j\leqslant n}A_{ij}\]
\end{homework}
\begin{homework}[7(10')]
    令$f(t)=\displaystyle\sum_{k=0}^{n-1}t^kx_k$.设$\zeta^0,\zeta^1,\cdots,\zeta^{n-1}\in\C$是所有$n$次单位根.证明:
    \[\begin{vmatrix}
        x_0&x_{n-1}&\cdots&x_1\\
        x_1&x_0&\cdots&x_2\\
        \vdots&\vdots&\ddots&\vdots
        x_{n-1}&x_{n-2}&\cdots&x_0
    \end{vmatrix}=\prod_{i=0}^{n-1}f\left(\zeta^{i}\right)\]
\end{homework}
\begin{homework}[8(10')]
    设$\mat{A},\mat{P}$均为$n$级方阵,矩阵$\mat{P}$为若干$\mat{P}(i,j)$型初等矩阵的成绩.令$\mat{B}=\mat{P}\mat{A}\mat{P}'$.判断$a_{ij}$在$\mat{A}$中的代数余子式$A_{ij}$是否等于$b_{ij}$在$\mat{B}$中的代数余子式$B_{ij}$,并说明理由.
\end{homework}
\end{document}