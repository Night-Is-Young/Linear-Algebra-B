\documentclass{ctexart}
\usepackage{note}
\begin{document}\pagestyle{empty}
\begin{center}
    \tbf{\Large 北京大学数学科学学院2022-23学年第二学期线性代数B期末试题}
\end{center}
\begin{homework}[1]
    判断二次型
    \[f(x_1,x_2,x_3)=4x_1^2+5x_2^2+6x_3^2+4x_1x_2-4x_2x_3\]
    是否正定,并说明理由.
\end{homework}
\begin{homework}[2]
    实二次型$f(x_1,x_2,x_3)$的矩阵为$\mat{A}$,其特征值为$1$(二重),$-1$.属于特征值$1$的特征向量是
    \[\bs\alpha_1=\begin{bmatrix}
        1\\1\\1
    \end{bmatrix},\quad\bs\alpha_2=\begin{bmatrix}
        2\\2\\1
    \end{bmatrix}\]
    \begin{enumerate}
        \item 求属于特征值$-1$的全部特征向量.
        \item 求$\mat{A}$.
    \end{enumerate}
\end{homework}
\begin{homework}[3]
    将二次型
    \[f(x,y,z)=x^2+2y^2+3z^2-4xy-4yz\]
    做正交替换化为标准形.
\end{homework}
\begin{homework}[4]
    设$\mathcal{A}:M_2(\K)\mapsto M_2(\K)$对任意$\mat{X}\in M_2(\K)$都有
    \[\mathcal{A}(\mat{X})=\mat{X}^\t\]
    \begin{enumerate}
        \item 证明: $\mathcal{A}$是线性变换.
        \item 求$\mathcal{A}$在基$\mat{E}_{11},\mat{E}_{12},\mat{E}_{21},\mat{E}_{22}$下的矩阵.
    \end{enumerate}
\end{homework}
\begin{homework}[5]
    证明:任意$\C$上的$n$级矩阵$\mat{A}$,总存在$\C$上的$n$级上三角矩阵$\mat{B}$使得$\mat{A}$与$\mat{B}$相似.
\end{homework}
\begin{homework}[6]
    证明:所有二级正交矩阵均可表示为
    \[\begin{bmatrix}
        \cos\theta&-\sin\theta\\\sin\theta&\cos\theta
    \end{bmatrix}\quad\text{或}\quad\begin{bmatrix}
        \cos\theta&\sin\theta\\\sin\theta&-\cos\theta
    \end{bmatrix}\]
    中的一种,其中$\theta\in\R$.
\end{homework}
\begin{homework}[7]
    
\end{homework}
\end{document}