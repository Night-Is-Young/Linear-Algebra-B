\documentclass{ctexart}
\usepackage{Note}
\begin{document}
\section{$n$维向量空间$K^n$}
\subsection{向量空间及其子空间}
\begin{definition}[$n$维向量空间]
    设$K$为数域,则所有$n$维向量组成的集合
    \[K^n=\left\{\begin{pmatrix}
        a_1&a_2&\cdots&a_n
    \end{pmatrix}\mid a_i\in K, i=1,2,\ldots,n\right\}\]
    称为$n$维向量空间.
\end{definition}
\begin{definition}[子空间]
    如果$U\subseteq K^n$满足
    \begin{enumerate}[label=\tbf{\arabic*}.,topsep=0pt,parsep=0pt,itemsep=0pt,partopsep=0pt]
        \item $\forall\boldsymbol{\alpha},\boldsymbol{\beta}\in U,\ \ \boldsymbol{\alpha}+\boldsymbol{\beta}\in U$.
        \item $\forall\boldsymbol{\alpha}\in U,k\in K,\ \ k\boldsymbol{\alpha}\in U$.
    \end{enumerate}
    则称$U$为$K^n$的一个子空间.
\end{definition}
\begin{definition}[张成空间]
    $K^n$中的向量组$\boldsymbol{\alpha}_1,\cdots,\boldsymbol{\alpha}_m$的所有线性组合构成的集合$W$是$K^n$的一个子空间,称为由$\boldsymbol{\alpha}_1,\cdots,\boldsymbol{\alpha}_m$张成的空间,记为
    \[W=\langle\boldsymbol{\alpha}_1,\boldsymbol{\alpha}_2,\cdots,\boldsymbol{\alpha}_m\rangle\]
\end{definition}
上述证明均略.
\subsection{线性相关与线性无关}
\begin{definition}[线性相关]
    称$K^n$中的向量组$\boldsymbol{\alpha}_1,\boldsymbol{\alpha}_2,\cdots,\boldsymbol{\alpha}_m$线性相关,如果存在不全为零的数$k_1,k_2,\cdots,k_m\in K$,使得
    \[k_1\boldsymbol{\alpha}_1+k_2\boldsymbol{\alpha}_2+\cdots+k_m\boldsymbol{\alpha}_m=\boldsymbol{0}\]
\end{definition}
\begin{definition}[线性无关]
    称$K^n$中的向量组$\boldsymbol{\alpha}_1,\boldsymbol{\alpha}_2,\cdots,\boldsymbol{\alpha}_m$线性无关,如果
    \[k_1\boldsymbol{\alpha}_1+k_2\boldsymbol{\alpha}_2+\cdots+k_m\boldsymbol{\alpha}_m=\boldsymbol{0}\]
    当且仅当$k_1=k_2=\cdots=k_m=0$.
\end{definition}
显然,$K^n$中的向量组要么线性相关,要么线性无关.
\subsection{极大线性无关组与向量组的秩}
\begin{definition}[极大线性无关组]
    设$K^n$中的向量组$\boldsymbol{\alpha}_1,\boldsymbol{\alpha}_2,\cdots,\boldsymbol{\alpha}_m$线性相关,则其中的一个线性无关子组,且任一向量加入该子组后都变成线性相关,称为该向量组的一个极大线性无关组.
\end{definition}
\begin{definition}[秩]
    设$K^n$中的向量组$\boldsymbol{\alpha}_1,\boldsymbol{\alpha}_2,\cdots,\boldsymbol{\alpha}_m$的一个极大线性无关组含$r$个向量,则称$r$为该向量组的秩,记为$r(\boldsymbol{\alpha}_1,\boldsymbol{\alpha}_2,\cdots,\boldsymbol{\alpha}_m)$.
\end{definition}
\end{document}